\documentclass[review, 10pt]{elsarticle}
% \documentclass[preprint, 10pt]{elsarticle}

\usepackage{amsmath,amsthm,amsfonts,amssymb,mathtools}
\usepackage{graphicx} 
\usepackage[T1]{fontenc}
\usepackage[hmarginratio=1:1,top=32mm,columnsep=20pt]{geometry} % Document margins
\usepackage{multicol} % Used for the two-column layout of the document
%%! Additional usepackage
\usepackage{physics}
\usepackage[hidelinks]{hyperref}
% \usepackage{hyperref} 
% \hypersetup{
	%     colorlinks=true,
	%     linkcolor=blue,
	%     filecolor=magenta,       
	%     urlcolor=cyan, 
	%     } 
\usepackage{subcaption}
\usepackage{booktabs}
\usepackage{float}
\usepackage[labelfont=bf]{caption} % Custom captions under/above floats in tables or figures

\usepackage{algorithm}% http://ctan.org/pkg/algorithms
\usepackage{algorithmicx} 
\usepackage{algpseudocode}% http://ctan.org/pkg/algorithmicx
\renewcommand{\algorithmicrequire}{\textbf{Input:}}
\renewcommand{\algorithmicensure}{\textbf{Output:}}

\usepackage{csquotes}
\usepackage{enumitem}
\usepackage{lineno}

\usepackage{xcolor}
\usepackage[draft]{changes}
\setauthormarkup{} % use empty author markup
\setdeletedmarkup{} % use empty deleted markup
\definecolor{harryModColor}{rgb}{0.8, 0, 0.3}
\definecolor{r1color}{rgb}{0.8, 0.1, 0.1}
\definecolor{r2color}{rgb}{0.1, 0.1, 0.8}
\definechangesauthor[name={Hanyu Zhou},color=harryModColor]{harry}
\definechangesauthor[name={Reviewer 1},color=r1color]{r1}
\definechangesauthor[name={Reviewer 2},color=r2color]{r2}
\definechangesauthor[name={Reviewer 1\&2},color=orange]{r12}
\usepackage{siunitx}

\journal{Journal of Computational Physics}

\begin{document}
	
\begin{frontmatter}
	
	%% Title, authors and addresses
	
	%% use the tnoteref command within \title for footnotes;
	%% use the tnotetext command for theassociated footnote;
	%% use the fnref command within \author or \address for footnotes;
	%% use the fntext command for theassociated footnote;
	%% use the corref command within \author for corresponding author footnotes;
	%% use the cortext command for theassociated footnote;
	%% use the ead command for the email address,
	%% and the form \ead[url] for the home page:
	%% \title{Title\tnoteref{label1}}
	%% \tnotetext[label1]{}
	%% \author{Name\corref{cor1}\fnref{label2}}
	%% \ead{email address}
	%% \ead[url]{home page}
	%% \fntext[label2]{}
	%% \cortext[cor1]{}
	%% \address{Address\fnref{label3}}
	%% \fntext[label3]{}

\title{Positivity-preserving implicit finite volume methods on unstructured grids for compressible flows}
% \title{Positivity-preserving algorithm for implicit finite volume methods simulating compressible flows}
% \title{Implicit positivity-preserving finite volume schemes on unstructured grids for compressible Euler and Navier-Stokes equations}

%% use optional labels to link authors explicitly to addresses:
%% \author[label1,label2]{}
%% \address[label1]{}
%% \address[label2]{}

\author[THU]{Hanyu Zhou}
\ead{zhy22@mails.tsinghua.edu.cn}
\author[CSRC]{Qian Wang\corref{cor1}}
\cortext[cor1]{Corresponding author.}
\ead{qian.wang@csrc.ac.cn}

\address[THU]{Department of Engineering Mechanics, Tsinghua University, Beijing 100084, China}
\address[CSRC]{Mechanics Division, Beijing Computational Science Research Center, Beijing 100193, China}
	
	
\begin{abstract}
    %% Text of abstract
    \hspace{10pt}  
    A frequently encountered issue in numerical simulations of complex compressible flows is the failure to preserve positivity of density or pressure, which leads to nonphysical solutions or numerical instabilities. It is a significant challenge to develop efficient and accurate positivity-preserving algorithms for numerical schemes with implicit time integration, where solutions are updated by solving nonlinear equations iteratively. To address this challenge, we propose a positivity-preserving algorithm for finite volume schemes with implicit time integration on unstructured grids. 
    \replaced[id=r2]
    {
    In this algorithm, admissible cell-averages are obtained by limiting time step sizes to control solution variations, 
    while increment correction is introduced to directly enforce positivity 
    during iteration.
    }
    {
    In this algorithm, admissible cell-averages are obtained by limiting time step sizes to control solution variations.
    }
    \replaced[id=r2]{
    To overcome the difficulty of unknown solution changes in time step limiting, we employ a simple linear approximation to estimate the future state, based on the available residual of the semi-discrete finite volume scheme.}
    {To overcome the difficulty of unknown solution changes in time step limiting, we employ explicit time discretization to efficiently estimate future states.} Allowable time step sizes are determined by constraining the relative solution changes to ensure the positivity of the updated solutions. Given positive cell-averages, admissible reconstruction polynomials are obtained by using a scaling limiter. Importantly, the proposed positivity-preserving algorithm is also accuracy-preserving. This algorithm is applied to a high-order variational finite volume method with explicit first stage singly diagonally implicit Runge-Kutta (ESDIRK) time integration. Numerical results for a series of benchmark test cases demonstrate the high accuracy, high resolution,
    \added[id=r2]{
        high efficiency
    }
    and robustness of the positivity-preserving implicit high-order finite volume method. 
\end{abstract}
	
\begin{keyword}
    %% keywords here, in the form: keyword \sep keyword
    Positivity-preserving algorithm \sep Finite volume method \sep Implicit time stepping \sep Accuracy-preserving property \sep Compressible flows
    %% PACS codes here, in the form: \PACS code \sep code
    
    %% MSC codes here, in the form: \MSC code \sep code
    %% or \MSC[2008] code \sep code (2000 is the default)
    
\end{keyword}
	
\end{frontmatter}

\linenumbers
% !TeX root = main.tex


\section{Introduction}
\label{sec:intro}

Computational fluid dynamics (CFD) has become a powerful tool for investigating flow problems of scientific or industrial significance, driven by advancements in computer hardware and the development of numerical methods such as finite difference (FD), finite volume (FV), finite element (FE), and spectral methods.
Second-order FV method on unstructured grids \cite{van1979towards}
%\cite{van1979towards,jameson1981numerical,aftosmis1995behavior,hubbard1999multidimensional,haselbacher2000accurate,diskin2010comparison}
is widely used in commercial and open-source CFD codes, thus being the workhorse of engineering
flow simulations. Over recent decades, various high-order methods on unstructured grids have been developed,
such as the high-order FV \cite{ollivier1997quasi_ENO,dumbser2007quadrature_WENO,wang2017compact_VR,nishikawa2023efficient},
discontinuous Galerkin (DG) \cite{reed1973triangularDG,cockburn2001rungeDG},
hybrid FV/DG \cite{dumbser2008unified,li2022reconstructed,zhang2012class1},
residual distribution (RD)  \cite{abgrall2003construction} and
flux reconstruction (FR) \cite{huynh2007flux,vincent2011new,wang2009unifying}.
Such methods, compared with their second-order counterparts,
have higher-order accuracy and lower dissipation/dispersion,
while maintaining the capability of handling complex geometries.
However, numerical methods, particularly high-order ones, often experience reduced robustness when applied to complex flow problems, even on high-quality computational meshes.
A frequently encountered robustness issue in compressible flow simulations is the failure to
preserve positivity of density or pressure, which leads to nonphysical solutions or numerical instabilities.

Tremendous efforts have been made on developing positivity-preserving algorithms to enhance the robustness of numerical methods,
especially high-order methods, for compressible flow simulations.
\replaced[id=r1]{For numerical methods employing explicit time integration, positivity-preserving algorithms generally fall into three main categories.}{For numerical methods employing explicit time integration, commonly used positivity-preserving algorithms can be broadly classified into two categories \cite{xu2017bound}.}
The first category includes positivity-preserving scaling limiters \cite{zhang2010maximum,zhang2010positivity,wang2012robust,cheng2014positivity,du2022high},
which compress the solution polynomials to obtain positive density and pressure distributions,
while being accuracy-preserving.
It is proved that, by using the scaling limiter,
a monotone numerical flux and a suitable CFL condition,
the FV and DG methods using strong stability preserving (SSP) Runge-Kutta
time integrations are positivity-preserving \cite{zhang2010positivity}.
\replaced[id=r1]{This approach has also been extended to FD \cite{zhang2012positivity,fan2022positivity} and FR methods \cite{huang2023high,dzanic2023positivity}.}{This approach has also been extended to FD methods \cite{zhang2012positivity,fan2022positivity}.}
The second category encompasses positivity-preserving flux limiters \cite{hu2013positivity,christlieb2015high,kuzmin2022bound},
which construct a convex combination of the first-order monotone flux and the high-order flux to ensure positivity under certain conditions. Compared to scaling limiters,
flux limiters offer the advantage of straightforward applicability to high-order FD methods for conservation laws and to high-order schemes
for convection-diffusion problems \cite{xu2017bound}.
However, the accuracy-preserving property of the flux limiter is often demonstrated
by extensive numerical results \cite{xiong2013parametrized},
as it is difficult to be proved for general cases \cite{xu2017bound}.
\added[id=r1]{The third category employs the \textit{a posteriori} Multi-dimensional Optimal Order Detection (MOOD) strategy \cite{clain2011high,diot2013multidimensional}. This method identifies problematic regions after each solution update and reduces the polynomial degree locally before recomputing the solution. In extreme cases, the polynomial degree can be reduced to zero near discontinuities, resulting in a stable, locally first-order finite volume scheme, while maintaining high-order accuracy in smooth regions. It has been proven \cite{zhang2010maximum,zhang2010positivity} that a finite volume method with zero-degree polynomials preserves positivity when combined with monotone numerical fluxes, SSP Runge-Kutta time integration, and appropriate CFL conditions.
In addition to these three primary categories, several alternative approaches have also been explored. Chan et al. \cite{chan2021positivity} and Gallice et al. \cite{gallice2022entropy} designed positivity-preserving first-order finite volume schemes through the construction of specialized approximate Riemann solvers, which can be extended to high-order accuracy by using a high-order polynomial reconstruction and a MOOD limiting strategy. Bourgeois and Lee \cite{bourgeois2022gp} proposed a positivity-preserving finite volume method using polynomial-free Gaussian process reconstruction to achieve high-order accuracy, supplemented by the MOOD limiting strategy for positivity enforcement. Dzanic and Witherden \cite{dzanic2022positivity} proposed a positivity-preserving entropy-based adaptive filtering technique for shock capturing in discontinuous spectral element methods, which enforces positivity by adapting the filter strength. Upperman and Yamaleev \cite{upperman2022positivity} developed a first-order, positivity-preserving and entropy-stable finite volume scheme for the compressible Navier-Stokes equations, which introduces artificial dissipation and time step restriction to guarantee the positivity of density and internal energy. This scheme was later extended to high-order accuracy by employing a flux limiter \cite{yamaleev2022positivity}.}

Explicit time stepping methods suffer from the CFL constraint that restricts the time step sizes.
For simulations with very small allowable time step sizes, such as the high Reynolds number turbulent flow simulations with large aspect ratio grids in the near-wall region, implicit time stepping methods are significantly more efficient.
Although most of the effort has been made for increasing accuracy of the time discretization and for increasing the efficiency of the nonlinear solver, only a few works exist in the literature concerning the positivity-preserving property of implicit methods \cite{qin2018implicit}, as it is difficult to verify if an implicit numerical scheme is positivity-preserving, even for a low-order one \cite{huang2024general}. This difficulty comes from the fact that, in implicit time stepping, the solutions are updated by solving a system of nonlinear equations iteratively. Batten et al. \cite{batten1997average} developed a positive FD scheme employing the Patankar trick \cite{patankar2018numerical} for compressible turbulent flows. Moryossef and Levy \cite{moryossef2006unconditionally,mor2009unconditionally} developed implicit unconditional positive FV schemes for unsteady turbulent flows by constructing $M$-matrix Jacobians. These methods are low-order accurate and are complicated to generalize to high order \cite{qin2018implicit}. Parent \cite{parent2018positivity} developed a positivity-preserving dual-time stepping scheme for high-resolution FV methods on structured grids solving Euler equations based on a Cauchy-Kowalevski procedure. Lian et al. \cite{lian2009solution} proposed a solution-limited time stepping scheme for FV methods on unstructured grids to enhance the reliability in steady-state compressible flow simulations. \added[id=r1]{Cossart et al. \cite{cossart2025toward} designed robust linearized implicit finite volume schemes on unstructured grids for steady-state hypersonic flow simulations, based on linear stability analysis.} Qin et al. \cite{qin2018implicit} developed an implicit high-order positivity-preserving DG method for steady-state compressible Euler equations. Recently, Huang et al. \cite{huang2024general} proposed a positivity-preserving algorithm based on an iterative flux correction procedure, for finite volume methods simulating compressible flows on unstructured grids with a second-order backward difference (BDF2) dual-time stepping.
%Based on the literature survey, it is found that it is of great interest to develop positivity-preserving algorithms for general high-order numerical methods on unstructured grids for unsteady compressible flow simulations.

\replaced[id=r2]{In this paper, we develop a positivity-preserving algorithm for arbitrary high-order finite volume schemes on unstructured grids with implicit dual-time stepping, towards robust and efficient simulations of unsteady compressible flows. 
In finite volume methods, the physical solution within a control volume is typically approximated by a reconstructed polynomial. Given admissible cell averages, the positivity-preserving scaling limiter \cite{zhang2010positivity} can be applied to obtain admissible reconstruction polynomials. Thus, the central challenge in designing positivity-preserving implicit finite volume schemes is to ensure that cell averages remain positive throughout implicit time integration.
% In this work, we focus on finite volume methods using dual-time stepping, where the solution at the next time level is obtained by iteratively solving nonlinear implicit equations, forming inner iterations. In other words, the solution at the next time level is obtained by advancing the solution in the pseudo-time direction until convergence. Therefore, both the converged and all intermediate states of the iteration process need to be guaranteed admissible to obtain admissible updated cell averages. 
We focus on finite volume methods employing dual-time stepping, where the solution at each physical time level is obtained by iteratively solving nonlinear implicit equations, effectively advancing the solution in the pseudo-time direction until convergence. Consequently, it is essential to guarantee the admissibility of both the converged and all intermediate states during the iteration process, so that the updated cell averages remain physically meaningful.
A key difficulty in ensuring the positivity of the converged state is that it is unknown prior to convergence. To address this, we introduce a time step limiting procedure. In this approach, we estimate the future state using a simple linear approximation based on the current residual. The estimated solution increment is proportional to the time step size, while the estimation error is of second order in the time step size. The allowable time step is then determined by imposing a lower bound on the estimated state, with the bound chosen to exceed the estimation error, thereby ensuring positivity of the future state.
In this work, the lower bound is set as a fraction of the current cell average, effectively restricting the relative change of the solution at each time step. 
The time step limiting procedure can be also applied in the pseudo-time direction. 
However, since each pseudo-time step typically uses only a single iteration to solve the linearized system, the solution increment is inexact. Therefore, an increment correction is applied to further restrict the relative change and ensure the admissibility of intermediate states throughout the inner iterations.
}
{In this paper, we develop a positivity-preserving algorithm for arbitrary high-order finite volume schemes on unstructured grids with implicit time integration, towards robust and efficient simulations of unsteady compressible flows.
In the positivity-preserving algorithm, admissible cell averages are obtained by limiting time step sizes to control solution changes. To overcome the difficulty of unknown solution changes in time step limiting, we employ explicit time discretization to efficiently estimate future states. 
The allowable time step sizes are determined by constraining the relative solution changes, ensuring the positivity of the updated solutions. At each stage of the implicit Runge-Kutta time integration, the solution is obtained by iteratively solving an implicit nonlinear equation system using the dual-time stepping technique, which forms an inner iteration process in the pseudo-time direction. Both physical and pseudo-time step size limitations are applied to ensure the admissibility of converged and intermediate solutions during the inner iteration process, respectively.
Given positive cell averages, admissible reconstruction polynomials can be obtained by applying a positivity-preserving scaling limiter \cite{zhang2010positivity}.} 

It is well-established that limiting time step sizes does not compromise the order of accuracy. Furthermore, as previously mentioned, the scaling limiter has been proven to preserve accuracy. Consequently, the proposed positivity-preserving algorithm is inherently accuracy-preserving. 
The positivity-preserving algorithm is applied to a high-order variational finite volume method \cite{wang2017compact_VR} with an explicit first stage singly diagonally implicit Runge-Kutta (ESDIRK) time integration \cite{bijl2002implicitBDFvESDIRK}, and verified using a series of benchmark test cases. Numerical results demonstrate the high accuracy, high resolution and robustness of the implicit positivity-preserving high-order finite volume method.
\added[id=r1]{Specifically, accuracy test results demonstrate that the positivity-preserving implicit finite volume method can achieve the desired order of accuracy, and a high-order method is significantly more efficient that its second-order counterpart for applications demanding high accuracy.}
\added[id=r2]{The numerical results for the hypersonic flow past an open cavity demonstrate the significant efficiency advantage of the implicit positivity-preserving finite volume scheme over its explicit counterpart.}

The remainder of this paper is organized as follows. Section \ref{sec:CFV} presents the high-order finite volume method on unstructured grids using implicit time stepping. Section \ref{sec:PP} presents the positivity-preserving algorithm for implicit finite volume methods. \added[id=r2]{Section \ref{sec:analysis-of-pp} presents the analysis of the proposed positivity-preserving algorithm.}
Section \ref{sec:Results} presents the numerical results for benchmark test cases. Section \ref{sec:Conclusions} gives concluding remarks.
% !TeX root = main.tex

\section{Implicit finite volume method}
\label{sec:CFV}

\newcommand{\trans}{^\mathrm{T}}
\newcommand{\U}{\mathbf{U}}
\newcommand{\Ulim}{\widetilde{\mathbf{U}}}
\renewcommand{\F}{\mathbf{F}}
\newcommand{\x}{\mathbf{x}}
\newcommand{\OO}{\mathbf{\Omega}}
\newcommand{\UM}{\overline{\U}}
\newcommand{\Fn}{\tilde{\F}}
\newcommand{\n}{\mathbf{n}}
\newcommand{\uu}{\overline{\mathbf{U}}}
\newcommand{\R}{\mathbf{R}}
\newcommand{\inc}{\mathrm\Delta}
\newcommand{\Tau}{\mathrm{T}}
\renewcommand{\real}{\mathrm{Re}}
\newcommand{\imag}{\mathrm{Im}}

\newcommand{\CFLt}{\text{CFL}_t}
\newcommand{\CFLtau}{\text{CFL}_\tau}
\newcommand{\CFL}{\text{CFL}}
\newcommand{\eeqref}[1]{Eq.\eqref{#1}}
\renewcommand{\um}{\overline{u}}
\renewcommand{\us}{\mathbf{u}}
\newcommand{\SAll}{\mathcal{S}}

\newcommand{\FF}{\mathcal{F}}

\newcommand{\eye}{\mathbf{I}}
\newcommand{\uv}{\mathbf{u}}

\newcommand{\supsm}{^{\left(s,m\right)}}
% \newcommand{\supsmp1}{^{\left(s,m+1\right)}}
\newcommand{\supsmPOne}{^{\left(s,m\right)}}

\subsection{Governing equations}
\label{ssec:GovEq}

The Navier-Stokes equations describing compressible viscous flows can be expressed as
\begin{equation}
    \label{eq:NS}
    \dfrac{\partial \U}{\partial t} +
    \nabla \cdot (\F - \F_v)= 0,
\end{equation}
where $\U$ is the conservative variable vector, $\F$ is the inviscid flux tensor and $\F_v$
is the viscous flux tensor defined by
\begin{equation}
\label{eq:def-U-F-Fv}
    \U = \begin{pmatrix}
        \rho \\ \rho \mathbf{u} \\ \rho E
    \end{pmatrix},\ \
    \F= \begin{pmatrix}
        \rho \mathbf{u}                                \\
        \rho \mathbf{u}\otimes \mathbf{u}+p \mathbf{I} \\
        (\rho E+p)\mathbf{u}                           \\
    \end{pmatrix},\ \
    \F_{v} = \begin{pmatrix}
        0                                                    \\
        \boldsymbol{\tau}                                    \\
        \boldsymbol{\tau} \cdot \mathbf{u} + \kappa \nabla T \\
    \end{pmatrix}.
\end{equation}
Here $\rho$ is the density, $\mathbf{u}$ is the velocity, $p$ is the pressure and $T$ is the temperature of the fluid. $\mathbf{I}$ is the identity tensor. $E$ is the total energy defined as
\begin{equation}
    E = \dfrac{1}{\gamma-1} \dfrac{p}{\rho} + \frac{1}{2} \mathbf{u} \cdot \mathbf{u},
\end{equation}
where $\gamma$ is the ratio of specific heat. The shear stress tensor $\boldsymbol{\tau}$ is defined as
\begin{equation}
    \boldsymbol{\tau}= \mu \left[ \nabla \mathbf{u} + \left(\nabla \mathbf{u}\right)\trans - \frac{2}{3} \left(\nabla \cdot \mathbf{u}\right) \mathbf{I} \right],
\end{equation}
\replaced[id=r1]{with $\mu$ being the dynamic viscosity computed by using Sutherland's law \cite{white2006viscous}}{with $\mu$ being the dynamic viscosity}. The heat conductivity is computed by $\kappa=C_p \mu /Pr$, where $C_p$ is the specific heat at constant pressure and $Pr$ is the Prandtl number. Equation \eqref{eq:NS} is closed by an equation of state
\begin{equation}
\label{eq:eos}
    p= \rho R T,
\end{equation}
where $R= \left(\gamma-1\right) C_p/\gamma$.
Neglecting viscous effects, i.e., $\mu=0$, the Navier-Stokes equations reduce to the Euler equations describing compressible inviscid flows.

\subsection{Semi-discrete finite volume scheme}
\label{ssec:FV}

This subsection presents the general framework of high-order cell-centered finite volume method on unstructured grids for compressible flows.
The computational domain $\OO$ is partitioned into $N$ non-overlapping control volumes, i.e., $\OO= \cup^N_{i=1}{\OO_i}$. \added[id=r1]{A control volume \(\OO_i\) can be a triangle or quadrilateral in a two-dimensional mesh, or a tetrahedron, prism, hexahedron, or pyramid in a three-dimensional mesh, as shown in Figure \ref{fig:controlvolume}.} 
\begin{figure}[htbp!]
    \centering
    \includegraphics[width=0.8 \linewidth]{pics/control_volume.pdf}
    \caption{\replaced[id=r1]{Two- and three-dimensional control volumes}{A triangular element}.}
    \label{fig:controlvolume}
\end{figure}

% \begin{figure}[htbp!]
%     \centering
%     \includegraphics[width=0.6\linewidth]{pics/3d_control_volume.pdf}
%     \caption{\added[id=r1]{Three-dimensional control volumes}.}
%     \label{fig:3dcontrolvolume}
% \end{figure}

By integrating the governing equation \eqref{eq:NS} over control volume $\OO_i$, a semi-discrete finite volume scheme is obtained as
\begin{equation}
    \label{eq:Semi-FV}
    \dfrac{ \dd\UM_i }{\dd t} = -\dfrac{1}{\overline{\OO}_i} \oint_{\partial \OO_i} \left(\F - \F_v \right) \cdot \n \ \dd A,
\end{equation}
where $\overline{\OO}_i$ is the volume of $\OO_i$, $\partial \OO_i$ is the boundary of $\OO_i$ and $\n$ is the outward unit normal of $\partial \OO_i$, as shown in Figure \ref{fig:controlvolume}. The cell-average defined by
\begin{equation}
    \label{eq:FVMean}
    \UM_i \left(t\right)= \frac{1}{\overline{\OO}_i}\int_{\OO_i}\U \left(\x,t\right)\ \dd V,
\end{equation}
is the degree of freedom (DOF) of the finite volume method on cell $\OO_i$.
%where $\overline{\OO}_i$ is the volume of $\OO_i$.
The flux integral in \eqref{eq:Semi-FV} is computed by using a Gauss quadrature
\begin{equation}
    \label{eq:Flux-Integral}
    % \oint_{\partial \OO_i} \left(\F - \F_v \right) \cdot \n \ \dd A \approx \sum_{f \in \partial \OO_i} \sum_{g=1}^{N_g} w_g \left[\F \left(\U \left(\x_{f,g},t\right)\right) - \F_v \left(\U\left(\x_{f,g},t\right), \nabla \U \left(\x_{f,g},t\right)\right)\right] \cdot \n_f \ A_f,
    \oint_{\partial \OO_i} \left(\F - \F_v \right) \cdot \n \ \dd A \approx \sum_{f \in \partial \OO_i} \sum_{g=1}^{N_g} w_g \left. \left(\F - \F_v \right)\right|_{\x=\x_{f,g}} \cdot \n_f \ A_f,
\end{equation}
where $\n_f$ and $A_f$ are the outward unit normal and area of element interface $f$, respectively. $N_g$ is the number of quadrature points. $\x_{f,g}$ and $w_g$ are the position and weight of the $g$-th quadrature point on $f$, respectively.
\added[id=r1]{In this work, Gauss-Legendre quadrature formulas are employed to evaluate the flux integral along a line segment. Tensor products of the one-dimensional Gauss-Legendre quadrature are used to compute the flux integral on a quadrilateral. Symmetric quadrature rules based on triangular coordinates \cite{dunavant1985high} are used to compute the flux integral on a triangle.}

Given known cell-averages, a reconstruction is performed to obtain an approximate solution distribution on the computational domain $\OO$, to compute the states at quadrature points on element interfaces. Specifically, the solution on each control volume is approximated by a polynomial, i.e.,
\begin{equation}
    \label{eq:FVRec}
    \U_i(\x,t) = \UM_i \left(t\right) + \sum_{l=1}^{\mathrm{N_b}(k)}{\U_i^l \left(t\right) \varphi_{i,l}(\x) }, \quad \forall \ \x \in \OO_i, \quad i=1, \cdots, N,
\end{equation}
where $k$ is the degree of the polynomial, $\{\varphi_{i,l}(\x)\}$ are the polynomial basis functions and $\mathrm{N_b}(k)$ is the number of basis functions. The basis coefficients $\{\U^l_i\}$ are determined by using a reconstruction scheme, such as the variational reconstruction \cite{wang2017compact_VR} that will be presented in Section \ref{ssec:VR}.

\begin{figure}[htbp!]
    \centering
    \includegraphics[width=0.3\linewidth]{pics/Riemann}
    \caption{\replaced[id=r1]{Riemann problem on the interface between two triangular elements}{Riemann problem on cell interface}.}
    \label{fig:Riemann}
\end{figure}

Flux schemes based on solutions of Riemann problems are used to compute the numerical flux at the quadrature points in \eqref{eq:Flux-Integral}, since the piece-wise polynomial distribution \eqref{eq:FVRec} is discontinuous across cell interfaces, as shown in Figure \ref{fig:Riemann}.
The numerical flux is computed by
\replaced[id=r2]{
\begin{equation}
    \label{eq:num-flux}
    % \left[\F \left(\U \left(\x_{f,g},t\right)\right) - \F_v \left(\U\left(\x_{f,g},t\right), \nabla \U \left(\x_{f,g},t\right)\right)\right] \cdot \n_f = \tilde{\F} \left(\U_L, \U_R,\n_f\right)- \tilde{\F}_v \left(\U_L, \nabla \U_L, \U_R, \nabla \U_R, \n_f\right),
    \left. \left(\F - \F_v \right)\right|_{\x=\x_{f,g}} \cdot \n_f \approx \tilde{\F} \left(\U_L, \U_R,\n_f\right)- \tilde{\F}_v \left(\U_L, \U_R, \nabla \U_L, \nabla \U_R, \n_f\right),
\end{equation}
}{\begin{equation}
    \label{eq:num-flux}
    % \left[\F \left(\U \left(\x_{f,g},t\right)\right) - \F_v \left(\U\left(\x_{f,g},t\right), \nabla \U \left(\x_{f,g},t\right)\right)\right] \cdot \n_f = \tilde{\F} \left(\U_L, \U_R,\n_f\right)- \tilde{\F}_v \left(\U_L, \nabla \U_L, \U_R, \nabla \U_R, \n_f\right),
    \left. \left(\F - \F_v \right)\right|_{\x=\x_{f,g}} \cdot \n_f = \tilde{\F} \left(\U_L, \U_R,\n_f\right)- \tilde{\F}_v \left(\U_L, \U_R, \nabla \U_L, \nabla \U_R, \n_f\right),
\end{equation}}
where $\tilde{\F}$ and $\tilde{\F}_v$ are the inviscid and viscous flux schemes, respectively, with the left and right states at the quadrature point defined by
\begin{equation}
    \begin{dcases}
         & \U_L= \U_i\left(\x_{f,g},t \right), \quad \nabla \U_L= \nabla \U_i\left(\x_{f,g},t \right),\\
         & \U_R= \U_j\left(\x_{f,g} ,t \right), \quad \nabla \U_R= \nabla \U_j\left(\x_{f,g},t \right).
    \end{dcases}
\end{equation}
\replaced[id=r1]{In this work, the inviscid flux scheme $\tilde{\F}$ is given by the local Lax-Friedrichs (LLF) flux 
\begin{equation}
    % \tilde{\F} \left(\U_L, \U_R,\n_f\right)= \dfrac{1}{2}\left(\F\left(\U_L\right)+\F\left(\U_R\right)\right) \cdot \n_f - \dfrac{1}{2}\lambda_{\max} \left(\U_R-\U_L\right),
    \tilde{\F} \left(\U_L, \U_R,\n_f\right) = \dfrac{1}{2} \left[\F\left(\U_L\right) + \F\left(\U_R\right)\right]\cdot\n_f - \dfrac{1}{2} \max \left\{\lambda\left(\U_L\right), \lambda\left(\U_R\right)\right\} \left(\U_R - \U_L\right),
\end{equation}
where $\lambda\left(\U\right)$ denotes the spectral radius of the Jacobian 
$\partial \left(\F \cdot \n_f\right)/\partial \U$, computed as
\begin{equation}
    % \lambda_{\max}= \max\left(\abs{\mathbf{u}_L\cdot \n_f}+ \sqrt{\gamma \dfrac{p_L}{\rho_L}}, \ \abs{\mathbf{u}_R\cdot \n_f}+ \sqrt{\gamma \dfrac{p_R}{\rho_R}}\right).
    \lambda\left(\U\right)= \abs{\mathbf{u}\cdot \n_f}+ \sqrt{\gamma p/\rho}.
\end{equation}
The viscous flux scheme $\tilde{\F}_v$ adopts the dGRP-type formulation \cite{gassner2007contribution,gassner2008discontinuous}, as developed in \cite{wang2017compact_VR}:
\begin{equation}
    \tilde{\F}_v \left(\U_L, \U_R, \nabla \U_L, \nabla \U_R, \n_f\right)= \F_v \left( \tilde{\mathbf{W}}, \nabla \tilde{\mathbf{W}}\right) \cdot \n_f,
\end{equation}
where $\tilde{\mathbf{W}}$ and $\nabla \tilde{\mathbf{W}}$ are the averaged state and gradient of the primitive variable $\mathbf{W}=\left(\rho, \mathbf{u},p\right)^\top$, respectively,  computed by
% where $\tilde{\mathbf{W}}$ and $\nabla \tilde{\mathbf{W}}$ are, respectively, the averaged primitive state and its gradient. The primitive variable is defined as $\mathbf{W} = \left(\rho, \mathbf{u}, p\right)^\top$, with
\begin{equation}
% \begin{aligned}
    \tilde{\mathbf{W}}= \dfrac{1}{2}\left[\mathbf{W}\left(\U_L\right)+\mathbf{W}\left(\U_R\right)\right], \quad
    %\ \mathbf{W}_L= \mathbf{W}\left(\U_L\right), \ \mathbf{W}_R= \mathbf{W}\left(\U_R\right),\\
    \nabla \tilde{\mathbf{W}}= \left. \left(\dfrac{\partial \U}{\partial \mathbf{W}}\right)^{-1} \right|_{\mathbf{W}=\tilde{\mathbf{W}}} \nabla \tilde{\U}.
% \end{aligned}
\end{equation}
The averaged gradient of the conservative variable is computed by
\begin{equation}
    \nabla \tilde{\U}=\dfrac{1}{2} \left(\nabla \U_L + \nabla \U_R\right) + \dfrac{1}{2} \dfrac{\left(\U_R- \U_L\right)}{L_f}\n_f,
\end{equation}
where the characteristic length scale $L_f$ is defined, according to Hartmann et al. \cite{hartmann2006symmetric}, as $L_f=\min\left(\overline{\OO}_L,\overline{\OO}_R\right)/A_f$.
}
{In this work, $\tilde{\F}$ is the local Lax-Friedrichs scheme and $\tilde{\F}_v$ is the viscous flux scheme in \cite{wang2017compact_VR}.}

%In this work, the inviscid numerical flux is computed by using the local Lax-Friedrichs scheme
%\begin{equation}
%	\label{eq:LLF}
%	\F \left(\U \left(\x_{f,g},t\right)\right) \cdot \n_f = \dfrac{1}{2} \left[\F\left(\U_L\right) + \F\left(\U_R\right)\right]\cdot\n_f - \dfrac{1}{2} \max \left(\lambda\left(\U_L\right), \lambda\left(\U_R\right)\right) \left(\U_R - \U_L\right),
%\end{equation}
%where 
%\begin{equation}
%	\U_L= \U_i\left(\x_{f,g},t \right), \quad \U_R= \U_j\left(\x_{f,g} ,t \right), \quad \lambda\left(\U\right)= \abs{\mathbf{u} \cdot \n_f} + \sqrt{\gamma p/\rho},
%\end{equation}
%with $j$ being the face-neighboring cell that shares $f$ with cell $i$, i.e., $\OO_i \cap \OO_j= f$.
%The viscous numerical flux is computed by using the dGRP \cite{gassner2007contribution,gassner2008discontinuous} type scheme in \cite{wang2017compact_VR} 
%\begin{equation}
%		\label{eq:dGRP}
%	\F_v \left(\U \left(\x_{f,g},t\right), \nabla \U \left(\x_{f,g},t\right)\right) \cdot \n_f = \F_v \left(\tilde{\U}, \nabla \tilde{\U}  \right) \cdot \n_f,
%\end{equation}
%where 
%\begin{equation}
%	\tilde{\U} = \dfrac{1}{2} \left(\U_L+\U_R\right), \quad \nabla \tilde{\U}= \dfrac{1}{2}\left(\nabla \U_L+ \nabla \U_R\right) + \dfrac{1}{2 \Delta \tilde{x}} \left(\U_R - \U_L\right), \quad \Delta \tilde{x} = \dfrac{\min \left(\overline{\OO}_i,\overline{\OO}_j\right)}{A_f}.
%\end{equation}

\replaced[id=r1]{Once the right-hand-side is computed, the semi-discrete finite volume scheme \eqref{eq:Semi-FV} is reduced to an ordinary differential equation (ODE)
% \begin{equation}
%     \label{eq:FVODECell}
%     \derivative{\uu_i}{t} = \R_i \left(t, \left\{\U_j\right\}_{j \in S_i} \right),
% \end{equation}
\begin{equation}
    \label{eq:FVODECell}
    \derivative{\uu_i}{t} = \R_i,
\end{equation}
where $\R_i= -\oint_{\partial \OO_i} \left(\F - \F_v \right) \cdot \n \ \dd A/\overline{\OO}_i$ is the right-hand-side of \eqref{eq:Semi-FV}. The spatial discretization procedures, including data reconstruction and flux computation, inherently couple the solutions across different control volumes. As a result, the cell-averages cannot be updated independently. Accordingly, the temporal evolution of the solution is governed by a system of ODEs
\begin{equation}
    \label{eq:FVODE}
    \derivative{\uu}{t} = \R \left(\uu\right),
\end{equation}
where 
\begin{equation}
    \uu= \left(\uu_1,\cdots,\uu_N\right)^\top, \quad \R= \left(\R_1,\cdots,\R_N\right)^\top.
\end{equation}
The ODE system \eqref{eq:FVODE} can be integrated in time to update the cell-averages in a step-by-step manner. The time integration method used in this work will be presented in Section \ref{ssec:TimeMarching}.}
{Once the right-hand-side is computed, the semi-discrete finite volume scheme \eqref{eq:Semi-FV} is reduced to an ordinary differential equation (ODE)
% \begin{equation}
%     \label{eq:FVODECell}
%     \derivative{\uu_i}{t} = \R_i \left(t, \left\{\U_j\right\}_{j \in S_i} \right),
% \end{equation}
\begin{equation}
    \label{eq:FVODECell}
    \derivative{\uu_i}{t} = \R_i,
\end{equation}
which can be integrated in time to update the cell-average in a step-by-step manner. The time integration method used in this work will be presented in Section \ref{ssec:TimeMarching}.}

\subsection{Variational reconstruction}
\label{ssec:VR}

This subsection presents a variational reconstruction \cite{wang2017compact_VR} that can achieve arbitrary high-order accuracy on a compact stencil involving only the current cell and its face-neighboring cells. The compact reconstruction stencil $S_i= \left\{i,j_1,j_2,j_3\right\}$ of a triangular element $i$ is shown in Figure \ref{fig:compactstencil}. 

\begin{figure}[htbp!]
    \centering
    \includegraphics[width=0.3\linewidth]{pics/compact_stencil}
    \caption{Compact reconstruction stencil of a triangular element.}
    \label{fig:compactstencil}
\end{figure}

% \added[id=r1]{The key component of a FV method to achieve high-order accuracy is the high-order representation of physical data inside the cell. 
% The large stencil has been the major drawback of conventional high-order reconstructions on unstructured grids, which can cause stability and efficiency issues \cite{abgrall2011construction,zhang2012class,li2014efficient}. A series of reconstruction procedures, which can achieve high-order accuracy on a compact stencil, have been developed to overcome solve this bottleneck problem \cite{wang2016compact1_VR,wang2017compact_VR,zhang2019compact_VR}. The reconstruction \cite{wang2017compact_VR} is provably non-singular and is thus used in this work.
\added[id=r1]{A key factor in enabling high-order accuracy in FV methods is the high-order representation of the physical solution within each cell. In traditional high-order reconstructions on unstructured grids, such as $k$-exact \cite{delanaye1999quadratic,barth1990higher,ollivier2002high} and ENO \cite{abgrall1994essentially}/WENO \cite{friedrich1998weighted_WENO,hu1999weighted_WENO,dumbser2007quadrature_WENO}, the polynomial approximation of the solution within each control volume is determined by enforcing a mean-preserving condition across a local stencil. 
% This condition leads to large reconstruction stencils because the number of neighboring cells required must at least match the number of unknown coefficients in the polynomial, which grows rapidly with the polynomial degree. 
However, this condition necessitates large reconstruction stencils, as the number of neighboring cells included must at least equal the number of polynomial coefficients, which increases rapidly with polynomial degree. Such large reconstruction stencils can pose challenges in terms of numerical stability and computational efficiency \cite{abgrall2011construction, zhang2012class, li2014efficient}. The large reconstruction stencil has been the major bottleneck problem in developing high-order finite volume schemes on unstructured grids. To address this limitation, several reconstruction procedures have been developed to achieve high-order accuracy on compact stencils \cite{wang2016compact1_VR, wang2017compact_VR, zhang2019compact_VR}. Among these compact reconstructions, the variational reconstruction proposed in \cite{wang2017compact_VR} is provably non-singular and is therefore adopted in this study. For clarity, the two-dimensional case will be used to illustrate the variational reconstruction.}
\deleted[id=r1]{In general, the conservative variables are reconstructed separately, allowing an implementation of solution reconstruction in a variable-by-variable manner. For the sake of presentation, the two-dimensional case is considered to illustrate the variational reconstruction.}

\added[id=r1]{In general, the conservative variables are reconstructed separately, allowing an implementation of solution reconstruction in a variable-by-variable manner.} On control volume $\OO_i$, a conservative variable $u \in \U$ is approximated as
\begin{equation}
    \label{eq:recon_u}
    u_i \left(\x\right)= \overline{u}_i + \sum^{\mathrm{N_b}\left(k\right)}_{l=1} u_i^l \varphi_{i,l} \left(\x\right),
\end{equation}
where $\{\varphi_{i,l}\}$ are the basis functions, $\{u^l_i\}$ are the unknown basis coefficients and $\mathrm{N_b}(k)$ is the number of basis functions. For a two-dimensional case, $\mathrm{N_b}(k)=  (k+2)(k+1)/2 -1$. In \eqref{eq:recon_u}, the time $t$ is omitted as it remains a constant during the reconstruction procedure.
%In order to determine the coefficients of polynomial bases $\U_i^l$ (or $u_i^l$ for each scalar) in
%\eeqref{eq:FVRec}, a reconstruction method needs to be specified.
%Traditional 2nd order FV methods for unstructured grid
%needs only to reconstruct a $k=1$ polynomial, namely linear
%distribution on each cell.
%The variational reconstruction \cite{wang2017compact_VR}
%is a compact high-order
%reconstruction scheme, and it achieves high-order
%accuracy on a compact stencil.
%The current section will explain the variational reconstruction briefly
%and specify details concerning following numerical tests.
In this work, the basis functions with the zero-mean property
\begin{equation}
    \frac{1}{\overline{\OO}_i} \int_{\OO_i} \varphi_{i,l} \left(\x\right) \ \dd V=0,
\end{equation}
are used to make the reconstruction polynomial \eqref{eq:recon_u} automatically satisfy the conservation condition
\begin{equation}
    \label{eq:zero-mean}
    \frac{1}{\overline{\OO}_i}\int_{\OO_i}u_i \left(\x\right)\ \dd V= \overline{u}_i.
\end{equation}
The zero-mean basis functions are defined by
\begin{equation}
    \varphi_{i,l} =
    \left(\frac{x - x_{i}}{\inc x_i}\right)^{p_l}
    \left(\frac{y - y_{i}}{\inc y_i}\right)^{q_l}
    -
    \overline{
        \left(\frac{x - x_{i}}{\inc x_i}\right)^{p_l}
        \left(\frac{y - y_{i}}{\inc y_i}\right)^{q_l}
    },
\end{equation}
with
\begin{equation}
    \overline{
        \left(\frac{x - x_{i}}{\inc x_i}\right)^{p_l}
        \left(\frac{y - y_{i}}{\inc y_i}\right)^{q_l}
    }= \dfrac{1}{\overline{\OO}_i}\int_{\OO_i}   \left(\frac{x - x_{i}}{\inc x_i}\right)^{p_l}
    \left(\frac{y - y_{i}}{\inc y_i}\right)^{q_l}  \dd V,
\end{equation}
where $\left(x_i,y_i\right)$ and $\left(\inc x_i, \inc y_i\right)$ are the barycenter and characteristic length scales of cell $i$, respectively, and $\left(p_l,q_l\right)$ are the powers of the basis functions organized in ascending order of $p_l+q_l$. For instance, the powers of the cubic ($k=3$) reconstruction polynomial are
\begin{equation}
    \left\{ \left(p_l,q_l\right) \right\}^9_{l=1}= \left\{ \left(1,0\right), \left(0,1\right), \left(2,0\right),\left(1,1\right),\left(0,2\right),\left(3,0\right),\left(2,1\right),\left(1,2\right),\left(0,3\right) \right\}.
\end{equation}
%The mean value term makes the basis zero-mean, which is calculated with:
%\begin{equation}
%    \overline{
%        \left(\frac{x - x_{c,i}}{\inc x_i}\right)^{p_l}
%        \left(\frac{y - y_{c,i}}{\inc y_i}\right)^{q_l}
%    }
%    =
%    \frac{1}{\overline{\OO}_j}\int_{\OO_j}{
%        \left(\frac{x - x_{c,i}}{\inc x_i}\right)^{p_l}
%        \left(\frac{y - y_{c,i}}{\inc y_i}\right)^{q_l}
%    }\dd \Omega
%\end{equation}
The characteristic length scales are used to non-dimensionalize the basis functions, to avoid a growth of the condition number of the reconstruction matrix during grid refinement \cite{abgrall1994essentially,friedrich1998weighted_WENO}. In this work, the length scales are set as $ \inc x_i = \inc y_i = \max_{\x \in \OO_i} \left\{\|\x - \x_{i}\|_2\right\}$.
%\begin{equation}
%    \inc x_i = \inc y_i = \max_{\x \in \OO_i} \left\{\|\x - \x_{i}\|_2\right\}.
%\end{equation}

The objective of a reconstruction is to determine the unknown basis coefficients $u_i^l$,  $l=1$, $\cdots$, $\mathrm{N_b}\left(k\right)$, $i=1,\cdots,N$, given the cell-averages $\overline{u}_j$, $j=1,\cdots,N$. In the variational reconstruction \cite{wang2017compact_VR}, the linear equation system to determine the unknown basis coefficients is derived by minimizing a cost function using the variational method. Different cost function results in different reconstruction schemes. \replaced[id=r1]{In this work, the cost function is defined as
\begin{equation}
    \label{eq:cost-function}
    I = \sum^{N_f}_{f=1} \omega_f {I_f},
\end{equation}
where $N_f$ is the number of cell interfaces on the computational domain, $I_f$ is an interfacial jump integration (IJI) on cell interface $f$ defined as
\begin{equation}
    \label{eq:IJI}
    I_f =  \sum_{p+q=0}^{k} \omega_{p,q} \int_{f}{ 
            \left(
            \pdv{^{p+q}u_L}{x^p\partial y^q} \left(d_{LR}\right)^{p+q}
            -
            \pdv{^{p+q}u_R}{x^p\partial y^q} \left(d_{LR}\right)^{p+q}
            \right)^2
        \ \dd A
    },
\end{equation}
and $\omega_f >0 $ is the weight of $I_f$.
In the IJI definition \eqref{eq:IJI}, $\left\{\omega_{p,q}|\omega_{p,q}>0\right\}_{p+q=0}^k$ are the weights of the integral terms, $L$ and $R$ are the two cells sharing the interface $f$, and $d_{LR}= \left\| \x_L -\x_R \right\|_2$ is the distance between the centroids of elements $L$ and $R$. The IJI measures the jumps of the reconstruction polynomial and its spatial derivatives on the cell interface. The cost function \eqref{eq:cost-function} is the weighted sum of the IJIs over the entire computational domain.}{The cost function is defined as
\begin{equation}
    \label{eq:cost-function}
    I = \sum^{N_f}_{f=1} {I_f},
\end{equation}
where $I_f$ is an interfacial jump integration (IJI) on cell interface $f$ and $N_f$ is the total number of cell interfaces on the computational domain. In this work, the IJI is defined as
\begin{equation}
    \label{eq:IJI}
    I_f = \omega_f^G \sum_{p+q=0}^{k} \int_{f}{
        \left[
            \omega_f^D(p,q)
            \left(
            \partialderivative{^{p+q}u_L}{x^p\partial y^q}
            -
            \partialderivative{^{p+q}u_R}{x^p\partial y^q}
            \right)
            \right]^2
        \dd A
    },
\end{equation}
where $L$ and $R$ are the two cells sharing the interface $f$, $\omega_f^G$ is the geometric weight and $\omega_f^D$ is the derivative weight.
The IJI measures the jumps of the reconstruction polynomial and its spatial derivatives on the cell interface.
The derivative weights of the cubic variational reconstruction are
\begin{equation}
    \begin{aligned}
        \omega_f^D(0,0) & = \omega_D(0),                                                                               \\
        \omega_f^D(1,0) & = \omega_f^D(0,1) = \omega_D(1) ,                                                            \\
        \omega_f^D(2,0) & = \omega_f^D(0,2) = \omega_D(2),\ \omega_f^D(1,1) = \sqrt{2}\ \omega_D(2)  ,                 \\
        \omega_f^D(3,0) & = \omega_f^D(0,3) = \omega_D(3),\ \omega_f^D(1,2) = \omega_f^D(2,1) = \sqrt{3}\ \omega_D(3), \\
    \end{aligned}
    \label{eq:wdRotRatio}
\end{equation}
with
\begin{equation}
    \omega_D(0) = 1, \ \omega_D(1) = d_{LR}, \ \omega_D(2) = \frac{\left(d_{LR}\right)^2}{2}, \ \omega_D(3) = \frac{\left(d_{LR}\right)^3}{6},
    \label{eq:wdHQMOPT}
\end{equation}
where $d_{LR}= \left\| \x_L -\x_R \right\|_2$ is the distance between the centroids of elements $L$ and $R$.
As only nearly uniform and isotropic meshes are used in the numerical simulations in this paper,
the geometric weights
$\omega^G_f$ are all set as 1.}
The unknown basis coefficients are determined by minimizing the cost function \eqref{eq:cost-function}, resulting in a \textquote{smoothest} piece-wise polynomial distribution which has the smallest jumps measured by the IJIs on cell interfaces.
Linear reconstruction equations can be obtained by using the variational method, i.e.,
\begin{equation}
    \label{eq:minimization}
    \partialderivative{I}{u_i^l} = 0, \ l=1,\cdots, \mathrm{N_b}(k),\  i=1,\cdots N.
\end{equation}
\replaced[id=r1]{It is observed from the definition \eqref{eq:IJI} that, the polynomial $u_i$ is only involved in the IJIs on the faces of of cell $i$. Therefore, linear equations can be derived according to \eqref{eq:minimization} to determine the unknown coefficients of cell $i$ using the unknown reconstruction polynomials of its face-neighbors, indicating that the variational reconstruction is compact and implicit. Substituting \eqref{eq:recon_u}, \eqref{eq:cost-function} and \eqref{eq:IJI} into \eqref{eq:minimization}, we obtain the following linear equations
\begin{equation}
\label{eq:variational-linear-equation}
\begin{aligned}
    \sum_{f \in \partial \OO_i} \omega_{f} \sum_{p+q=0}^{k} \omega_{p,q} d^{2p+2q}_{ij} 
           \int_{f}{ \partialderivative{^{p+q}\varphi_{i,l}}{x^p\partial y^q}
             \partialderivative{^{p+q}}{x^p\partial y^q} \left(\overline{u}_i + \sum^{\mathrm{N_b}\left(k\right)}_{m=1} u_i^m \varphi_{i,m} - \overline{u}_j - \sum^{\mathrm{N_b}\left(k\right)}_{n=1} u_j^n \varphi_{j,n} 
             \right)
            \ \dd A}
            =0, &\\
        %     &\sum_{j \in S_i, j \neq i} \sum^{\mathrm{N_b}\left(k\right)}_{n=1} \omega_f \int_{f}{
        %     \sum_{p+q=0}^{k}
        %     \omega_{p,q} d^{2p+2q}_{ij}
        %     \partialderivative{^{p+q}\varphi_{i,l}}{x^p\partial y^q}
        %     \partialderivative{^{p+q}\varphi_{j,n}}{x^p\partial y^q}
        %     \ \dd A} \ u_j^n
        %     +
        %     \sum_{j \in S_i, j \neq i}
        % \omega_f \ \omega_{0,0}
        % \int_{f}{
        %     \varphi_{i,l} \left(\overline{u}_j - \overline{u}_i\right)
        %     \ \dd A}, \\
            l=1,\cdots, \mathrm{N_b}(k), \quad  i=1,\cdots N,&
\end{aligned}
\end{equation}
where $j$ is the neighboring cell sharing interface $f$ with cell $i$.
The linear equation system \eqref{eq:variational-linear-equation} can be written in a matrix form
\begin{equation}
    \label{eq:vrBlockEq}
    \mathbf{A}_{i} \us_i
    =
    \sum_{j \in S_i, j \neq i} \mathbf{B}^j_{i} \us_j + \mathbf{b}_{i}, \quad i=1, \cdots, N,
\end{equation}
where the elements of matrices
$\mathbf{A}_i$, $\mathbf{B}^j_i \in \mathbb{R}^{\mathrm{N_b}(k) \times \mathrm{N_b}(k)}$
and vectors $\mathbf{u}_i$, $\mathbf{u}_j$, $\mathbf{b}_i \in \mathbb{R}^{\mathrm{N_b}(k)}$ are
\begin{equation}
    \label{eq:vrCoeffs}
    \begin{aligned}
         & \mathbf{A}_{i} [l,m]= \sum_{j \in S_i, j \neq i} \omega_{f} 
           \int_{f}{
             \sum_{p+q=0}^{k} \omega_{p,q} d^{2p+2q}_{ij}
            \partialderivative{^{p+q}\varphi_{i,l}}{x^p\partial y^q}
            \partialderivative{^{p+q}\varphi_{i,m}}{x^p\partial y^q}
            \ \dd A},
        \\
         & \mathbf{B}^j_{i} [l,n]=
        \omega_f \int_{f}{
            \sum_{p+q=0}^{k}
            \omega_{p,q} d^{2p+2q}_{ij}
            \partialderivative{^{p+q}\varphi_{i,l}}{x^p\partial y^q}
            \partialderivative{^{p+q}\varphi_{j,n}}{x^p\partial y^q}
            \ \dd A},
        \\
         & \mathbf{b}_{i} [l]=
        \sum_{j \in S_i, j \neq i}
        \omega_f \ \omega_{0,0}
        \int_{f}{
            \varphi_{i,l} \left(\overline{u}_j - \overline{u}_i\right)
            \ \dd A},
        \\
         & \mathbf{u}_i [m]= u^m_i, \\
             & \mathbf{u}_j [n]= u^n_j.
    \end{aligned}
\end{equation}
The face integral terms in the reconstruction matrices can be computed exactly using Gauss quadrature formulas with an adequate number of quadrature points.}
{By substituting \eqref{eq:recon_u} into \eqref{eq:IJI}, \eqref{eq:cost-function} and \eqref{eq:minimization}, a linear equation system is obtained as follows
\begin{equation}
    \label{eq:vrBlockEq}
    \mathbf{A}_{i} \us_i
    =
    \sum_{j \in S_i, j \neq i} \mathbf{B}^j_{i} \us_j + \mathbf{b}_{i}, \quad i=1, \cdots, N,
\end{equation}
where the elements of matrices
$\mathbf{A}_i$, $\mathbf{B}^j_i \in \mathbb{R}^{\mathrm{N_b}(k) \times \mathrm{N_b}(k)}$
and vectors $\mathbf{u}_i$, $\mathbf{u}_j$, $\mathbf{b}_i \in \mathbb{R}^{\mathrm{N_b}(k)}$ are
\begin{equation}
    \label{eq:vrCoeffs}
    \begin{aligned}
         & \mathbf{A}_{i} [m,n]=
        \sum_{j \in S_i, j \neq i} \omega_f^G\int_{f}{
            \sum_{p+q=0}^{k}
            \left(\omega_f^D(p,q)\right)^2
            \partialderivative{^{p+q}\varphi_{i,m}}{x^p\partial y^q}
            \partialderivative{^{p+q}\varphi_{i,n}}{x^p\partial y^q}
            \ \dd A},
        \\
         & \mathbf{B}^j_{i} [m,n]=
        \omega_f^G\int_{f}{
            \sum_{p+q=0}^{k}
            \left(\omega_f^D(p,q)\right)^2
            \partialderivative{^{p+q}\varphi_{i,m}}{x^p\partial y^q}
            \partialderivative{^{p+q}\varphi_{j,n}}{x^p\partial y^q}
            \ \dd A},
        \\
         & \mathbf{b}_{i} [m]=
        \sum_{j \in S_i, j \neq i}
        \omega_f^G \left(\omega_f^D(0,0)\right)^2
        \int_{f}{
            \varphi_{i,m} \left(\overline{u}_j - \overline{u}_i\right)
            \ \dd A},
        \\
         & \mathbf{u}_i [m]= u^m_i, \\
         & \mathbf{u}_j [m]= u^m_j,
    \end{aligned}
\end{equation}
with $f= \partial \OO_i \cap \partial \OO_j$ being the cell interface.}
\deleted[id=r1]{It is observed from \eqref{eq:vrBlockEq} that the variational reconstruction is implicit, as the unknown coefficients of the face-neighboring cells are required to determine the unknown coefficients of the current cell.}
By assembling the linear equations \eqref{eq:vrBlockEq} of all cells, we obtain a global linear equation system
\begin{equation}
    \label{eq:global-system}
    \mathbf{A} \mathbf{u} = \mathbf{b},
\end{equation}
where
\begin{equation}
    \begin{aligned}
         & \mathbf{A}= \mathbf{D} -  \mathbf{L} -  \mathbf{U},
        \ \mathbf{D}= \left\{\mathbf{A}_i\right\},
        \ \mathbf{L}= \left\{\mathbf{B}^j_i, j<i\right\},
        \ \mathbf{U}= \left\{\mathbf{B}^j_i, j>i\right\},
        \\
         & \mathbf{u}= \left\{\mathbf{u}_i\right\}, \ \mathbf{b}= \left\{\mathbf{b}_i\right\}.
    \end{aligned}
\end{equation}
It is proved in \cite{wang2017compact_VR} that the large and sparse matrix $\mathbf{A}$ is symmetric and positive definite, which guarantees the existence and uniqueness of the solution of the linear equation system \eqref{eq:global-system}. This is a significant advantage of the variational reconstruction over other existing high-order reconstructions on unstructured grids. \replaced[id=r1]{The linear equation system \eqref{eq:global-system} is solved iteratively using a block Gauss-Seidel method
\begin{equation}
\label{eq:gauss-seidel-iteration}
    \mathbf{u}_i^{\left(s+1\right)}= \sum_{j \in S_i, j < i} \mathbf{A}^{-1}_i \mathbf{B}^j_i \mathbf{u}^{\left(s+1\right)}_j + \sum_{j \in S_i, j > i} \mathbf{A}^{-1}_i \mathbf{B}^j_i \mathbf{u}^{\left(s\right)}_j + \mathbf{A}^{-1}_i \mathbf{b}_i, \quad i=1,\cdots,N,
\end{equation}
where $s$ denotes the iteration step. It is also proved in \cite{wang2017compact_VR} that the block Gauss-Seidel iteration converges for the linear equation system of variational reconstruction.}
{The linear equation system \eqref{eq:global-system} is solved iteratively using the block Gauss-Seidel method, of which the convergence is proved in \cite{wang2017compact_VR}. Each block Gauss-Seidel iteration is compact as it only relies on the information of the current and face-neighboring cells.}
%It is proved in \cite{wang2017compact_VR} that the block Gauss-Seidel method is convergent for the linear equation system \eqref{eq:global-system}.

\added[id=r1]{A linear variational reconstruction, capable of approximating smooth solutions with second-order accuracy, is employed to demonstrate the fundamental procedures involved in variational reconstruction. For the triangular element $\OO_i$ shown in Figure \ref{fig:compactstencil}, the linear reconstruction polynomial is
\begin{equation}
    u_i\left(\x\right)= \overline{u}_i + \sum^{2}_{l=1} u_i^l \varphi_{i,l} \left(\x\right)= \overline{u}_i + u^1_i \dfrac{x-x_i}{\Delta x_i} + u^2_i \dfrac{y-y_i}{\Delta y_i},
\end{equation}
where $\mathbf{u}_i= \left(u^1_i,u^2_i\right)^\top$ are the coefficients to be determined. The unknown polynomial coefficients of all the cells in the computational domain are determined by minimizing the following cost function
\begin{equation}
    I= \sum^{N_f}_{f=1} \omega_f I_f,
\end{equation}
where the IJI is defined as
\begin{equation}
    % I_f=  \omega_{0,0} \int_f \left(u_L-u_R\right)^2 \dd A + \omega_{1,0} \int_f \left(\partialderivative{u_L}{x} d_{LR}-\partialderivative{u_R}{x} d_{LR} \right)^2 \dd A + \omega_{0,1} \int_f \left(\partialderivative{u_L}{y}d_{LR}-\partialderivative{u_R}{y}d_{LR}\right)^2 \dd A.
    I_f=  \int_f \omega_{0,0}  \left(u_L-u_R\right)^2 + \omega_{1,0} \left(\partialderivative{u_L}{x} d_{LR}-\partialderivative{u_R}{x} d_{LR} \right)^2 + \omega_{0,1} \left(\partialderivative{u_L}{y}d_{LR}-\partialderivative{u_R}{y}d_{LR}\right)^2 \dd A.
\end{equation}
% Based on the following relations
% \begin{equation}
%     u_i= \overline{u}_i + u^1_i \varphi_{i,1} + u^2_i\varphi_{i,2}, \quad \partialderivative{u_i}{x}= u^1_i\partialderivative{\varphi_{i,1}}{x} + u^2_i\partialderivative{\varphi_{i,2}}{x}, \quad \partialderivative{u_i}{y}= u^1_i\partialderivative{\varphi_{i,1}}{y} + u^2_i\partialderivative{\varphi_{i,2}}{y},
% \end{equation}
By applying the variational method to the minimization of the cost function, a system of linear equations is obtained as follows
\begin{equation}
\label{eq:2nd-global-system}
    \mathbf{A}_i \mathbf{u}_i= \mathbf{B}_i^{j_1} \mathbf{u}_{j_1} + \mathbf{B}_i^{j_2} \mathbf{u}_{j_2} + \mathbf{B}_i^{j_3} \mathbf{u}_{j_3} + \mathbf{b}_i, \quad i=1,\cdots,N,
\end{equation}
where the elements of matrices $\mathbf{A}_i$, $\mathbf{B}^j_i \in \mathbb{R}^{2 \times 2}$ and $\mathbf{b}_i \in \mathbb{R}^{2}$ are
\begin{equation}
    \begin{aligned}
         & \mathbf{A}_{i} [l,m]= \sum^{j_3}_{j=j_1} \omega_{f} 
           \int_{f}{ 
             \omega_{0,0}
            \varphi_{i,l}\varphi_{i,m}
            +
            \omega_{1,0} d^{2}_{ij}
            \partialderivative{\varphi_{i,l}}{x}
            \partialderivative{\varphi_{i,m}}{x}
            +
             \omega_{0,1} d^{2}_{ij}
            \partialderivative{\varphi_{i,l}}{y}
            \partialderivative{\varphi_{i,m}}{y}
            \ \dd A},
        \\
         & \mathbf{B}^j_{i} [l,n]=
        \omega_f \int_{f}{ 
             \omega_{0,0}
            \varphi_{i,l}\varphi_{j,n}
            +
            \omega_{1,0} d^{2}_{ij}
            \partialderivative{\varphi_{i,l}}{x}
            \partialderivative{\varphi_{j,n}}{x}
            +
             \omega_{0,1} d^{2}_{ij}
            \partialderivative{\varphi_{i,l}}{y}
            \partialderivative{\varphi_{j,n}}{y}
            \ \dd A},
        \\
         & \mathbf{b}_{i} [l]=
        \sum^{j_3}_{j=j_1}
        \omega_f \ \omega_{0,0}
        \int_{f}{
            \varphi_{i,l} \left(\overline{u}_j - \overline{u}_i\right)
            \ \dd A}.
    \end{aligned}
\end{equation}
The linear equation system \eqref{eq:2nd-global-system} can be solved iteratively using the block Gauss-Seidel method \eqref{eq:gauss-seidel-iteration}.
}

\added[id=r1]{In the variational reconstruction, there are two categories of weights: the geometric weight $\omega_f$ in \eqref{eq:cost-function} that governs the relative significance of the IJIs on cell interfaces, and the derivative weights $\left\{\omega_{p,q}\right\}^k_{p+q=0}$ in \eqref{eq:IJI} that determine the relative importance assigned to jump integrals involving various spatial derivatives. These weights are the parameters that affect the accuracy and efficiency of the high-order FV method using the variational reconstruction. Different weights actually result in different variational reconstruction schemes. Therefore, the weights should be optimized for an improvement in accuracy or efficiency. However, optimization of numerical schemes on unstructured grids is challenging, due to the complexity of mesh geometry that makes widely used optimization techniques on structured grids, such as spectral property analysis \cite{sun2011class}, not applicable. Recently, Zhou et al. \cite{zhou2024machine} developed an optimization framework based on machine learning for variational reconstruction on triangular meshes. Optimal weights on general unstructured grids are still not available. The geometric weight is designed to work effectively mainly on grids with large aspect ratios. Therefore, the following geometric weight  
\begin{equation}
    \omega_f= \sqrt{\dfrac{A_f}{d_{LR}}},
\end{equation}
obtained through a spectral property optimization on highly stretched quadrilateral grids by Huang et al. \cite{huang2022high}, is adopted in this work.
The following derivative weights
\begin{equation}
    \begin{aligned}
        \omega_{0,0} &=1, \\
        \omega_{1,0} & = \omega_{0,1} = 1,\\
        \omega_{2,0} & = \omega_{0,2} = 1/4, \ \omega_{1,1} = 1/2,\\
        \omega_{3,0} & = \omega_{0,3} = 1/36, \ \omega_{2,1} = \omega_{1,2} = 1/12,
    \end{aligned}
    \label{eq:wdRotRatio}
\end{equation}
which can be used in linear to cubic variational reconstructions, are derived from the factorial form used by Pan et al. \cite{pan2018high_VR}.
Modifications are added to the cross-derivative weights to establish
rotational invariance of the reconstruction functional.
}

\added[id=r1]{
The global linear system to be solved in variational reconstruction is
tested on 4 meshes displayed in Figure \ref{fig:meshes_test_cond}.
Fourth order variational reconstruction (cubic polynomial) with 
periodic boundary conditions is applied to the meshes and 
the global reconstruction matrix $\mathbf{A}$ from \eqref{eq:global-system} 
is analyzed for condition number. 
Meanwhile, spectral radius of the iteration matrix arising from block Gauss-Seidel iteration in \eqref{eq:gauss-seidel-iteration}, $\rho_{GS}$, is calculated on these meshes to investigate the convergence behavior of variational reconstruction. 
}

\added[id=r1]{
Results of condition number and iteration spectral radius for meshes in Figure \ref{fig:meshes_test_cond} are shown in Table \ref{tab:cond_and_rho}. 
It is shown that on regular meshes, 4th order (cubic) variational reconstruction has relatively lower global condition number and 
lower spectral radius in block Gauss-Seidel iteration.
On distorted meshes, the condition number and spectral radius are increased,
but still remain in reasonable ranges. 
The spectral radius results all stay below $1$ according to the general theoretical
analysis of block Gauss-Seidel iteration in varitaional reconstruction\cite{wang2017compact_VR}.
}

\begin{table}[htbp!]
    \centering
    \caption{}
    \label{tab:cond_and_rho}
    % \footnotesize
    % \begin{tabular}{|c|c|c|c|c|}
    \setlength{\tabcolsep}{12.5pt} % Increase column spacing
    \renewcommand{\arraystretch}{1.2}
    \begin{tabular}{ c c c c c}
        \toprule
        Mesh & Quad & Quad distorted & Tri & Tri distorted \\
        \midrule
        Condition number &339.4 & 609.8 & 145.8 & 1438 \\
        % \hline
        $\rho_{GS}$ &0.7652 & 0.7820 & 0.7859 & 0.8422\\
    \bottomrule
    \end{tabular}
\end{table}

\begin{figure}[htbp]
   \centering
   \begin{subfigure}{0.4\textwidth}
       \includegraphics[width=\textwidth]{pics/IV_10_10_Quad.png}
       \caption[]{Quadrilateral elements}
   \end{subfigure}
%    \hfill
   \begin{subfigure}{0.4\textwidth}
       \includegraphics[width=\textwidth]{pics/IV_10_10_Quad_Pert0.png}
       \caption[]{Quadrilateral elements, distorted}
   \end{subfigure}
   \begin{subfigure}{0.4\textwidth}
       \includegraphics[width=\textwidth]{pics/IV_10_10_Tri.png}
       \caption[]{Triangular elements}
   \end{subfigure}
%    \hfill
   \begin{subfigure}{0.4\textwidth}
       \includegraphics[width=\textwidth]{pics/IV_10_10_Tri_Pert0.png}
       \caption[]{Triangular elements, distorted}
   \end{subfigure}
   \caption{Meshes used for condition number and iteration spectral radius test of variational reconstruction.}
   \label{fig:meshes_test_cond}
\end{figure}

\added[id=r1]{Boundary conditions, such as symmetry and non-slip solid wall, should be implemented in the reconstructions on boundary cells to ensure the accuracy of the numerical solution. The boundary conditions can be conveniently implemented in the variational reconstruction procedure by simply adding IJIs on the boundary faces to the cost function \cite{wang2017compact_VR} as follows
\begin{equation}
    I= \sum^{N_f}_{f=1} \omega_f I_f + \sum^{N_{bf}}_{bf=1} \omega_{bf} I_{bf},
\end{equation}
where $N_{bf}$ is the total number of boundary faces, $I_{bf}$ is the IJI on boundary face $bf$, and $\omega_{bf}$ is the geometric weight of $I_{bf}$.
We take the non-slip solid wall boundary condition as an example to illustrate the construction of the boundary IJI term $I_{bf}$. A non-slip solid wall boundary face $bf$ that belongs to a triangular element $\OO_i$ is shown in Figure \ref{fig:bundary_cell}. The geometric weight $\omega_{bf}$ is computed by
\begin{equation}
    \omega_{bf}= \sqrt{\dfrac{A_{bf}}{2\|\x_{i} - \x_{bf} \|_2}},
\end{equation}
where $\x_{bf}$ is the midpoint of face $bf$.
The boundary IJI term for a conservative variable $u \in \U$ is defined as
\begin{equation}
    I_{bf}= \int_{bf} \left( u_i - u_{bf} \right)^2 \ \dd A,
\end{equation}
where $u_{bf}$ is the boundary value of $u$, which is determined according to the specific boundary condition. For instance, given a stationary non-slip isothermal solid wall with a specified wall temperature $T_{wall}$, the boundary values of the conservative variables are defined as \cite{mengaldo2014guide} 
\begin{equation}
    \rho_{bf}= \rho_i, \ \left(\rho \mathbf{u}\right)_{bf}= -\left(\rho \mathbf{u}\right)_i, \ \left(\rho E\right)_{bf}= C_p \rho_i T_{wall}/\gamma.
\end{equation}
Boundary IJI terms for other types of boundary conditions can be constructed in a similar way.
}

\begin{figure}[htbp!]
    \centering
    \includegraphics[width=0.4\linewidth]{pics/bundary_cell}
    \caption{\added[id=r1]{A boundary cell in a two-dimensional mesh.}}
    \label{fig:bundary_cell}
\end{figure}

In the reconstruction of a solution that involves discontinuities, a WBAP limiter \cite{li2011multi,li2012multi} is used to suppress spurious oscillations. The limiting process on the element $i$ in Figure \ref{fig:compactstencil} is presented for illustration. The limited reconstruction polynomial
\begin{equation}\label{eqn:Limited_Polynomial}
    \tilde{u}_{i} \left(\x\right) = \overline{u}_{i} + \sum_{l=1}^{\mathrm{N_b}\left(k\right)} \tilde{u}_i^l \varphi_{i,l}  \left(\x\right),
\end{equation}
is a non-linearly weighted average of the reconstruction polynomial $u_i \left(\x\right)$ and three additional candidate polynomials $u_{j \rightarrow i} \left(\x\right)$, $j=j_1, j_2, j_3$ obtained through a secondary reconstruction \cite{li2012high}. The limited coefficients $\tilde{u}^l_i$ are computed by
\begin{equation}\label{eq:Limiting}
    \tilde{u}_i^l= L\left(u_i^l, u^l_{j_1 \rightarrow i}, u^l_{j_2 \rightarrow i}, u^l_{j_3 \rightarrow i}\right),
\end{equation}
where
\begin{equation}
    L(a_0,a_1,\cdots, a_J)=a_0\cdot{W\left(1,\frac{a_1}{a_0}, \cdots, \frac{a_J}{a_0}\right)},
\end{equation}
with the WBAP limiting function defined as
\begin{equation}
W(1,\theta_1,\cdots, \theta_J)=\frac{n+\sum_{m=1}^{J}{1/\theta_m^{p-1}}}{n+\sum_{m=1}^{J}{1/\theta_m^{p}}}.
\end{equation}
\added[id=r2]{In the WBAP limiting function, $n$ and $p$ are the two important parameters that affect the properties of the limiter. The parameter $n \geq 1$ controls the relative weighting of the first input, corresponding to the coefficient $u_i^l$ of the current cell in the limiting process described by equation \eqref{eq:Limiting}. Increasing $n$ allows the limited polynomial to more closely approximate its unlimited counterpart. However, excessively large values of $n$ can result in inadequate suppression of spurious numerical oscillations. The parameter $p$, which is an even integer, governs the dissipation introduced by the limiter. Larger values of $p$ are well-suited to limiting procedures in higher-order numerical schemes. Following previous studies \cite{wang2017compact_VR,zhang2019compact_VR,zhou2024machine}, the parameters of the limiting function as set as $n=10$ and $p=4$ in this work.} The limiting is performed in a successive manner in characteristic space \cite{li2012multi}.

\subsection{Implicit time integration}
\label{ssec:TimeMarching}

\added[id=r2]{
In engineering CFD simulations, implicit time marching methods are commonly employed to overcome linear stability constraints that restrict time step sizes, thus enhancing computational efficiency. Therefore, this research aims to develop positivity-preserving algorithms for finite volume methods that utilize implicit time marching, extending their applicability to practical engineering problems.
% In engineering CFD scenarios, implicit time marching is widely adopted 
% to overcome the strict linear stability constraint on the physical CFL number.
% Therefore, the current research focuses on positivity-preserving implicit time marching methods to extend the application of positivity-preserving
% algorithms to engineering problems.
Another motivation for adopting implicit time marching is the implicit nature of the variational reconstruction.}
As mentioned in the previous subsection, the linear equation system of the variational reconstruction is solved iteratively. The reconstruction will be very expensive if the iteration needs to reach convergence at each single time step. A reconstruction and implicit dual-time stepping coupled iteration procedure has been proposed in \cite{wang2016compact1_VR} to address this computational efficiency issue. In the coupled iteration procedure, the reconstruction iteration is performed only once at each pseudo time step. The coupling between the reconstruction and the time integration makes these two procedures achieve convergence synchronously. By using the coupled iteration, the implicit nature of the variational reconstruction does not result in additional cost, thus ensuring the high computational efficiency of the variational finite volume method.

\added[id=r1]{For unsteady flow simulations, achieving high-order accuracy requires both high-order spatial and temporal discretizations. Multi-stage implicit Runge-Kutta methods \cite{butcher1964implicit} are widely adopted for high-order time integration. In this paper, a fourth-order finite volume scheme is used to perform the numerical experiments presented in Section \ref{sec:Results}. Specifically, a cubic variational reconstruction is used to achieve fourth-order accuracy in space, and a six-stage, fourth-order explicit first stage singly diagonal implicit Runge-Kutta (ESDIRK4) \cite{bijl2002implicitBDFvESDIRK} method is used to achieve fourth-order accuracy in time. 
The ESDIRK4 method is designed to be $L$-stable and stiffly accurate while maintaining relatively high-order accuracy \cite{kennedy2003additiveARK}. Among its six stages, the solution at the explicit first stage is directly set as the solution at the current time, whereas the solutions for the subsequent five stages are obtained implicitly. 
% Due to the diagonally implicit structure characteristic of ESDIRK methods, these implicit stages can be solved sequentially, each involving only a single unknown solution at a time. This feature reduces storage requirements and simplifies the implementation.
Owing to the diagonally implicit structure characteristic of ESDIRK methods, these implicit stages can be solved sequentially, each involving only one unknown stage at a time, thus reducing storage requirements and simplifying the implementation. An important feature of ESDIRK methods is that the converged solution from the final stage directly becomes the updated solution, offering a significant advantage in preserving solution positivity. This aspect is discussed in detail in Section~\ref{sec:PP}.
% ESDIRK4 is desgined to be $L$-stable and stiffly accurate with relatively high order of accuracy \cite{kennedy2003additiveARK}.
% The explicit first stage in ESDIRK4 directly uses the $t^n$ solution, 
% and all 5 subsequent unknown stages $\uu^{(s)}, s=2,3,\dots 6$ are implicitly solved. 
% Due to the DIRK trait of ESDIRK methods,  stages $\uu^{(s)}, s=2,3,\dots 6$ can be solved one by one successively. 
% With only one unknown stage to solve at a time, the stage solving procedure 
% is similar with a steady problem, which reduces storage and simplifies implementation.
% A significant feature of ESDIRK methods is that the converged solution from the final stage directly serves as the updated solution, providing a notable advantage in preserving its positivity. This aspect is further elaborated in Section~\ref{sec:PP}.
}


% In unsteady flow simulations, solutions are advanced in time in a step-by-step manner. Specifically, given the solution $\uu^n$ at time $t^n$, the solution at next time instance $t^{n+1}$, $\uu^{n+1}$, can be obtained by integrating the semi-discrete finite volume scheme \eqref{eq:FVODECell} over a time step $\inc t^n= t^{n+1} - t^n$ using the ESDIRK4 method as follows
\replaced[id=r1]{In unsteady flow simulations, solutions are advanced step-by-step in time. Given the solution $\uu^{n}$ at time $t^n$, the solution at the subsequent time instance $t^{n+1}$, denoted as $\uu^{n+1}$, is obtained by integrating the ODE system \eqref{eq:FVODE} over the time step $\inc t^n= t^{n+1} - t^n$ using the ESDIRK4 method:% as follows:
\begin{subequations}\label{eq:ESDIRK4}
    \begin{align}
         & \uu^{\left(1\right)} \ = \uu^{n}, \label{eq:esdirk4-1}                                                                  \\
         & \uu^{\left(s\right)} \ = \uu^{n} + \inc t^n \sum_{q=1}^{s} a_{sq} \R^{\left(q\right)},\ \ s = 2, \cdots, 6, \label{eq:esdirk4-s} \\
         & \uu^{n+1} = \uu^{\left(6\right)}, \label{eq:esdirk4-6}
    \end{align}
\end{subequations}
%\begin{equation}
%    \begin{aligned}
%    	&\uu^{n,1}_i \ = \uu^{n}_i, \\        
%        &\uu^{n,s}_i \ = \uu^{n}_i + \inc t^n \sum_{q=1}^{s} a_{sq} \R_i^{n,q},\ \ s = 2, \cdots, 6 \\
%        &\uu^{n+1}_i = \uu^{n,6}_i, \\
%    \end{aligned}
%\end{equation}
where
\begin{equation}
    \R^{\left(q\right)}  = \R \left(\uu^{\left(q\right)} \right).
\end{equation}
The coefficients $a_{sq}$ can be found
in \cite{bijl2002implicitBDFvESDIRK}. 
% It is worth noting that, although the ESDIRK4 method consists of six stages, only the final five require iterative solution to \eqref{eq:esdirk4-s} that can be rewritten as
In the final five stages of the ESDIRK4 method, the stage equation \eqref{eq:esdirk4-s}, which can be rewritten as
\begin{equation}
\label{eq:stage-equation}
    \dfrac{\uu^{\left(s\right)} - \uu^{n}}{\inc t^n}= \sum_{q=1}^{s-1} a_{sq} \R^{\left(q\right)} + a_{ss} \R^{\left(s\right)},
\end{equation}
needs to be solved iteratively, due to the implicit and nonlinear nature of $\R^{\left(s\right)}$. This is typically done using iterative methods such as the approximate-Newton iterative procedure \cite{rai1987navier} and the dual-time stepping approach \cite{jameson1991time,arnone1995integration,derango1997improvements}. Both methods employ inner (or sub-) iterations at each physical time level. 
% The dual-time stepping method is the more general approach and the approximate-Newton method can be regarded as a subset of it \cite{venkateswaran1995dual}. 
The dual-time stepping approach introduces a pseudo time derivative, in addition to the physical time derivative, to drive the solution toward steady-state at each stage. 
% The approximate-Newton method is a special case of dual-time stepping, which offers greater flexibility by allowing the pseudo time to be tuned for improved convergence \cite{venkateswaran1995dual}.
% % {\color{blue}The advantage of dual-time stepping over the approximate Newton is that the pseudo time derivative may be chosen so as to optimize the convergence of the inner iterations \cite{venkateswaran1995dual}.}
This approach can be viewed within the broader context of pseudo-transient continuation techniques, which leverage the inherent time-dependent structure of PDEs to circumvent stagnation issues associated with local minima common in nonlinear solvers \cite{kelley1998convergence}.
The approximate-Newton method can be regarded as a special case of dual-time stepping with an infinite pseudo-time step size \cite{venkateswaran1995dual}. By contrast, dual-time stepping provides greater flexibility, allowing the pseudo-time step size to be tuned for improved convergence. As a result, dual-time stepping has been widely adopted for implicit time integration in compressible flow simulations \cite{zhang2004block,jameson2009assessment,liu2018dynamic}.
}
{In this work, the time integration scheme for unsteady flow simulation is the six-stage, fourth-order explicit first stage singly diagonal implicit Runge-Kutta (ESDIRK4) \cite{bijl2002implicitBDFvESDIRK}. By integrating the ODE \eqref{eq:FVODECell} in time, the cell-average is updated by
\begin{subequations}\label{eq:ESDIRK4}
    \begin{align}
         & \uu^{n,1}_i \ = \uu^{n}_i, \label{eq:esdirk4-1}                                                                  \\
         & \uu^{n,s}_i \ = \uu^{n}_i + \inc t^n \sum_{q=1}^{s} a_{sq} \R_i^{n,q},\ \ s = 2, \cdots, 6, \label{eq:esdirk4-s} \\
         & \uu^{n+1}_i = \uu^{n,6}_i, \label{eq:esdirk4-6}
    \end{align}
\end{subequations}
%\begin{equation}
%    \begin{aligned}
%    	&\uu^{n,1}_i \ = \uu^{n}_i, \\        
%        &\uu^{n,s}_i \ = \uu^{n}_i + \inc t^n \sum_{q=1}^{s} a_{sq} \R_i^{n,q},\ \ s = 2, \cdots, 6 \\
%        &\uu^{n+1}_i = \uu^{n,6}_i, \\
%    \end{aligned}
%\end{equation}
where
\begin{equation}
    \R^{n,q}_i  = \R_i \left(t^n + c_q \inc t^n, \left\{\U^{n,q}_j\right\}_{j \in S_i}\right),
\end{equation}
with $\inc t^n= t^{n+1} - t^n$ being the time step size, which is in general a user-specified constant.
The coefficients $a_{sq}$ and $c_q$ of ESDIRK4 can be found
in \cite{bijl2002implicitBDFvESDIRK}.}

\replaced[id=r1]{
In this work, the stage equation is solved iteratively using the dual-time stepping technique. In this approach, a pseudo-time derivative is introduced alongside the physical-time derivative, modifying equation~\eqref{eq:stage-equation} into the following form:
% In this work, the stage equation is solved iteratively using the dual-time stepping technique \cite{jameson1991time, jameson2009assessment}, wherein a pseudo time derivative is introduced in addition to the physical time derivative in \eqref{eq:stage-equation}:
\begin{equation}
\label{eq:dual-time-stage-equation}
    \dfrac{\partial \uu^{\left(s\right)}}{\partial \tau} +\dfrac{\uu^{\left(s\right)} - \uu^{n}}{\inc t^n}= \sum_{q=1}^{s-1} a_{sq} \R^{\left(q\right)} + a_{ss} \R^{\left(s\right)}.
\end{equation}
An inner iteration is performed to advance the solution in the direction of pseudo time $\tau$. The intermediate solution at iteration step $m$ is denoted as $\uu^{\left(s,m\right)}$. 
The pseudo time derivative can be discretized by using a simple backward difference as 
\begin{equation}
\label{eq:backward-euler-tau}
    \dfrac{\partial \uu^{\left(s\right)}}{\partial \tau} \approx \dfrac{\uu^{\left(s,m+1\right)}-\uu^{\left(s,m\right)}}{\inc \tau^m}.
\end{equation}
The solution at pseudo time step $m+1$ can be sought by solving the nonlinear equation system 
\begin{equation}
    \dfrac{\uu^{\left(s,m+1\right)}-\uu^{\left(s,m\right)}}{\inc \tau^m} +\dfrac{\uu^{\left(s,m+1\right)} - \uu^{n}}{\inc t^n}= \sum_{q=1}^{s-1} a_{sq} \R^{\left(q\right)} + a_{ss}\R^{\left(s,m+1\right)}.
\end{equation}
Based on the fact that the implicit right-hand-side term $\R$ can be linearized as
\begin{equation}
\label{eq:linarization-R}
    \R^{\left(s,m+1\right)} \approx \R^{\left(s,m\right)} + \partialderivative{\R}{\uu} \left(\uu^{\left(s,m+1\right)}-\uu^{\left(s,m\right)}\right),
\end{equation}
the solution can be updated more conveniently by solving the following linear equation system 
\begin{equation}
    % \label{eq:pseudo-time-equation}
    \left(\frac{\eye}{\inc \tau^m} + \frac{\eye}{\inc t^n} -a_{ss}\partialderivative{\R }{\uu} \right) \inc \uu^{\left(s,m\right)}
    = \tilde{\R}^{\left(s,m\right)},
    \label{eq:linearTauUpdate}
\end{equation}
where
\begin{equation}
    \label{eq:define-inc}
    \inc \uu^{\left(s,m\right)}= 
    \uu^{\left(s,m+1\right)} - \uu^{\left(s,m\right)}, \quad 
    \tilde{\R}^{\left(s,m\right)}= 
    \sum_{q=1}^{s-1} a_{sq} \R^{\left(q\right)} + 
    a_{ss} \R^{\left(s,m\right)} - 
    \dfrac{\uu^{\left(s,m\right)} - \uu^{n}}{\inc t^n}.
\end{equation}
The linear equation system \eqref{eq:linearTauUpdate} is solved by using a matrix-free LU-SGS approach \cite{luo1998fast}. The solution is advanced in pseudo time until $\uu^{\left(s,m+1\right)}$ converges to the stage solution $\uu^{\left(s\right)}$. The convergence criterion of the inner iteration is that the $L^1$ norm of the pseudo-time derivative decreases by a certain number of orders of magnitude.
% When the pseudo time derivative vanishes, the intermediate solution $\uu^{\left(s,m+1\right)}$ converges to the stage solution $\uu^{\left(s\right)}$.
}
{The implicit and nonlinear equation \eqref{eq:esdirk4-s} is solved iteratively using a dual-time stepping technique, in which a pseudo-time variable $\tau$ is introduced. The pseudo-time integration scheme is %for  \eqref{eq:esdirk4-s} is
\begin{equation}
    \dfrac{\uu^{n,s,m+1}_i-\uu^{n,s,m}_i}{\inc \tau_i} + \dfrac{\uu^{n,s,m+1}_i - \uu^{n}_i}{\inc t^n} = \sum_{q=1}^{s-1} a_{sq} \R^{n,q}_i + a_{ss} \R_i^{n,s,m+1},
\end{equation}
where $m$ is the index for pseudo-time step, and
\begin{equation}
    \R^{n,s,m+1}_i  = \R_i \left(t^n + c_s \inc t^n, \left\{\U^{n,s,m+1}_j\right\}_{j \in S_i}\right).
\end{equation}
By using the linear approximation
\begin{equation}
    \R^{n,s,m+1}_i \approx \R^{n,s,m}_i + \sum_{j \in S_i} \dfrac{\partial \R_i}{\partial \UM_j} \left(\uu^{n,s,m+1}_j-\uu^{n,s,m}_j\right),
\end{equation}
we obtain a linear equation system to update the cell-average in the pseudo-time direction as follows
\begin{equation}
    % \label{eq:pseudo-time-equation}
    \left(\frac{\eye}{\inc \tau_i} + \frac{\eye}{\inc t^n} -a_{ss}\partialderivative{\R_i }{\uu_i} \right) \inc \uu^{n,s, m}_i
    -
    a_{ss}\sum_{j\in S_i,j\neq i} {
        \partialderivative{\R_i}{\uu_j} \inc \uu^{n,s, m}_j
    }
    = \tilde{\R}^{n,s,m}_i,
    \label{eq:linearTauUpdate}
\end{equation}
where
\begin{equation}
    \label{eq:define-inc}
    \inc \uu^{n,s,m}_i= \uu^{n,s,m + 1}_i - \uu^{n,s,m}_i, \quad \tilde{\R}^{n,s,m}_i= \sum_{q=1}^{s-1} a_{sq} \R^{n,q}_i + a_{ss} \R_i^{n,s,m} - \dfrac{\uu^{n,s,m}_i - \uu^{n}_i}{\inc t^n}.
\end{equation}
The linear equation system \eqref{eq:linearTauUpdate} is solved by using the matrix-free LU-SGS approach \cite{luo1998fast}. The solution is advanced in pseudo-time until $\uu^{n,s,m+1}_{i}$ converges to $\uu^{n,s}_{i}$, forming an inner iteration at the $s$-th stage of ESDIRK4. The convergence criterion of the inner iteration is that the $L^1$ norm of the pseudo-time derivative decreases by a certain number of orders of magnitude.
}

\replaced[id=r1]{In dual-time stepping, the physical time step size $\inc t^n$ is usually a constant specified according to the time scale of the underlying flow problem. The pseudo time step size can be adjusted based on local flow characteristics. In this work, the local pseudo time step size is computed by
\begin{equation}
    \label{eq:local-pseudo-time-step}
    \inc \tau^m_{i} = \frac{\CFLtau \overline{\OO}_i }
    {\sum_{f \in \partial \OO_i}{A_f}\lambda_{f}},
\end{equation}
where $\CFLtau$ is the CFL number used to control the convergence speed of inner iteration, and $\lambda_{f}$ is the spectral radius at cell interface $f$ defined as
\begin{equation}
    \begin{aligned}
        &\lambda_f= \lambda_{c,f} + \lambda_{v,f} A_f \left(\dfrac{1}{\overline{\OO}_L}+\dfrac{1}{\overline{\OO}_R}\right),\\
        &\lambda_{c,f}=\left|\mathbf{u}_f \cdot \n_f\right| + \sqrt{\gamma \dfrac{p_f}{\rho_f}}, \quad \lambda_{v,f}= \dfrac{\mu_f}{\rho_f} \max\left(\dfrac{4}{3},\dfrac{\gamma}{Pr}\right),
    \end{aligned}
    \label{eq:lambda-face-estimation}
\end{equation}
with the averaged primitive variables computed by
\begin{equation}
    \mathbf{W}_f= \left(\rho_f,\mathbf{u}_f,p_f\right)^\top=\dfrac{1}{2}\left[\mathbf{W} \left(\uu_L\right) + \mathbf{W} \left(\uu_R\right) \right].
\end{equation}
The dynamic viscosity on interface $f$ is computed by
\begin{equation}
    \mu_f= \mu \left(T_f\right), \quad T_f= \dfrac{p_f}{\rho_f R},
\end{equation}
using Sutherland's law \cite{white2006viscous}. 
}
{The local pseudo-time step size is computed by
\begin{equation}
    \label{eq:local-pseudo-time-step}
    \inc \tau_{i} = \frac{\CFLtau \overline{\OO}_i }
    {\sum_{f \in \partial \OO_i}{A_f}\lambda_{f}}
\end{equation}
where $\lambda_{f}$ is the spectral radius estimated on cell interface $f$ and $\CFLtau$ is the CFL number used to control the convergence speed of inner iteration.}

% \added[id=r1]
% {
% Dual time stepping technique has been widely discussed within the 
% general pseudo transient continuation framework.
% Pseudo transient continuation, 
% used for steady-state or implicit PDE solutions,
% can take advantage of the PDE's time-dependent structure  and 
% overcome the local minima stagnation problems of general nonlinear
% solving methods \cite{kelley1998convergence}.
% Due to the usage of linearized implicit Euler method in pseudo time, 
% the iteration method \eqref{eq:linearTauUpdate} is a continuation of 
% Newton-Raphson method. When $\inc\tau \rightarrow\infty$,  \eqref{eq:linearTauUpdate} becomes exactly the Newton-Raphson iteration.
% Due to the algorithmic structure of dual time stepping, the solver can be directly extended to steady-state problems efficiently, with the positivity-preserving algorithms consistently extended to steady-state solving as well.
% By simply substituting ESDIRK4 with implicit Euler and applying a infinitely large $\inc t$, the iteration \eqref{eq:linearTauUpdate} immediately becomes
% \begin{equation}
%     % \label{eq:pseudo-time-equation}
%     \left(\frac{\eye}{\inc \tau^m} -\partialderivative{\R }{\uu} \right) \inc \uu^{\left(s,m\right)}
%     = \R,
%     \label{eq:linearTauUpdate}
% \end{equation}
% which is standard pseudo-time continuation of Newton-Raphson iteration for steady problems \cite{SU2economon2016su2, FUN3Dwang2021improvements}. 
% }

% \added[id=r1]{Another important reason to use dual time stepping is to provide physical based control of inner iteration. 
% $\inc \tau$ in \eqref{eq:linearTauUpdate} can be viewed as  an implicit relaxation factor added to exact Newton-Raphson iteration.  
% By using a smaller $\inc \tau$, the increment calculated is smaller asymptotically.
% In Section \ref{sec:PP}, it will be shown that the addition of $\inc \tau$ to Newton-Raphson iteration provides a valuable opportunity to design positivity-preserving algorithms in the inner iterations.
% }

%The pseudo time step is determined locally on each cell to account for
%the different time scales in the computational region.
%The current paper uses first order LLF approximate evaluation
%of the Jacobian terms in \eeqref{eq:linearTauUpdate} and uses matrix-free
%LU-SGS to solve the increment values \cite{luo1998fast,luo2001accurate},
%which grants enough convergence in unsteady problems. 
%The pseudo time iteration is stopped when
%\begin{equation}
%    \|\FF^{n,s,m}\| \leq \varE_{\tau} \|\FF^{n,s,0}\|
%\end{equation}
%and $\FF^{n,s,m}$ is the globally assembled residual vector from 
%$\FF^{n,s,m}_i$, $\varE_{\tau}$ is the convergence threshold for
%the nonlinear residual $\FF^{n,s,m}$, and $\|\cdot\|$ is 
%a norm. For transient flow the current research finds setting $\varE_{\tau}$
%from $10^{-3}$ to $10^{-4}$ sufficiently accurate.
% !TeX root = main.tex

\begingroup
\color{r1color}

\section{Positivity-preserving algorithm}
\label{sec:PP}



Numerical methods, ranging from traditional second-order to recently developed high-order approaches, are widely employed for simulating compressible flows. However, these methods, especially the high-order ones, often suffer from reduced robustness when applied to challenging compressible flow problems. A critical robustness issue is the failure to maintain positive density or pressure, which can cause the numerical solution to blow up.

% Numerical methods, including the traditional second-order and the recently developed high-order methods, have been widely used to perform simulations of compressible flows. However, these numerical methods, especially the high-order ones, often experience reduced robustness when applied to challenging compressible flow problems. 
% A frequently encountered robustness issue is the failure to preserve positivity of density or pressure, which may cause blow-up of the numerical solution. 

% In compressible flow simulations, density and pressure need to be preserved positive, to avoid nonphysical solutions or numerical instabilities. Therefore, a positivity-preserving property is desired for numerical methods, especially high-order numerical methods, to perform robust simulations of compressible flows.

In recent decades, positivity-preserving algorithms have been extensively developed to enhance the robustness of numerical methods. However, most of these algorithms have primarily targeted numerical methods employing explicit time integration, with only a limited number of studies addressing positivity preservation in implicit numerical schemes \cite{qin2018implicit}. This scarcity arises from the inherent difficulty in verifying the positivity-preserving property of implicit methods, even for low-order schemes \cite{huang2024general}, primarily because implicit time-stepping updates solutions by iteratively solving nonlinear equation systems.

% Over recent decades, positivity-preserving algorithms have been developed to enhance the robustness of numerical methods. However, most of these algorithms are designed for numerical methods employing explicit time integration. Only a few works exist in the literature concerning the positivity-preserving property of implicit numerical methods \cite{qin2018implicit}, as it is difficult to verify if an implicit numerical scheme is positivity-preserving, even for a low-order one \cite{huang2024general}. This difficulty comes from the fact that, in implicit time stepping, the solutions are updated by solving a system of nonlinear equations iteratively. 

In this paper, we propose a positivity-preserving algorithm for arbitrary high-order implicit finite volume methods on unstructured grids, to perform robust simulations of complex compressible flows. The basic idea, detailed methodology, implementation, and theoretical analysis will be presented in the rest of this section.

%The finite volume method with implicit time marching
%is able to steadily produce numerical results in
%a wide range of flow problems.
%However, when flow conditions are extreme,
%the numerical process could fail due to
%emergence of invalid states including
%negative density and negative internal energy.
%High-order finite volume methods suffer from this issue especially,
%which could even fail due to the initial transients
%of a low Mach flow problem.
%This kind of numerical failure is referred to
%as positivity-preserving problems in the current research.
%
%The current section will introduce a series of simple
%and straightforward algorithms
%to mitigate the change of encountering positivity-preserving problems.

\subsection{Overview of the algorithm}

Positivity-preserving algorithms are employed to ensure that density and pressure remain positive throughout simulations. Formally, we define the set of admissible states as:
\begin{equation}
    G= \left\{
    \U =  \begin{pmatrix}
        \rho \\ \rho \mathbf{u} \\ \rho E
    \end{pmatrix} \middle|
    \rho >  0
    \text{ and }
    p=\left(\gamma-1\right) \left(\rho E - \dfrac{1}{2} \rho \mathbf{u} \cdot \mathbf{u}\right) >  0
    \right\}.
\end{equation}
Thus, a solution $\mathbf{U}$ is admissible if and only if $\mathbf{U} \in G$.
Given the equation of state \eqref{eq:eos}, the following relation holds:
\begin{equation}
    \mathrm{if} \ \ \rho>0, \ \mathrm{then} \ \ p>0 \ \Leftrightarrow \ e >0,  
\end{equation}
where $e = C_p T / \gamma$ is the internal energy.
From the definitions of conserved variables in \eqref{eq:def-U-F-Fv}, the density $\rho$ emerges as a linear function of $\mathbf{U}$, satisfying:
\begin{equation}\label{eq:density-linear}
\rho\left(w\mathbf{U}_1 + (1 - w)\mathbf{U}_2\right) = w\rho(\mathbf{U}_1) + (1 - w)\rho(\mathbf{U}_2), \quad 0 \leq w \leq 1.
\end{equation}
Furthermore, pressure $p$ is a concave function of $\mathbf{U}$, provided $\rho(\mathbf{U}) \geq 0$, and fulfills the inequality:
\begin{equation}
    \label{eq:pressure-Jensen}
    p \left(w\U_1+(1-w)\U_2\right)  \geq w p\left(\U_1\right) + (1-w) p \left(\U_2\right), \ 0 \leq w \leq 1, \ \mathrm{if} \ \rho\left(\U_1\right) \geq 0, \ \rho\left(\U_2\right) \geq 0.
\end{equation}
These properties of density and pressure are essential for verifying solution positivity.

% Positivity-preserving algorithms are employed to preserve positivity of density and pressure during the simulations. Accordingly, the set of admissible states is defined as
% \begin{equation}
%     G= \left\{
%     \U =  \begin{pmatrix}
%         \rho \\ \rho \mathbf{u} \\ \rho E
%     \end{pmatrix} \middle|
%     \rho >  0
%     \text{ and }
%     p=\left(\gamma-1\right) \left(\rho E - \dfrac{1}{2} \rho \mathbf{u} \cdot \mathbf{u}\right) >  0
%     \right\}.
% \end{equation}
% It is note that, given the equation of state \eqref{eq:eos}, we have the following relation
% \begin{equation}
%     \mathrm{if} \ \rho>0, \ \mathrm{then} \ p>0 \ \Leftrightarrow \ e >0,  
% \end{equation}
% where $e= C_p T /\gamma$ is the internal energy. We now present properties of the density and pressure functions that are useful in examining the positivity of a solution. 
% Density $\rho$ is a linear function of $\U$, and satisfies
% \begin{equation}
%     \label{eq:density-linear}
%     \rho\left(w\U_1+(1-w)\U_2\right) = w\rho\left(\U_1\right) + (1-w) \rho \left(\U_2\right) , \quad 0 \leq w \leq 1.
% \end{equation}
% Pressure $p$ is a concave function of $\U$ if $\rho\left(\U\right) \geq 0$, and satisfies inequality
% \begin{equation}
%     \label{eq:pressure-Jensen}
%     p \left(w\U_1+(1-w)\U_2\right)  \geq w p\left(\U_1\right) + (1-w) p \left(\U_2\right), \quad 0 \leq w \leq 1, \quad \mathrm{if} \ \rho\left(\U_1\right) \geq 0, \ \rho\left(\U_2\right) \geq 0.
% \end{equation}
% Based on the above definitions, a solution $\U$ is admissible if $\U \in G$. 

A numerical scheme is said to be positivity-preserving if it satisfies the condition:
\begin{equation}
     \mathrm{if} \ \U^{n} \in G, \ \mathrm{then} \ \U^{n+1} \in G.
\end{equation}
In a finite volume method, the numerical solution at $t^{n+1}$ is
\begin{equation}
    %	\label{eq:FVRec}
    \U^{n+1}_i (\x)= \UM^{n+1}_i + \sum_{l=1}^{\mathrm{N_b}(k)} \U_i^{n+1,l} \varphi_{i,l}(\x), \quad i=1, \cdots,N,
\end{equation}
where the degrees of freedom $\{\uu^{n+1}_i\}_{i=1}^{N}$ are updated based on the solution from the previous time step, and the reconstruction polynomial $\{\U^{n+1}_i (\x)\}_{i=1}^N$ are determined from the cell-averages $\{\uu^{n+1}_i\}_{i=1}^N$.
By using an implicit finite volume method, a solution $\U^{n+1} \in G$ can be obtained in the following two steps:
\begin{enumerate}[label=(\arabic*)]
    \item Given $\U^n \in G$, obtain $\UM^{n+1} \in G$ through implicit time integration,
    \item Given $\UM^{n+1} \in G$, obtain $\U^{n+1} \in G$ through reconstruction polynomial limiting.
\end{enumerate}
The second step can be accomplished by using the scaling limiter of Zhang et al. \cite{zhang2010positivity}. Therefore, the key of developing a positivity-preserving implicit finite volume method is to design an implicit time-stepping scheme that leads to admissible updated cell-averages.

In this work, we design an algorithm to preserve positivity of cell-averages for implicit finite volume methods. In implicit finite volume methods, the future state is obtained through iterations. For instance, in each of the final five stages of the ESDIRK4 time integration presented in Section \ref{ssec:TimeMarching}, inner iterations are performed to obtain the solution to a nonlinear equation system. The solution at next time step $\uu^{n+1}$ is the converged state of the last stage. Therefore, to obtain admissible numerical solution, we need to ensure that:
\begin{enumerate}[label=(\alph*)]
    \item the future state is admissible, i.e., $\uu^{n+1} \in G$;
    \item the intermediate states are admissible, i.e., $\uu^{\left(s,m\right)} \in G$,
\end{enumerate}
where $\uu^{\left(s,m\right)}$ is the solution at inner iteration step $m$ of stage $s$. However, the solution at next time step or next pseudo time step, is unknown at the beginning of the time step or pseudo time step, posing significant challengs in designing positivity-preserving algorithms. 
To address this challenge, we propose to estimate the solution change within one time step, based on a simple linear approximation. The estimated solution change is proportional to the time step size. Admissible solution at next time step can be guaranteed by limiting the time step size to ensure that the relative solution change is above a certain threshold, which imposes a lower bound for the estimated future state. If the lower bound is larger than the estimation error, the real future state is guaranteed to be positive.   

The strategy to limit physical time step size is briefly presented to explain the basic idea of the proposed algorithm. Given an admissible state $\uu^n \in G$, the right-hand-side of the semi-discrete finite volume scheme, $\R^n$, can be computed through solution reconstruction and flux evaluation, as described in Section \ref{sec:CFV}. The future state can be estimated as
\begin{equation}
    \uu^{n+1} \approx \uu^{n+1,*}=\uu^n + \inc t^n \R^n,
\end{equation}
which is actually a linear approximation or first-order Taylor series approximation. The allowable time step size can be obtained by constraining the relative solution change. For instance, the following condition
\begin{equation}
    % \rho \left(\uu^n + \inc t^n_{pp} \R^n\right) - \rho \left(\uu^n\right) \geq -\eta \rho \left(\uu^n\right)  \ \Leftrightarrow \ \rho \left(\uu^n + \inc t^n_{pp} \R^n\right) \geq \left(1-\eta \right) \rho \left(\uu^n\right), \ \eta \in \left(0,1\right),
    \rho \left(\uu^{n+1,*}\right) \geq \left(1-\eta_t \right) \rho \left(\uu^n\right), \ \eta_t \in \left(0,1\right),
\end{equation}
can be enforced to determine the allowable time step size for positivity of density. By choosing a sufficiently small $\eta_t$, the lower bound of estimated state is larger than the error of estimation, i.e.,
\begin{equation}
    \left(1-\eta_t \right) \rho \left(\uu^n\right) > \left|\rho\left(\uu^{n+1}\right)-\rho\left(\uu^{n+1,*}\right) \right|,
\end{equation}
then it can be guaranteed that the future state is positive, i.e.,
\begin{equation}
\begin{aligned}
    \rho(\uu^{n+1}) &= \rho(\uu^{n+1,*}) + \rho\left(\uu^{n+1}\right)-\rho\left(\uu^{n+1,*}\right)  \\
    &\geq \left(1-\eta_t \right) \rho \left(\uu^n\right) - 
    \left|\rho\left(\uu^{n+1}\right)-\rho\left(\uu^{n+1,*}\right) \right| \\
    &>0.
\end{aligned}
\end{equation}
The fundamental concept behind time step limiting is illustrated in Figure \ref{fig:dt_limiting}.

\begin{figure}[htbp!]
    \centering
    \includegraphics[width=0.8\linewidth]{pics/dt_limiting.pdf}
    \caption{Illustration on the principle of time step limiting.}
    \label{fig:dt_limiting}
\end{figure}

The time step size limiting strategy can be also implemented in the direction of pseudo time. However, it is noted that, the solution at next pseudo time step, is obtained by solving the linearized stage equation using only one iteration. As a result, the updated solution is not the solution to the linearized stage equation and thus may be far from the estimation. Therefore, the positivity of the updated solution cannot be guaranteed by only limiting the pseudo time step size. To address this issue, an increment correction, which can be viewed as a local under-relaxation, is applied to ensure admissible updated states. 

% In this paper, we propose a positivity-preserving algorithm for finite volume methods with implicit dual-time stepping. In this algorithm, the physical and pseudo time step sizes are limited by controlling the solution changes, to obtain admissible cell-averages at the next physical and pseudo time steps, respectively. 

In summary, physical time step size limiting is performed to ensure admissible future state, and pseudo time step size limiting and increment correction are performed to ensure admissible intermediate states. The flowchart of the algorithm implementation in one time step is illustrated in Figure \ref{fig:sketch}. The details of the proposed positivity-preserving algorithm will be presented in the following subsections.

\begin{figure}[htbp!]
    \centering
    \includegraphics[width=\linewidth]{pics/sketch.pdf}
    \caption{Flowchart of the proposed positivity-preserving algorithm.}
    \label{fig:sketch}
\end{figure}

\subsection{Preserving positivity of cell-averages}
\subsubsection{Physical time step limiting}
\label{ssec:physical-limiting}
A limiting procedure for physical time step size is developed to ensure that the future state is admissible. However, the future state is unknown at the beginning of a time step, posing a significant challenging in designing algorithms to enforce positivity. To overcome this challenge, we propose to make an estimation of the future state, and then impose a lower bound on the estimated state. The future state is guaranteed to be admissible if the lower bound is positive and larger than the error of estimation. An allowable physical time step size can be determined by requiring that the estimated state is not smaller than the lower bound. 

In this work, a simple linear approximation is used to estimate the future state. At time $t^n$, given an admissible state $\uu^n$, the future state $\uu^{n+1}$ is estimated as
\begin{equation}
\label{eq:linear-approximation}
    \uu^{n+1} \approx \uu^{n+1,*}=\uu^n + \inc t^n \R^n.
\end{equation}
The estimation errors of density and pressure are
\begin{equation}
    % Es \left(\rho\right) = \rho\left(\uu^{n+1}\right) - \rho\left(\uu^{n+1,*}\right)  \sim \mathcal{O} \left(\left(\inc t^n\right)^2\right), \quad Es \left(p\right) = p\left(\uu^{n+1}\right) - p\left(\uu^{n+1,*}\right)  \sim \mathcal{O} \left(\left(\inc t^n\right)^2\right).
    \begin{dcases}
        &E_{\rho} = \left|\rho\left(\uu^{n+1}\right) - \rho\left(\uu^{n+1,*}\right)\right|  \sim \mathcal{O} \left(\left(\inc t^n\right)^2\right), \\
        &E_{p} = \left|p\left(\uu^{n+1}\right) - p\left(\uu^{n+1,*}\right)\right|  \sim \mathcal{O} \left(\left(\inc t^n\right)^2\right).
    \end{dcases}
    % \rho\left(\uu^{n+1}\right) - \rho\left(\uu^{n+1,*}\right)  \sim \mathcal{O} \left(\left(\inc t^n\right)^2\right), \quad p\left(\uu^{n+1}\right) - p\left(\uu^{n+1,*}\right)  \sim \mathcal{O} \left(\left(\inc t^n\right)^2\right).
\end{equation}
We then impose lower bounds for estimated density and pressure as follows
\begin{equation}
\begin{dcases}
\label{eq:impose-lower-bounds}
    \rho\left(\uu^{n+1,*}\right) \geq \rho_{min}, \\
    p\left(\uu^{n+1,*}\right) \geq p_{min},
\end{dcases}
\end{equation}
where $\rho_{min}>0$ and $p_{min}>0$. If the lower bounds are larger than the estimation error, i.e.,
\begin{equation}
    \begin{dcases}
        \rho_{min} > E_\rho, \\
        p_{min} > E_p,
    \end{dcases}
\end{equation}
then it is derived that
\begin{equation}
    \begin{dcases}
        \rho\left(\uu^{n+1}\right) = \rho\left(\uu^{n+1,*}\right) + \rho\left(\uu^{n+1}\right) - \rho\left(\uu^{n+1,*}\right) \geq \rho_{min} - E_\rho>0, \\
        p\left(\uu^{n+1}\right) = p\left(\uu^{n+1,*}\right) + p\left(\uu^{n+1}\right) - p\left(\uu^{n+1,*}\right) \geq \rho_{min} - E_p>0,
    \end{dcases}
\end{equation}
which means that the future state is admissible, i.e., $\uu^{n+1} \in G$. Based on this analysis, the key is to design appropriate lower bounds. 

In this work, the lower bounds are set as 
\begin{equation}
\label{eq:lower-bounds}
    \begin{dcases}
        \rho_{min}= (1-\eta_t) \rho \left(\uu^n\right), \\
        p_{min}= (1-\eta_t) p \left(\uu^n\right),
    \end{dcases}
\end{equation}
where $\eta_t \in (0,1)$ is a free parameter. It is observed that, 
the lower bounds are of order $\mathcal{O}\left(1\right)$. Therefore,
it is not difficult to find suitable value of $\eta_t$ to make the lower bounds larger than the errors of estimation which are of order $\mathcal{O}\left(\left(\inc t^n\right)^2\right)$. Substituting \eqref{eq:lower-bounds} to \eqref{eq:impose-lower-bounds}, we obtain
\begin{equation}
    \begin{dcases}
        \left[\rho \left(\uu^{n+1,*}\right)- \rho \left(\uu^n\right)\right]/\rho \left(\uu^n\right) \geq -\eta_t , \\
        \left[p \left(\uu^{n+1,*}\right)- p \left(\uu^n\right)\right]/p \left(\uu^n\right) \geq -\eta_t ,
    \end{dcases}
\end{equation}
which means that the relative change of the estimated solution is constrained. Furthermore, the parameter $\eta_t$ can be used to control the allowable solution change. A larger $\eta_t$ leads to a smaller threshold for the relative solution change. The principle to select value of parameter $\eta_t$ is discussed in Section \ref{ssec:influence-parameters}.

In unsteady flow simulations using implicit time stepping, the physical time step sizes are typically specified as user-defined constants, chosen based on a priori estimates of the flow’s characteristic time scales. 
In this work, we also define a constant $\Delta t_{max}$ for each problem, serving as an upper bound on the physical time step size. At the beginning of each time step, an allowable time step size is computed, and the actual time step is taken as the minimum of the allowable and predefined values.
The allowable physical time step size is determined by enforcing lower bounds on the estimated state variables, as shown in \eqref{eq:impose-lower-bounds}. For each cell, a local time step is computed to ensure that both the density and pressure in the estimated future state remain above their respective lower bounds. The allowable global time step size is then taken as the minimum of these local time steps.


% For challenging problems, there are transient states with very low density or pressure, making it difficult for the numerical method to preserve positivity using the original time step size $\inc t_{max}$.
% This issue may be addressed by reducing the time step size temporarily. Therefore, we propose to control the physical time step size dynamically based on a positivity-preserving principle
% \begin{equation}
%     \label{eq:ppLimit}
%     %		\begin{aligned}
%     \rho\left(\uu_i^{n+1} \right)  >0, \quad
%     p\left(\uu_i^{n+1}\right) > 0 ,
%     %		\end{aligned}
% \end{equation}
% for all cells $i=1, \cdots,N$.
% However, the solution $\UM_i^{n+1}$ is unknown at the beginning of the time step. To overcome this challenge, we propose a solution estimation based on a forward Euler time discretization as follows
% \begin{equation}
%     \label{eq:estimation-dt}
%     \UM_i^{n+1} \approx  \uu_i^n + \inc t^{n}  \R_i^n.
% \end{equation}
% A time step size $\inc t^{n}_{pp}$ satisfying the positivity-preserving principle \eqref{eq:ppLimit} can be determined by limiting the solution changes as follows
% \begin{equation}
%     \label{eq:ppLimitDt}
%     \begin{aligned}
%         \rho\left(\uu_i^n + \inc t^{n}_{pp}\R_i^n\right)
%          & \geq
%         (1- \eta_t)
%         \rho\left(\uu_i^n\right)
%         >  0, 
%         % \Leftrightarrow \rho\left(\uu_i^n + \inc t^{n}_{pp}\R_i^n\right) - \rho\left(\uu_i^n\right)
%         %  & \geq -\eta_t \rho\left(\uu_i^n\right),
%          \\
%         p\left(\uu_i^n + \inc t^{n}_{pp}\R_i^n\right)
%          & \geq
%         (1- \eta_t)
%         p\left(\uu_i^n\right)
%         >  0,
%     \end{aligned}
% \end{equation}
% for all cells $i=1, \cdots,N$, where $\eta_t\in(0,1)$ is the parameter used to control the relative solution changes. A larger $\eta_t$ results in a larger allowable time step. The local truncation error of the forward Euler method \eqref{eq:estimation-dt} is approximately proportional to $\left(\inc t^{n}\right)^2$, resulting in estimation errors of density and pressure as follows 
% % $\mathcal{O}(\inc t^2)$. 
% \begin{equation}
% \label{eq:estimation-error-dt}
% \begin{aligned}
%     e_\rho= \rho\left( \UM_i^{n} + \inc t^{n}_{pp}\R_i^n \right) - \rho\left( \UM_i^{n+1}\right) \sim \mathcal{O}\left(\left(\inc t^{n}_{pp}\right)^2\right),\\ 
%     e_p= p\left( \UM_i^{n} + \inc t^{n}_{pp}\R_i^n \right) - p\left( \UM_i^{n+1}\right) \sim \mathcal{O}\left(\left(\inc t^{n}_{pp}\right)^2\right).
% \end{aligned}
% \end{equation}
% These estimation errors are significantly smaller than the allowable solution changes
% \begin{equation}
%     \begin{aligned}
%         -\eta_t \rho\left(\uu_i^n\right) \sim \mathcal{O}(1), \\
%         -\eta_t p\left(\uu_i^n\right) \sim \mathcal{O}(1),
%     \end{aligned}
% \end{equation}
% which are obtained from \eqref{eq:ppLimitDt}.
% % $\eta_t \rho\left(\uu_i^n\right) \sim \mathcal{O}(1)$ and $\eta_t p\left(\uu_i^n\right) \sim \mathcal{O}(1)$, 
% Therefore, the forward Euler time discretization is sufficiently accurate for solution estimation.
% %the positivity of $\UM_i^{n+1}$  can be preserved, provided that $\eta_t$ is not overly large.
% Based on \eqref{eq:ppLimit}, \eqref{eq:ppLimitDt} and \eqref{eq:estimation-error-dt}, we have
% \begin{equation}
%     \eta_t \geq \max \left\{\dfrac{\rho\left(\uu_i^n\right)-\rho\left( \UM_i^{n} + \inc t^{n}_{pp}\R_i^n \right)}{\rho\left(\uu_i^n\right)}, \dfrac{p\left(\uu_i^n\right)-p\left( \UM_i^{n} + \inc t^{n}_{pp}\R_i^n \right)}{p\left(\uu_i^n\right)} \right\} \sim \mathcal{O} \left(\inc t^{n}_{pp}\right),
% \end{equation}
% and
% \begin{equation}
%     \eta_t < 1- \max \left\{\dfrac{e_\rho}{\rho\left(\uu_i^n\right)}, \dfrac{e_p}{p\left(\uu_i^n\right)} \right\} \sim 1- \mathcal{O}\left(\left(\inc t^{n}_{pp}\right)^2\right),
% \end{equation}
% indicating that the parameter $\eta_t$ must not be too close to zero or one to ensure the positivity of solution $\UM_i^{n+1}$. 

The allowable local time step size for cell $i$ can be determined in two steps.
First, we assume an allowable time step size
$\alpha_{t,i}^{\rho} \inc t_{max}$, with $0 <\alpha_{t,i}^{\rho}\leq 1$, to preserve positivity of density. According to \eqref{eq:density-linear}, \eqref{eq:linear-approximation}, \eqref{eq:impose-lower-bounds} and \eqref{eq:lower-bounds}, we have
\begin{equation}
\label{eq:alpha-t-rho-ineq}
    \begin{split}
        \rho\left(\UM^n_i + \alpha_{t,i}^{\rho} \inc t_{max} \R^n_i\right) &= \left(1- \alpha_{t,i}^{\rho}\right)\rho\left(\UM^n_i\right) + \alpha_{t,i}^{\rho} \rho\left(\UM^n_i + \inc t_{max} \R^n_i\right) \\
    &\geq
    (1- \eta_t)
    \rho\left(\uu_i^n\right),
    \end{split}
\end{equation}
and thus obtain
\begin{equation}
\label{eq:alpha-t-rho}
    \alpha_{t,i}^{\rho} = \begin{dcases}
    \min\left\{1, \dfrac{-\eta_t \rho\left(\UM^n_i\right)}{\delta \rho^n_i} \right\}, & \
        \mathrm{if} \ 
        \delta \rho^n_i= \rho\left(\UM^n_i + \inc t_{max} \R^n_i\right) - \rho\left(\UM^n_i\right) < 0,
        \\
        1,  & \ \mathrm{else}.
    \end{dcases}
\end{equation}
% where 
% \begin{equation}
%     \delta \rho^n_i= \rho\left(\UM^n_i + \inc t_{max} \R^n_i\right) - \rho\left(\UM^n_i\right).
% \end{equation}
Second, we assume an allowable time step size
$\alpha_{t,i}^{p}  \alpha_{t,i}^{\rho} \inc t_{max}$, with $0<\alpha_{t,i}^{p} \leq 1$,
to preserve positivity of pressure. According to \eqref{eq:pressure-Jensen}, \eqref{eq:linear-approximation}, \eqref{eq:impose-lower-bounds} and \eqref{eq:lower-bounds}, we have
\begin{equation}
\begin{split}
    p\left(\UM^n_i +  \alpha_{t,i}^{p} \alpha_{t,i}^{\rho} \inc t_{max} \R^n_i\right) &\geq \left(1-\alpha_{t,i}^{p}\right)p\left(\UM^n_i\right) + \alpha_{t,i}^{p} p\left(\UM^n_i + \alpha_{t,i}^{\rho} \inc t_{max} \R^n_i\right) \\
    &\geq
    (1- \eta_t)
    p\left(\uu_i^n\right),
\end{split}
\end{equation}
and thus obtain a sufficient condition
\begin{equation}
\label{eq:alpha-t-p}
    \alpha_{t,i}^{p} = \begin{dcases}
        \min\left\{1, \dfrac{-\eta_t p \left(\UM^n_i\right)}{\delta p_i^n} \right\}, & \ 
        \mathrm{if} \ \delta p_i^n=  p \left(\UM^n_i + \alpha_{t,i}^{\rho} \inc t_{max} \R^n_i\right) -  p \left(\UM^n_i\right) <0, \\
        1, & \ \mathrm{else}.
    \end{dcases}
\end{equation}
Once these two steps are completed, we determine the allowable local time step size for cell $i$ as $\Delta t^n_i= \alpha_{t,i}^{p}  \alpha_{t,i}^{\rho} \inc t_{max}$.
The global time step size is computed by
\begin{equation}
    \label{eq:alpha-t-global}
    \inc t^n = \min_i \left\{\Delta t^n_i \right\}.
\end{equation}
% where
% \begin{equation}
%     \alpha_t = \min_i(\alpha_{t,i}^{\rho}\alpha_{t,i}^{p}).
% \end{equation}
% It can be observed from the above equation that $\inc t^n \leq \inc t_{max}$, as $0<\alpha_t \leq 1$, indicating an upper bound of the global physical time step size.
From the equation above, we observe that $\Delta t^n \leq \Delta t_{\text{max}}$ since $\Delta t^n_i \leq \Delta t_{max}$. This establishes an upper bound for the global physical time step size.
An analysis on the lower bound of the physical time step size is presented in Section \ref{ssec:analysis-time-step}. It is proved that, 
\begin{equation}
    \dfrac{\inc t^n}{\inc t_{max}} \sim \mathcal{O} \left(1\right),
\end{equation}
which means that the limited physical time step size $\inc t^n$ is not infinitely small compared to $\inc t_{max}$.

% \subsubsection{Positive intermediate states}

\subsubsection{Pseudo time step limiting}
\label{ssec:pseudo-limiting}

The physical time step limiting strategy described in the previous subsection is used to obtain an admissible state $\UM^{n+1} \in G$, which is the converged solution of the final stage of ESDIRK4 \eqref{eq:ESDIRK4}. However, intermediate solutions with negative density or pressure may arise during the inner iterations of each stage. To address this issue, we propose to control solution changes by limiting pseudo-time step sizes during the inner iteration process, following a similar limiting strategy to that used in the physical-time direction described in Section \ref{ssec:physical-limiting}. It is important to note, however, that since local time steps are used in the pseudo-time direction, the limiting is applied in an element-wise manner only.

For the $m$-th inner iteration at the $s$-th stage of ESDIRK4, we have the following positivity-preserving requirement
\begin{equation}
    \label{eq:ppLimit-Tau}
    %		\begin{aligned}
    \rho\left(\uu_i^{\left(s,m+1\right)} \right)>0, \quad
    p\left(\uu_i^{\left(s,m+1\right)}\right)>0,
    %		\end{aligned}
\end{equation}
for all cells $i=1,\cdots,N$. However, $\UM^{\left(s,m+1\right)}_i$ is unknown at the beginning of the pseudo-time step. Therefore, we propose to use a linear approximation to estimate the solution at next pseudo-time step. Based on the dual-time discretized stage equation \eqref{eq:dual-time-stage-equation}, the pseudo-time derivative on cell $i$ can be approximated as:
\begin{equation}
    \dfrac{\uu^{\left(s,m+1\right)}_i-\uu^{\left(s,m\right)}_i}{\inc \tau_i} \approx \sum_{q=1}^{s-1} a_{sq} \R^{\left(q\right)}_i + a_{ss} \R_i^{\left(s,m\right)} - \dfrac{\uu^{\left(s,m\right)}_i - \uu^{n}_i}{\inc t^n}= \tilde{\R}^{\left(s,m\right)}_i ,
\end{equation}
which leads to a linear approximation
\begin{equation}
    \label{eq:estimation-dtau}
    \UM_i^{\left(s,m+1\right)} \approx  \UM_i^{\left(s,m\right)} + \inc \tau_i  \tilde{\R}^{\left(s,m\right)}_i,
\end{equation}
where the local time step size $\Delta \tau_i$ is computed by using \eqref{eq:local-pseudo-time-step}.
% with $ \tilde{\R}_i^{n,s,m}$ defined in \eqref{eq:define-inc}.
An allowable pseudo-time step size $ \inc \tau_{i,pp}$ can be determined by imposing lower bounds on the estimated density and pressure as follows
\begin{equation}
    \label{eq:ppLimitDtau}
    \begin{aligned}
        \rho\left(\uu_i^{\left(s,m\right)} +  \inc \tau_{i,pp} \tilde{\R}_i^{\left(s,m\right)}\right)
         & \geq
        (1- \eta_\tau)
        \rho\left(\uu_i^{\left(s,m\right)}\right)
        , \\
        p\left(\uu_i^{\left(s,m\right)} + \inc  \tau_{i,pp} \tilde{\R}_i^{\left(s,m\right)}\right)
         & \geq
        (1- \eta_\tau)
        p\left(\uu_i^{\left(s,m\right)}\right)
        ,
    \end{aligned}
\end{equation}
where $\eta_\tau \in (0,1)$ is the parameter used to control the relative solution changes.

The allowable pseudo-time step size can be determined according to \eqref{eq:ppLimitDtau}
using the same two-step procedure presented in Section \ref{ssec:physical-limiting}.
The allowable time step size is 
\begin{equation}
    \label{eq:alpha-tau}
    \inc \tau_{i,pp}= \alpha_{\tau,i}^{p}\alpha_{\tau,i}^{\rho} \inc  \tau_{i},
\end{equation}
where
%\begin{equation}
%	\rho\left(\UM^n_i + \alpha_{t,i}^{\rho} \inc t_{max} \R^n_i\right)= \left(1- \alpha_{t,i}^{\rho}\right)\rho\left(\UM^n_i\right) + \alpha_{t,i}^{\rho} \rho\left(\UM^n_i + \inc t_{max} \R^n_i\right)  \geq
%	(1- \eta_t)
%	\rho\left(\uu_i^n\right),
%\end{equation}
%and thus obtain
\begin{equation}
    \label{eq:alpha-tau-rho}
    \alpha_{\tau,i}^{\rho} = \begin{dcases}
        \min\left\{1, \dfrac{-\eta_\tau \rho\left(\UM^{\left(s,m\right)}_i\right)}{\delta \rho_i^{\left(s,m\right)}} \right\}, & \mathrm{if} \ \delta \rho_i^{\left(s,m\right)}= \rho\left(\UM^{\left(s,m\right)}_i + \inc \tau_{i} \tilde{\R}^{\left(s,m\right)}_i\right) - \rho\left(\UM^{\left(s,m\right)}_i\right) < 0, \\
        1, & \mathrm{else},
    \end{dcases}
\end{equation}
and
\begin{equation}
    \label{eq:alpha-tau-p}
    \alpha_{\tau,i}^{p} = \begin{dcases}
        \min\left\{1, \dfrac{-\eta_\tau p \left(\UM^{\left(s,m\right)}_i\right)}{\delta p_i^{\left(s,m\right)}} \right\}, & \mathrm{if} \ \delta p_i^{\left(s,m\right)} = p \left(\UM^{\left(s,m\right)}_i + \alpha_{\tau,i}^{\rho} \inc \tau_{i} \tilde{\R}^{\left(s,m\right)}_i\right) -p \left(\UM^{\left(s,m\right)}_i\right) < 0, \\
        1, &\mathrm{else}.
    \end{dcases}
\end{equation}

It is noted that, at each pseudo-time step, the updated intermediate solution is obtained through a single LU-SGS iteration. Consequently, this solution does not represent the fully converged state of the linear system described by \eqref{eq:linearTauUpdate}. As a result, the accuracy of the estimation in \eqref{eq:estimation-dtau} cannot be theoretically guaranteed, nor can the positivity of the updated solution. %$\UM^{n,s,m+1}_i$.
To overcome this limitation, an increment correction will be introduced in the next subsection.

\subsubsection{Increment correction}
\label{ssec:increment-correction}

As pointed out in the previous subsection, even with limited physical and pseudo time step sizes, the intermediate results may exhibit negative density or pressure prior to the convergence of the inner iteration.
Therefore, a pseudo-time increment correction is designed to ensure the positivity of the updated intermediate solutions. The increment correction is performed in a cell-by-cell manner.
For cell $i$, the increment $\inc \uu_i^{\left(s,m\right)}$ is corrected based on the following conditions
\begin{equation}
    \label{eq:ppLimitInc}
    \begin{aligned}
        \rho\left(\uu_i^{\left(s,m\right)} + \alpha_{\inc,i} \inc \uu_i^{\left(s,m\right)}\right)
         & \geq
        (1- \eta_\inc)
        \rho\left(\uu_i^{\left(s,m\right)}\right), \\
        p\left(\uu_i^{\left(s,m\right)} + \alpha_{\inc,i} \inc \uu_i^{\left(s,m\right)}\right)
         & \geq
        (1- \eta_\inc)
        p\left(\uu_i^{\left(s,m\right)}\right),
    \end{aligned}
\end{equation}
where $\eta_\inc \in (0,1)$ is a relaxation parameter to control the relative solution changes.
By using the same two-step approach used in physical/pseudo time step limiting, the relaxation parameter is computed by
\begin{equation}
\label{eq:alpha-inc}
    \alpha_{\inc,i}= \alpha_{\inc,i}^{\rho} \alpha_{\inc,i}^{p},
\end{equation}
where
\begin{equation}
\label{eq:alpha-inc-rho}
    \alpha_{\inc,i}^{\rho} = \begin{dcases}
        \min\left\{1, \dfrac{-\eta_\inc \rho\left(\UM^{\left(s,m\right)}_i\right)}{\delta \rho_{\inc,i}} \right\}, & \mathrm{if} \ \delta \rho_{\inc,i} = \rho\left(\UM^{\left(s,m\right)}_i +\inc \uu_i^{\left(s,m\right)}\right)  -\rho\left(\UM^{n,s,m}_i\right) < 0, \\
        1, &
        \mathrm{else},
    \end{dcases}
\end{equation}
and
\begin{equation}
\label{eq:alpha-inc-p}
    \alpha_{\inc,i}^{p} = \begin{dcases}
        \min\left\{1, \dfrac{-\eta_\inc p \left(\UM^{\left(s,m\right)}_i\right)}{\delta p_{\inc,i}} \right\}, & \mathrm{if} \ \delta p_{\inc,i} = \ p \left(\UM^{\left(s,m\right)}_i + \alpha_{\inc,i}^{\rho} \inc \uu_i^{\left(s,m\right)}\right) - p \left(\UM^{\left(s,m\right)}_i\right) < 0, \\
        1, &
        \mathrm{else}.
    \end{dcases}
\end{equation}
A solution $\uu^{\left(s,m+1\right)}_i \in G$ is then computed by
\begin{equation}
\label{eq:increment-correction}
    \uu_i^{\left(s,m+1\right)} = \uu_i^{\left(s,m\right)} + \alpha_{\inc,i} \inc \uu_i^{\left(s,m\right)}.
\end{equation}

% By using the increment correction, the intermediate states and hence the converged state, can be guaranteed to be positive. As mentioned in Section \ref{ssec:TimeMarching}, the ESDIRK4 method is used to perform implicit time integration in this work. A notable feature of the ESDIRK methods is that, the last stage gives the solution at the new time level. This means that, if the last stage converges, the converged state is the solution at the new time level, which can be guaranteed to be admissible.
By applying increment correction, the positivity of both intermediate states and the final converged state can be ensured. As discussed in Section~\ref{ssec:TimeMarching}, the ESDIRK4 method is employed for implicit time integration in this work. A notable feature of ESDIRK methods is that the final stage directly yields the solution at the new time level. Consequently, once the last stage converges, the resulting state becomes the updated solution and is guaranteed to be admissible. This offers a distinct advantage of ESDIRK methods over other implicit Runge–Kutta schemes, where additional reconstruction or projection steps may be required.

\subsection{Preserving positivity of reconstruction polynomials}
\label{ssec:rec-pp-limiter}

The density and pressure at quadrature points, interpolated using high-order reconstruction polynomials, may not remain positive despite the positivity of the cell-averaged values. To address this issue, we employ the scaling limiter developed by Zhang et al. \cite{zhang2010maximum,zhang2010positivity,zhang2012positivity}, which ensures positive density and pressure distributions within control volumes, given admissible cell-averages. Additionally, this scaling limiter is proven to be accuracy-preserving \cite{zhang2010positivity}.

% The indices $n$,$s$ and $m$ are omitted here for all the discussions concerning reconstruction limiting happens within each implicit iteration and only the current mean values $\uu_i^{n,s,m}$  and reconstruction polynomials $\Ulim_i^{n,s,m}$of that iteration are considered. 

To prevent the interpolated point values from being too close to zero, which may cause numerical difficulties, we introduce a new admissible set $G^\varepsilon \subset G$ defined as
\begin{equation}
    G^\varepsilon= \left\{
    \U =  \begin{pmatrix}
        \rho \\ \rho \mathbf{u} \\ \rho E
    \end{pmatrix} \middle|
    \rho  \geq \varepsilon_\rho
    \text{ and }
    p=\left(\gamma-1\right) \left(\rho E - \dfrac{1}{2} \rho \mathbf{u} \cdot \mathbf{u}\right) \geq \varepsilon_p
    \right\},
\end{equation}
where $\varepsilon_\rho$ and $\varepsilon_p$ are problem-specific lower bounds computed by
\begin{equation}\label{eq:epsilons}
    \varepsilon_\rho=\min\left\{\varepsilon \rho_0, \min_i\{\rho(\uu_i)\}\right\}, \quad 
    \varepsilon_p=\min\left\{\varepsilon p_0, \min_i\{p(\uu_i)\}\right\}.
\end{equation}
Here $\rho_0$ and $p_0$ are characteristic density and pressure, respectively, and $\varepsilon=10^{-11}$. The definitions in \eqref{eq:epsilons} inherently ensure that the cell-averages are included in the admissible set, i.e., $\uu_i \in G^\varepsilon$.

By using the scaling limiter, the reconstruction polynomial on control volume $\Omega_i$
\begin{equation}
    \U_{i} \left(\x\right) = \uu_{i} + \sum_{l=1}^{\mathrm{N_b}(k)}{\U_i^l\varphi_{i,l}(\x)},
\end{equation}
is compressed as
\begin{equation}
\label{eq:polynomial-scaling}
    \hat{\U}_{i} \left(\x\right) = \uu_{i} + \beta_i \sum_{l=1}^{\mathrm{N_b}(k)}{\U_i^l\varphi_{i,l}(\x)}, \quad 0 \leq \beta_i \leq 1,
\end{equation}
to ensure an admissible state at each quadrature point, i.e.,
\begin{equation}
    \hat{\U}_{i} \left(\x_{i,g}\right) \in G^{\varepsilon},
\end{equation}
where $\x_{i,g}$ denotes a quadrature point for volume or surface integral. We refer to \cite{zhang2010positivity} for the detailed computation of $\beta_i$. 
It is proved in \cite{zhang2010positivity} that, the scaling limiter preserves accuracy in smooth regions.

\subsection{Algorithm implementation}

The proposed positivity-preserving algorithm employs a physical time step limiting procedure to determine a time step size that ensures an admissible future state. Additionally, it applies a pseudo time step limiting procedure and an increment correction procedure to maintain the positivity of intermediate states. Together, these procedures guarantee the positivity of the updated cell averages. Given admissible cell averages, a scaling limiter is then used to preserve the positivity of the reconstruction polynomials.
The implementation of the positivity-preserving algorithm for finite volume schemes with ESDIRK4 time integration is detailed in Algorithm \ref{alg:pp-algorithm}. An intuitive illustration of this algorithm is provided in Figure \ref{fig:sketch}.

As shown in Algorithm \ref{alg:pp-algorithm}, the implementation does not require additional vector or matrix allocations, thereby incurring no extra memory cost. Furthermore, as discussed in the computational cost analysis in Section \ref{ssec:sedov}, the positivity-preserving algorithm accounts for only $10\%$ of the total CPU time when reconstruction polynomial scaling is applied to more than $20\%$ of the cells, demonstrating the algorithm’s computational efficiency.

% The positivity-preserving algorithm developed in this paper is based on time step limiting, increment correction and reconstruction polynomial scaling.

% The propose positivity-preserving algorithm uses a physical time step limiting procedure to determine a time step size that leads to admissible future state; a pseudo time step limiting procedure and an increment correction to ensure positivity of intermediate states. These together guarantees the positivity of updated cell-averages. Given admissible cell-averages, a scaling limiter is used to preserve positivity of the reconstruction polynomials. 

% The implementation of the positivity-preserving algorithm for the finite volume schemes using ESDIRK4 time integration, as illustrated in Figure \ref{fig:sketch}, is described in detail in Algorithm \ref{alg:pp-algorithm}.

% As observed in Algorithm \ref{alg:pp-algorithm}, no additional vectors or matrices are allocated during its implementation, ensuring no extra memory cost. Furthermore, according to the computational cost analysis in Section \ref{ssec:sedov}, the positivity-preserving algorithm accounts for only 10\% of the total CPU time when the reconstruction polynomial scaling is activated on more than $20\%$ of the cells, demonstrating computational efficiency of the algorithm.

 \begin{algorithm}[htbp!]
        \renewcommand{\baselinestretch}{1.3}\selectfont
 	\caption{Positivity-preserving algorithm for FV schemes using ESDIRK4 time integration.}  
 	\label{alg:pp-algorithm}
 	\begin{algorithmic}[1] 
 		%		\Require Array 
 		%		\Ensure no
 		% \Function{$\left[ \left\{\UM_i^{n+1}\right\}_{i=1}^N, \Delta t^n \right]$=PPFV}{$\left\{\UM_i^{n}\right\}_{i=1}^N, \Delta t_{max}, \mathrm{CFL}_\tau,\eta_t, \ \eta_\tau,\eta_\Delta, \varepsilon_\rho, \varepsilon_p,n_{iter}$}
        \Function{$\left[ \UM^{n+1}, \Delta t^n \right]$=PPFV}{$\UM^{n}, \Delta t_{max}, \mathrm{CFL}_\tau,\eta_t, \eta_\tau,\eta_\Delta, \varepsilon_\rho, \varepsilon_p,n_{iter}$}
        
        \State Perform reconstruction (and limiting if needed) of solution $\UM^{n}$ %at time step $n$ %$\left\{\UM_i^{n}\right\}_{i=1}^N$
                
        \State Compute numerical flux integrals to obtain the right-hand-side of \eqref{eq:Semi-FV} %integrals to obtain residuals $\left\{\R_i^{n}\right\}_{i=1}^N$
        \State Compute allowable physical time step size for each cell according to \eqref{eq:alpha-t-rho} and \eqref{eq:alpha-t-p}
        \State Compute allowable global physical time step size $\Delta t^n$ according to \eqref{eq:alpha-t-global}
        \State Perform the explicit first stage of ESDIRK4 by setting $\UM^{\left(1\right)} =\UM^{n}$ %, for $i=1, \cdots,N$
        \For {$s= 2, 6$}

        \State Initialize intermediate solution by setting $\UM^{\left(s,1\right)}=\UM^{\left(s-1\right)}$ 
        
        \For {$m=1, n_{iter}$}

        \State Perform reconstruction (and limiting if needed) of solution $\UM^{\left(s,m\right)}$ %at time step $n$ %$\left\{\UM_i^{n}\right\}_{i=1}^N$
        \State Compute numerical flux integrals to obtain the right-hand-side of \eqref{eq:linearTauUpdate} %integrals to obtain residuals $\left\{\R_i^{n}\right\}_{i=1}^N$
        
        \State Compute local pseudo-time step size for each cell according to \eqref{eq:local-pseudo-time-step}

        \State Compute allowable local pseudo-time step size for each cell according to \eqref{eq:alpha-tau} to \eqref{eq:alpha-tau-p}
        \State Determine solution increments by solving \eqref{eq:linearTauUpdate} using the LU-SGS approach  %$\left\{\Delta \UM_i^{n,s,m}\right\}^N_{i=1}$

        \State Correct the increment for each cell according to \eqref{eq:alpha-inc} to \eqref{eq:alpha-inc-p}

        \State Obtain updated intermediate solution $\UM^{\left(s,m+1\right)}$ according to \eqref{eq:increment-correction} %$\UM_i^{n,s,m+1}=\UM_i^{n,s,m} + \Delta \UM_i^{n,s,m}$

        \State Exit the inner iteration if convergence is reached
        %the $L_1$-norm of the pseudo-time derivative has decreased by a certain number of orders of magnitude
		\EndFor
        \State Take the converged solution as the solution of the stage by setting $\UM^{\left(s\right)}=\UM^{\left(s,m+1\right)}$ %, for $i=1, \cdots,N$
		\EndFor
 	
        \State Obtain the solution at time step $n+1$ by setting $\UM^{n+1} =\UM^{\left(6\right)}$ %, for $i=1, \cdots,N$

        % \State Perform reconstruction (and limiting if needed) of solution $\UM^{n+1}$ %at time step $n$ %$\left\{\UM_i^{n}\right\}_{i=1}^N$
 	\EndFunction 
 		
 		% \item[]
 		
 		% \Function {$\left[ \mathbf{u}_i, \mathbf{u}_{j_1},  \mathbf{u}_{j_2},  \mathbf{u}_{j_3} \right] $=VR}{$S_i, \overline{u}_{i}, \overline{u}_{j_1}, \overline{u}_{j_2}, \overline{u}_{j_3}, u_0, f_{\text{NN}}$} 
 		
 		% \State Compute the derivative weights on cell interfaces using \eqref{eqn:Derivative_weights_computation}
 		
 		% \State  Compute the elements of the reconstruction matrix $\mathbf{A}$ and right-hand-side $\mathbf{b}$ corresponding to cost function \eqref{eqn:forward_cost_function} using formulae \eqref{eqn:Matrix_Elements} and \eqref{eqn:DLU}
 		
 		% \State Compute $\mathbf{u}= \left\{\mathbf{u}_i, \mathbf{u}_{j_1},  \mathbf{u}_{j_2},  \mathbf{u}_{j_3} \right\}$ by solving the linear equation system $\mathbf{A} \mathbf{u}= \mathbf{b}$
 		
 		% \EndFunction 
 		
 	\end{algorithmic}  
 \end{algorithm}  

 

\subsection{Application to steady-state problems}
\label{ssec:application-to-steady-state}

The numerical solution of a steady-state compressible flow is the solution to the following nonlinear equation
\begin{equation}
% \label{eq:dual-time-stage-equation}
    0 = \R \left(\uu\right).
\end{equation}
In the framework of dual time stepping, this equation is solved by advancing the solution to the following equation
\begin{equation}
% \label{eq:dual-time-stage-equation}
    \dfrac{\partial \uu}{\partial \tau} = \R \left(\uu\right),
\end{equation}
in the direction of pseudo time $\tau$, until convergence. By employing the backward Euler difference for pseudo time derivative
\begin{equation}
    \dfrac{\partial \uu}{\partial \tau} \approx \dfrac{\uu^{(m+1)}-\uu^{(m)}}{\inc t^m},
\end{equation}
and the linear approximation for residual
\begin{equation}
    \R^{(m+1)} \approx \R^{(m)} + \dfrac{\partial \R}{\partial \uu} \left(\uu^{(m+1)}-\uu^{(m)}\right),
\end{equation}
the intermediate state can be updated by solving the following linear equation system
\begin{equation}
    % \label{eq:pseudo-time-equation}
    \left(\frac{\eye}{\inc \tau^m} -\partialderivative{\R }{\uu} \right) \inc \uu^{\left(m\right)}
    = \R^{\left(m\right)},
    % \label{eq:linearTauUpdate}
\end{equation}
where $m$ denotes the iteration step. 

Towards an admissible converged state, the cell-averages and the reconstruction polynomials need to be guaranteed positive at each iteration step. The pseudo time step limiting in Section \ref{ssec:pseudo-limiting}, and the increment correction in Section \ref{ssec:increment-correction}, can be applied to preserve positivity of the intermediate states. Given positive cell-averages, the scaling limiter is used to ensure admissible states at quadrature points on cell interfaces. 




\endgroup
% !TeX root = main.tex
\begingroup
\color{r2color}

\section{Analysis of the positivity-preserving algorithm}

This section analyzes the properties of the positivity-preserving algorithm introduced in Section \ref{sec:PP}, including its accuracy-preserving behavior, sensitivity to parameter selection, bounds on the time step size, and overall applicability.

\subsection{Accuracy-preserving property}

The positivity-preserving algorithm consists of four fundamental procedures: physical time step limiting, pseudo time step limiting, increment correction, and reconstruction polynomial scaling. In this subsection, we analyze the impact of these procedures on the accuracy of the overall implicit finite volume method.

It is well-established that reducing the physical time step size does not impact the order of accuracy of the time integration.
% Generally, pseudo time step limiting and increment correction may be activated during the early stages of the inner iteration to stabilize the solution and are deactivated in later stages as the solution converges. 
% {\color{r1color}Even if pseudo-time step limiting and increment correction remain active, the unchanged convergence criterion ensures that the converged solution satisfies the nonlinear equation system with sufficient accuracy.}
% Therefore, these techniques do not affect the accuracy of the converged solution in the inner iteration process.
In general, pseudo time step limiting and increment correction are often activated during the initial stages of the inner iteration to stabilize the solution. These techniques are typically deactivated in later stages as the solution converges. Importantly, even if pseudo time step limiting and increment correction remain active, the unchanged convergence criterion ensures that the converged solution satisfies the nonlinear equation system with sufficient accuracy. As a result, these stabilization techniques in the inner iteration process do not compromise the accuracy of the converged solution.
Furthermore, as noted earlier, the scaling limiter effectively preserves the 
accuracy of the reconstruction in smooth regions \cite{zhang2010positivity}.

In conclusion, the positivity-preserving algorithm preserves accuracy in both space and time. The accuracy-preserving property is demonstrated through an accuracy test on a smooth vortex problem in Section \ref{ssec:accuracy-test}.

\subsection{Parameter sensitivity}
\label{ssec:influence-parameters}

The positivity-preserving algorithm includes three important parameters, $\eta_t$, $\eta_{\tau}$, and $\eta_{\inc}$, which are used to control the allowable relative changes in the solution. These parameters are all defined in the range $(0,1)$. Larger values of $\eta_t$ and $\eta_{\tau}$ result in larger physical and pseudo time step sizes, respectively. Larger values of $\eta_{\inc}$ results in larger increments, which may lead to faster convergence in later iteration steps while higher stiffness in earlier iteration steps. 

There is an upper bound for $\eta_t$ beyond which the time step limiting may fail to guarantee positivity of the future state, which is the solution at the next time level. As mentioned in Section \ref{ssec:physical-limiting}, to ensure positivity of the future state, a lower bound is imposed on the estimated future state, and the lower bound must be larger than the estimation error. An analysis of the positivity preservation of density is performed to show the properties of the upper bound of $\eta_t$. The estimated future state is computed by using a simple linear approximation as below:
\begin{equation} 
    \uu^{n+1,*}=\uu^n + \inc t^n \R^n,
\end{equation}
which has an error of order $\mathcal{O}\left(\left(\inc t^n\right)^2\right)$.
As any time integration scheme with at least 
1st order accuracy has truncation error 
of order $\mathcal{O}\left(\left(\inc t^n\right)^k\right),k\geq2$
(ESDIRK4's error being $\mathcal{O}\left(\left(\inc t^n\right)^5\right)$),
the error between real numerical implicit solution and estimated
solution is still $\mathcal{O}\left(\left(\inc t^n\right)^2\right)$.
It is observed from \eqref{eq:alpha-t-rho} that, the limited time step size $\inc t^n$ is proportional to $\eta_t$ and $\inc t_{max}$. Therefore, the estimation error can be expressed as
\begin{equation}
    E_\rho= K_\rho \left(\eta_t\right)^2 \rho\left(\uu^n\right) , \quad K_\rho \sim \mathcal{O}\left(\left(\inc t_{max}\right)^2\right).
\end{equation}
A lower bound is imposed on the density of the estimated future state, i.e.,
\begin{equation}
    \rho\left(\uu^{n+1,*}\right) \geq \rho_{min}= \left(1-\eta_t\right) \rho\left(\uu^n\right).
\end{equation}
To ensure the positivity of the future state, it is required that
\begin{equation}
    \rho_{min}= \left(1-\eta_t\right) \rho\left(\uu^n\right) > E_\rho= K_\rho \left(\eta_t\right)^2 \rho\left(\uu^n\right),
\end{equation}
which is simplified as 
\begin{equation}
    1-\eta_t > K_\rho \left(\eta_t\right)^2.
\end{equation}
Therefore, there is a critical upper bound $\eta^c_t$ that satisfies 
\begin{equation}
    1-\eta^c_t = K_\rho \left(\eta^c_t\right)^2
    % ,\ \ 
    % \eta_t^c = \frac{\sqrt{1 + 4K_\rho}-1}{2K_\rho}\approx1-K_\rho
\end{equation}
as shown in Figure \ref{fig:eta-t-curves}.
The parameter $\eta_t$ should be selected in the range $\left(0,\eta^c_t\right)$. 
% It is observed from Figure \ref{fig:eta-t-curves} that, 
It is obtained through analysis that, 
a smaller $K_\rho$ yields a larger $\eta^c_t$,
which can also be observed from Figure \ref{fig:eta-t-curves}. 
As $K_\rho$ is of 
order $\mathcal{O}\left(\left(\inc t_{max}\right)^2\right)$
and $1-\eta_t^c\sim K_\rho$,
the upper bound of $\eta_t$ is close to one, allowing an easy parameter selection. 
A similar analysis can be performed to determine the upper bound of $\eta_t$ that can further ensure positivity of pressure. 
In actual implementation, it is difficult to compute the solution dependent coefficient $K_\rho$, thus not feasible to obtain the exact upper bound $\eta_t$. 
Meanwhile, the real future state is 
obtained through a series of positivity-preserving inner iterations,
thus the only criterion for detecting an inadmissible real future 
state is the divergence of inner iterations.
It is observed in numerical results, when physical time step is 
too large (using no limiting for example), the solution might   
fail to converge and the norm of residual blows up. 
In the diverged scenarios, due to the usage of pseudo time limiting 
and increment correction, the latest solution is always admissible, 
but exponentially growing pressure or density is observed, rendering the 
solution non-physical. 
Based on the fact that the upper bound is quite close to one, an 
iterative posteriori approach can be used to determine $\eta_t$ as follows:
% \begin{enumerate} [label=(\alph*)]
%     \item try a parameter value $\eta^0_t$ to perform one time step computation;
%     \item if (a) fails, go back to the initial state, try with a halved parameter value;
%     \item if (b) fails, try (b) again.
% \end{enumerate}
\begin{enumerate} [label=(\alph*)]
    \item initialize the parameter as $\eta_t= \eta^0_t \in \left(0,1\right)$;
    \item compute a limited time step size $\inc t^n$, and then perform one step time integration to obtain stage values $\uu^{(s)},s=2,3\dots6$ and updated solution $\uu^{n+1}$;
    \item if $\uu^{(s)},s=2,3\dots6$ with $\uu^{(6)}=\uu^{n+1}$ all converged, proceed to next step; otherwise, try (b) again with a halved $\eta_t$.
\end{enumerate}
Since the upper bound $\eta^c_t$ is of order $\mathcal{O}(1)$, a value of $\eta_t$ that preserves positivity can always be found within a finite number of iterations. 
In the numerical experiments presented in Section~\ref{sec:Results}, the parameter is initialized as $\eta_t = 0.8$ for all test cases. 
This choice does not lead to solution divergence and therefore the iterative posteriori search for this parameter is not triggered.

\begin{figure}[htbp!]
    \centering
    \includegraphics[width=0.45\linewidth]{pics/eta-t-curves.pdf}
    \caption{Lower and upper bounds of the parameter $\eta_t$.}
    \label{fig:eta-t-curves}
\end{figure}

% There is not an upper bound for the parameter $\eta_\tau$ to be used to preserve positivity, since the pseudo time step limiting is followed by an increment correction that can guarantee positivity of the updated intermediate states. There is also not an upper bound for the parameter $\eta_{\inc}$, as the lower bounds are directly imposed on the updated intermediate states.
There is no upper bound required for the parameter $\eta_\tau$ to preserve positivity, 
as the pseudo-time step limiting is followed by an increment correction 
that ensures the positivity of the updated intermediate states. 
Similarly, no upper bound is needed for the parameter $\eta_{\inc}$, 
since the lower bounds are directly enforced on the updated intermediate states. 
However, there exists an asymptotic relation between $\eta_t$ and $\eta_\inc$,
which is discussed in Section \ref{ssec:analysis-time-step}.
A preferable choice is to make $\eta_\inc > \eta_\tau$, so that,
asymptotically, increment correction does not take effect after 
pseudo time limiting and primarily act as a safe guard.
In summary, the current work 
chooses both $\eta_\tau$ and $\eta_{\inc}$ to be within the range $(0,1)$, 
with $\eta_\inc > \eta_\tau$ required. 

\subsection{Time step size bounds}
\label{ssec:analysis-time-step}
In this subsection, we analyze the upper and lower bounds of the physical and pseudo time step sizes. As mentioned in Section \ref{ssec:physical-limiting}, the upper bound of the limited physical time step is the user-defined constant $\inc t_{max}$.
It is observed from \eqref{eq:alpha-tau} that, the upper bound of the limited pseudo time step for cell $i$ is the original pseudo time step $\inc \tau_i$, which is computed according to \eqref{eq:local-pseudo-time-step}. 
The following of this subsection analyzes the lower bounds
of the physical time step and pseudo time step after positivity-preserving limiting. 
Along with the pseudo time step's lower bound, the effect of 
increment correction is also investigated.

\subsubsection{Lower bound for physical time step limiting}

An analysis can be performed on the lower bound of the physical time step size.
The right-hand side of the semi-discrete finite volume scheme is computed by
\begin{equation}
    \label{eq:Semi-FV-1}
    \R_i = -\frac{1}{\overline{\OO}_i} \oint_{\partial \OO_i} \left(\tilde{\F} - \tilde{\F}_v \right) \cdot \n \ \dd A.
\end{equation}
On a cell interface $f$, a spectral radius $\tilde{\lambda}_f>0$ can be found such that
\begin{equation}
    \left\| \left(\tilde{\F} - \tilde{\F}_v \right) \cdot \n \right\| \leq \tilde{\lambda}_f \min \left\{\|\UM_L\|, \|\UM_R\| \right\},
\end{equation}
where $\|\cdot\|$ denotes the $L^2$ norm, if the ratio $\max \left\{\|\UM_L\|, \|\UM_R\| \right\}/\min \left\{\|\UM_L\|, \|\UM_R\| \right\}$ is bounded.
Therefore, we have
\begin{equation}
\label{eq:upper-bound-residual}
    \left\| \R_i \right\| 
    \leq \frac{1}{\overline{\OO}_i} \sum_{f \in \partial \Omega_i}{
        A_{f} \tilde{\lambda}_{f} \min \left\{\|\UM_L\|, \|\UM_R\| \right\}
    } 
    \leq \frac{\sum_{f \in \partial \Omega_i} A_f\tilde{\lambda}_f}{\overline{\OO}_i}\|\UM_i\|
    =
    \frac{\CFL_i}{\inc t_{max}}\|\UM_i\|,
\end{equation}
where $\CFL_i= \inc t_{max}\sum_{f \in \partial \Omega_i} A_f \tilde{\lambda}_f/\overline{\OO}_i$. 
Given that $\UM_i$ is finite and non-singular, we can find $\mathcal{O}(1)$ coefficients $C^{\rho}_i$, $C^{\rho E}_i$ and $C^{\rho \uv}_i$ such that
\begin{equation}
    \begin{aligned}
    \label{eq:inc_upper_cond_const_coef}
        \left| \rho\left( \inc t_{max} \R_i \right) \right|
        & \leq
        C^\rho_i    \CFL_i \rho\left( \UM_i \right),\\
        \left| \rho E\left( \inc t_{max} \R_i \right) \right|
        & \leq
        C^{\rho E}_i \CFL_i \rho E\left( \UM_i \right),\\
        \left\| \rho \uv\left( \inc t_{max} \R_i \right) \right\|
        & \leq 
        C^{\rho \uv}_i \CFL_i  \left\| \rho \uv\left( \UM_i \right)              \right\|,\\
    \end{aligned}
\end{equation}
where $\rho(\U)$, $\rho E(\U)$ and $\rho \uv(\U)$ are linear functions as they take directly the components of $\U$.
% $C^{\rho}_i$, $C^{\rho E}_i$ and $C^{\rho \uv}_i$ are $\mathcal{O}(1)$ coefficients determined by $\UM_i$.
We can show that $\alpha_{t,i}^{\rho}$ has a lower bound.
By substituting the following inequality 
\begin{equation}
    |\delta \rho_i^n| = \left|\rho\left(\UM^n_i + \inc t_{max} \R^n_i\right) - \rho\left(\UM^n_i\right)\right|
    =
    \left|\rho\left(\inc t_{max} \R^n_i\right)\right|
    \leq
    \CFL_i C^\rho_i \rho\left(\UM^n_i\right),
\end{equation}
into \eqref{eq:alpha-t-rho}, we have
\begin{equation}
    \label{eq:alpha-t-rho-lb}
    \alpha_{t,i}^{\rho} \geq \frac{\eta_t}{\CFL_i C_i^\rho},
\end{equation}
which indicates a finite lower bound for $\alpha_{t,i}^{\rho}$. 

\newcommand{\uincT}{\inc\U_i^{n,\rho}}
Next, we derive a lower bound for $\alpha_{t,i}^{p}$ using a similar approach.
We define $\inc\U_i^{n,\rho} =\alpha_{t,i}^{\rho} \inc t_{max} \R^n_i$.
According to \eqref{eq:alpha-t-rho-ineq}, we have 
\begin{equation}
    \rho\left(\UM^n_i+\uincT  \right) \geq (1-\eta_t)\rho\left(\UM^n_i\right).
\end{equation}
By applying the Cauchy-Schwarz inequality and the Triangle inequality, we obtain
% {\small
\begin{equation}
    \begin{aligned}
        \dfrac{|\delta p_i^n|}{\gamma-1} 
        &=  \dfrac{1}{\gamma-1} \left| p \left(\UM^n_i + \inc\U_i^{n,\rho}\right) -  p \left(\UM^n_i\right) \right| \\
        &=  \left| \rho E\left( \uincT \right) 
         - \frac{
         \rho\uv\left(\uincT\right)^2
         +
         \rho\uv\left(\UM^n_i\right)^2
         +
         2\rho\uv\left(\uincT\right)\cdot\rho\uv\left(\UM^n_i\right)
         }{
         2\rho\left(\UM^n_i+\uincT  \right)
         }
         + \frac{ \rho\uv\left(\UM^n_i\right)^2}{2\rho\left(\UM^n_i\right)}
        \right| \\
        &\leq  \left| \rho E\left( \uincT \right) \right|
         +
         \dfrac{
         \rho\uv\left(\uincT\right)^2
         +
         \rho\uv\left(\UM^n_i\right)^2
         +
         2\rho\uv\left(\uincT\right)\cdot\rho\uv\left(\UM^n_i\right)
         }{
         2\rho\left(\UM^n_i+\uincT  \right)
         }
         +
          \frac{ \rho\uv\left(\UM^n_i\right)^2}{2\rho\left(\UM^n_i\right)} \\
        &\leq  \left| \rho E\left( \uincT \right) \right|
         +
         \frac{1}{1-\eta_t}
         \frac{
         \rho\uv\left(\uincT\right)^2
         +
         \rho\uv\left(\UM^n_i\right)^2
         +
         2\rho\uv\left(\uincT\right)\cdot\rho\uv\left(\UM^n_i\right)
         }{
         2\rho\left(\UM^n_i \right)
         }
         +
          \frac{ \rho\uv\left(\UM^n_i\right)^2}{2\rho\left(\UM^n_i\right)} \\
        &\leq \alpha_{t,i}^\rho\CFL_i C^{\rho E}_i  \rho E\left( \UM^n_i \right) 
        +\left(
        \frac{(\alpha_{t,i}^\rho\CFL_i C^{\rho \uv}_i)^2}{1-\eta_t}
        +\frac{2\alpha_{t,i}^\rho\CFL_i C^{\rho \uv}_i}{1-\eta_t}
        + \frac{2-\eta_t}{1-\eta_t}
        \right)
        \frac{ \rho\uv\left(\UM^n_i\right)^2}{2\rho\left(\UM^n_i\right)},
    \end{aligned}
\end{equation}
% }
where $\rho\uv()^2$ is the short form of $\rho\uv()\cdot\rho\uv()$.
As $p\left(\UM^n_i\right) > 0$, we have
\begin{equation}
    \frac{ \rho\uv\left(\UM^n_i\right)^2}{2\rho\left(\UM^n_i\right)} < \rho E\left(\UM^n_i\right),
\end{equation}
and thus
\begin{equation}
    \begin{aligned}
        |\delta p_i^n| &\leq
        (\gamma-1)\left[
         \alpha_{t,i}^\rho\CFL_i C^{\rho E}_i  
        +\left(
        \frac{(\alpha_{t,i}^\rho\CFL_i C^{\rho \uv}_i)^2}{1-\eta_t}
        +\frac{2\alpha_{t,i}^\rho\CFL_i C^{\rho \uv}_i}{1-\eta_t}
        + \frac{2-\eta_t}{1-\eta_t}
        \right)
        \right]
        \rho E\left( \UM^n_i \right).
    \end{aligned}
\end{equation}
According to \eqref{eq:alpha-t-p}, we obtain
\begin{equation}
    \begin{aligned}
        \label{eq:alpha-t-p-lb}
        \alpha_{t,i}^{p} 
        & \geq \dfrac{\eta_t}{\gamma-1} \frac{p\left(\UM^n_i\right)}{
            \rho E\left(\UM^n_i\right)
        }
        \frac{1}{
         \alpha_{t,i}^\rho\CFL_i C^{\rho E}_i  
            +\left(
            \frac{(\alpha_{t,i}^\rho\CFL_i C^{\rho \uv}_i)^2}{1-\eta_t}
            +\frac{2\alpha_{t,i}^\rho\CFL_i C^{\rho \uv}_i}{1-\eta_t}
            + \frac{2-\eta_t}{1-\eta_t}
            \right)
        } \\
        & = \eta_t \frac{1}{
           1 + \frac{\gamma (\gamma-1) Ma_i^2}{2}
        }
        \frac{1}{
         \alpha_{t,i}^\rho\CFL_i C^{\rho E}_i  
            +\left(
            \frac{(\alpha_{t,i}^\rho\CFL_i C^{\rho \uv}_i)^2}{1-\eta_t}
            +\frac{2\alpha_{t,i}^\rho\CFL_i C^{\rho \uv}_i}{1-\eta_t}
            + \frac{2-\eta_t}{1-\eta_t}
            \right)
        },
    \end{aligned}
\end{equation}
where $Ma_i$ is the Mach number based on $\UM^n_i$. 
Consequently, the global time step size is bounded below by
\begin{equation}
    \label{eq:delta-t-lb}
    \frac{\inc t^n}{\inc t_{max}} \geq 
    \eta_t^2
    \min_i
    \left\{
    \frac{1}{
           1 + \frac{\gamma (\gamma-1) Ma_i^2}{2}
        }
    \frac{1}{\CFL_i C_i^\rho}
    \frac{1}{
     \alpha_{t,i}^\rho\CFL_i C^{\rho E}_i  
        +\left(
        \frac{(\alpha_{t,i}^\rho\CFL_i C^{\rho \uv}_i)^2}{1-\eta_t}
        +\frac{2\alpha_{t,i}^\rho\CFL_i C^{\rho \uv}_i}{1-\eta_t}
        + \frac{2-\eta_t}{1-\eta_t}
        \right)
    }
    \right\}.
\end{equation}
We assume that the state $\UM^n_i$ yields a finite Mach number $Ma_i$.
Under the conditions that $\eta_t\sim\mathcal{O}(1)$, $1-\eta_t\sim\mathcal{O}(1)$,
$\CFL_i\sim\mathcal{O}(1)$ and ${1}/\left(1 + \frac{\gamma (\gamma-1) Ma_i^2}{2}\right)\sim\mathcal{O}(1)$,
$\inc t^n$ is not infinitely small compared to $\inc t_{max}$.
In other words, when the CFL number $\CFL_i$ based on $\inc t_{max}$ is $\mathcal{O}(1)$, 
the CFL number determined by the scaled $\inc t^n$ also remains $\mathcal{O}(1)$.

\subsubsection{Lower bound for pseudo time step limiting}

A similar analysis can be carried out for the lower bounds of the local pseudo-time step sizes. In this case, the residual $\R_i$ is simply replaced with $\tilde{\R}_i^{(s,m)}$, for which an inequality analogous to \eqref{eq:upper-bound-residual} can be derived in the same spirit. 

For $\inc\tau$ analysis, 
the implicit time-stepping  for stage $s$ in \eqref{eq:define-inc}
has the form of
\begin{equation}
    \begin{aligned}
        \tilde{\R}_i\supsm & = \sum_{q=1}^{s-1} a_{sq} \R_i^{\left(q\right)} + 
    a_{ss} \R_i\supsm - 
    \dfrac{\uu_i\supsm - \uu_i^{n}}{\inc t^n} \\
    & =  a_{ss} \R_i\supsm - 
    \dfrac{\uu_i\supsm}{\inc t^n}
    + \hat{\R}_i,
    \end{aligned}
\end{equation}
where $\hat{\R}_i$ is the constant term independent of the unknown
$\uu_i\supsm$. 
Therefore, \eqref{eq:upper-bound-residual} becomes
\begin{equation}
\label{eq:upper-bound-residual-tau}
\begin{aligned}
    \left\| \tilde{\R}_i \right\| 
    \leq & \frac{a_{ss}}{\overline{\OO}_i} \sum_{f \in \partial \Omega_i}{
        A_{f} \tilde{\lambda}_{f} \min \left\{\|\UM_L\supsm\|, \|\UM_R\supsm\| \right\}
    } 
    + \dfrac{\left\|\uu_i\supsm\right\|}{\inc t^n}
    + \|\hat{\R}_i\|
    \\
    \leq & \left(
    \frac{1}{\inc t^n} + 
    \frac{a_{ss}\sum_{f \in \partial \Omega_i} A_f\tilde{\lambda}_f}{\overline{\OO}_i}
    \right)
    \|\UM_i\supsm\|
    + \|\hat{\R}_i\|\\
    = &
    \left(\frac{1}{\inc t^n} + \frac{a_{ss}\CFL_\tau}{\inc \tau_i} \right)\|\UM_i\supsm\| 
    + \|\hat{\R}_i\|\\
    = &
     \CFL_\tau\frac{1 / \CFL_i + a_{ss}}{\inc \tau_i}
     \|\UM_i\supsm\| 
    + \|\hat{\R}_i\|,
\end{aligned}
\end{equation}
with $\CFL_i$ being the CFL number corresponding to the limited
physical time step
$\inc t^n$.
Similar to \eqref{eq:inc_upper_cond_const_coef}, 
the residual value for each conservative variable can be 
bounded by the current values $\UM_i\supsm$ with additional 
intercept values $\left| \rho\left(\hat\R_i\right) \right|$, 
$\left | \rho E\left(\hat\R_i\right) \right|$, 
$\left\|\rho\uv\left(\hat\R_i\right) \right\|$. 
Non-singularity of the current state $\UM_i\supsm$ leads to 
\begin{equation}
    \begin{aligned}
    \label{eq:inc_upper_cond_const_coef_tau}
        \left| \rho\left( \inc\tau_i \tilde\R_i \right) \right|
        & \leq
        C^{\rho,\tau}_i \CFL_\tau(1 / \CFL_i + a_{ss})
        \rho\left( \UM_i\supsm \right)
        + \left| \rho\left(\inc\tau_i\hat\R_i\right) \right|,\\
        \left| \rho E\left(\inc\tau_i \tilde\R_i \right) \right|
        & \leq
        C^{\rho E,\tau}_i \CFL_\tau(1 / \CFL_i + a_{ss})
        \rho E\left( \UM_i\supsm \right)
        + \left | \rho E\left(\inc\tau_i\hat\R_i\right) \right|,\\
        \left\| \rho \uv\left( \inc\tau_i \tilde\R_i \right) \right\|
        & \leq
        C^{\rho \uv,\tau}_i \CFL_\tau(1 / \CFL_i + a_{ss})
        \left\| \rho \uv\left( \UM_i\supsm \right)              \right\| + \left\|\rho\uv\left(\inc\tau_i\hat\R_i\right) \right\|\\
    \end{aligned}
\end{equation}
with coefficients $C^{\rho ,\tau}_i$,$C^{\rho E,\tau}_i$ and $C^{\rho \uv,\tau}_i$
of order $\ord(1)$.
Making
\begin{equation}
    \begin{aligned}
    \label{eq:inc_upper_cond_const_coef_tau_recast_method}
        D^{\rho,\tau}_i
        & =
        C^{\rho,\tau}_i (1 / \CFL_i + a_{ss})
        + \frac{\left| \rho\left(\inc\tau_i \hat\R_i\right) \right|}
        { \CFL_\tau\rho\left( \UM_i\supsm \right)},
        \\
        D^{\rho E,\tau}_i
        & =
        C^{\rho E,\tau}_i (1 / \CFL_i + a_{ss})
        + \frac{\left| \rho E\left(\inc\tau_i \hat\R_i\right) \right|}
        { \CFL_\tau\rho E\left( \UM_i\supsm \right)},
        \\
        D^{\rho \uv,\tau}_i
        & =
        C^{\rho \uv,\tau}_i (1 / \CFL_i + a_{ss})
        + \frac{\left\|\rho\uv\left(\inc\tau_i \hat\R_i\right) \right\|}
        { \CFL_\tau\left\|\rho \uv\left( \UM_i\supsm \right)\right\|}
        ,\\
    \end{aligned}
\end{equation}
the inequalities \eqref{eq:inc_upper_cond_const_coef_tau} can be 
rewritten into 
\begin{equation}
    \begin{aligned}
    \label{eq:inc_upper_cond_const_coef_tau_recasted}
        \left| \rho\left( \inc\tau_i \tilde\R_i \right) \right|
        & \leq
        D^{\rho,\tau}_i \CFL_\tau
        \rho\left( \UM_i\supsm \right)
        ,\\
        \left| \rho E\left(\inc\tau_i \tilde\R_i \right) \right|
        & \leq
        D^{\rho E,\tau}_i \CFL_\tau
        \rho E\left( \UM_i\supsm \right)
        ,\\
        \left\| \rho \uv\left( \inc\tau_i \tilde\R_i \right) \right\|
        & \leq
        D^{\rho \uv,\tau}_i \CFL_\tau
        \left\| \rho \uv\left( \UM_i\supsm \right)              \right\| 
        ,\\
    \end{aligned}
\end{equation}
which is of the same form as \eqref{eq:inc_upper_cond_const_coef}.
The constant term  $\hat{\R}_i$ is
\begin{equation}
        \hat{\R}_i  = \sum_{q=1}^{s-1} a_{sq} \R_i^{\left(q\right)} +
    \dfrac{\uu_i^{n}}{\inc t^n} 
\end{equation}
and therefore
\begin{equation}
        \inc\tau_i\hat{\R}_i  = \inc\tau_i\sum_{q=1}^{s-1} a_{sq} \R_i^{\left(q\right)} +
    \dfrac{\CFL_\tau}{\CFL_i}\uu_i^{n} 
\end{equation}
where $\CFL_i$ is determined by the already limited $\inc t^n$.
Assuming the previous stage RHS terms are non-singular, while 
last step solution $\UM_i^n$ and 
latest current stage solution $\UM_i\supsm$ are non-singular, clearly the 
non-dimensional ratios 
$\frac{\left| \rho\left(\inc\tau_i \hat\R_i\right) \right|}
{ \CFL_\tau\rho\left( \UM_i\supsm \right)}$,
$\frac{\left| \rho E\left(\inc\tau_i \hat\R_i\right) \right|}
{ \CFL_\tau\rho E\left( \UM_i\supsm \right)}$ and 
$\frac{\left\|\rho\uv\left(\inc\tau_i \hat\R_i\right) \right\|}
{ \CFL_\tau\left\|\rho \uv\left( \UM_i\supsm \right)\right\|}$
are of order $\ord(1)$. 
Moreover, we also know that physical and pseudo-time CFL number $\CFL_i$, $\CFL_\tau$ and 
Runge-Kutta coefficient $a_{ii}$.
In summary, the new coefficients $D^{\rho,\tau}_i$, $D^{\rho E,\tau}_i$ 
and $D^{\rho \uv,\tau}_i$ defined in \eqref{eq:inc_upper_cond_const_coef_tau_recast_method}
are of order $\ord(1)$. 

As the inequalities \eqref{eq:inc_upper_cond_const_coef_tau_recasted}
are of the same form as \eqref{eq:inc_upper_cond_const_coef}, 
subsequent analysis remains the same as that for $\inc t$ analysis. 
It can be derived that similar to $\inc t$, the CFL number 
for $\inc \tau _i$ after limiting is still of order $\ord(1)$.
In other words,
under the same assumptions we used to analyze physical time step limiting,
the limited pseudo time step size $\inc \tau_{i,pp}$ is not infinitely small compared to $\inc \tau_i$.

% Assuming that $a_{ss}$ and $\CFL_i$ are $\ord(1)$, 
% while obviously constant term $\hat{\R}_i$ has $\ord(1)$ value, 
% a non-singular $\UM_i\supsm$ leads to finite constant coefficients in 
% \eqref{eq:inc_upper_cond_const_coef_tau}. 
% The only difference between \eqref{eq:inc_upper_cond_const_coef} and 
% \eqref{eq:inc_upper_cond_const_coef_tau} is that \eqref{eq:inc_upper_cond_const_coef_tau}
% has additional intercept values on the right hand side.

% The remainder of the analysis remains unchanged, as a linear approximation is still used to estimate the future state in the pseudo-time direction.
% The lower bounds of $\alpha_{\tau,i}^\rho$ and $\alpha_{\tau,i}^{p}$ are similar to \eqref{eq:alpha-t-rho-lb} and \eqref{eq:alpha-t-p-lb}.
% The additional constant intercept values in \eqref{eq:inc_upper_cond_const_coef_tau} will lead to additional constant values added to the denominators $\alpha_{\tau,i}^\rho$ and $\alpha_{\tau,i}^{p}$'s lower bounds. 
% However, it has be shown that any additional constants added to the denominators
% will be of $\ord(1)$ magnitude, which does not affect the qualitative result of 
% the lower bounds.
% It can be concluded that, the limited pseudo time step size $\inc \tau_{i,pp}$ is not infinitely small compared to $\inc \tau_i$.

\subsubsection{Analysis of increment correction}

A rigorous analysis of the lower bound of the relaxation factor $\alpha_{\inc,i}$ in the increment correction, computed according to \eqref{eq:alpha-inc}–\eqref{eq:alpha-inc-p}, is not feasible. 
This is because the increment of the intermediate state is obtained using LU-SGS approximation, and therefore does not represent the exact solution of the linearized system \eqref{eq:linearTauUpdate}. 
The exact increment calculated by LU-SGS relies on 
the ordering of cells, which is hard to analyze 
for a general case.
As a result, the discrepancy between the theoretical and actual increments cannot be accurately estimated. 
Nevertheless, since the increment is computed using a limited pseudo-time step size, it remains inherently bounded. 

Consider the increment defined in the linear system \eqref{eq:linearTauUpdate},
which can be rewritten to a more precise expression
\begin{equation}
    % \label{eq:pseudo-time-equation}
    \left(\sum_i\left(\frac{\eye_{c,i}}{\inc \tau^m_{i,pp}}\right) + \frac{\eye}{\inc t^n} -a_{ss}\partialderivative{\R }{\uu} \right) \inc \uu^{\left(s,m\right)}
    = \tilde{\R}^{\left(s,m\right)},
    \label{eq:linearTauUpdate_re}
\end{equation}
where $\eye_{c,i}$ is 
the identity matrix for cell $i$ and zero for others.
Then, the LU-SGS leads to the increment being
\begin{equation}
    \begin{aligned}
        \inc \uu^{\left(s,m\right)}
     = &
    \left(\sum_i\left(\frac{\eye_{c,i}}{\inc \tau^m_{i,pp}}\right) + \frac{\eye}{\inc t^n} -a_{ss}(D+U) \right)^{-1}
    \left(\sum_i\left(\frac{\eye_{c,i}}{\inc \tau^m_{i,pp}}\right) + \frac{\eye}{\inc t^n} -a_{ss}D \right)\\
    &
    \left(\sum_i\left(\frac{\eye_{c,i}}{\inc \tau^m_{i,pp}}\right) + \frac{\eye}{\inc t^n} -a_{ss}(D+L) \right)^{-1}
    \tilde{\R}^{\left(s,m\right)} \\
    = & (\hat{D}+\hat{U})^{-1}\hat{D}(\hat{D}+\hat{L})^{-1}
    \tilde{\R}^{\left(s,m\right)} , 
    \end{aligned}
    \label{eq:linearTauUpdate_re_lusgs}
\end{equation}
where $L$, $U$ and $D$ are strict block-lower, strict block-upper and block-diagonal 
parts of Jacobian $\partialderivative{\R }{\uu}$ and $\partialderivative{\R }{\uu} = L + D + U$.
The $\hat L$, $\hat U$ and $\hat D$ are strict block-lower, strict block upper and block-diagonal 
parts of the Jacobian $-\partialderivative{\tilde\R }{\uu}$ in \eqref{eq:linearTauUpdate_re}.
Taking partial derivative of $\inc\tau^m_{i,pp}$ on \eqref{eq:linearTauUpdate_re_lusgs} leads to
\begin{equation}
    \begin{aligned}
       \pdv{\inc \uu^{\left(s,m\right)}}{\inc\tau^m_{i,pp}}
    = & \frac{1}{\left(\inc\tau^m_{i,pp}\right)^2}
    (\hat{D}+\hat{U})^{-1}\eye_{c,i}(\hat{D}+\hat{U})^{-1}\hat{D}(\hat{D}+\hat{L})^{-1}
    \tilde{\R}^{\left(s,m\right)} \\
    - & \frac{1}{\left(\inc\tau^m_{i,pp}\right)^2}
    (\hat{D}+\hat{U})^{-1}\eye_{c,i}(\hat{D}+\hat{L})^{-1}
    \tilde{\R}^{\left(s,m\right)} \\
    + & \frac{1}{\left(\inc\tau^m_{i,pp}\right)^2}
    (\hat{D}+\hat{U})^{-1}\hat{D}(\hat{D}+\hat{L})^{-1}\eye_{c,i}(\hat{D}+\hat{L})^{-1}
    \tilde{\R}^{\left(s,m\right)}.
    \end{aligned}
    \label{eq:linearTauUpdate_re_lusgs_pdvtau}
\end{equation}
When $\inc\tau^m_{i,pp}\rightarrow 0^+$, clearly 
$\hat D $, $\hat D + \hat L$ and $\hat D + \hat U$ all 
approximate
$\sum_i\left(\frac{\eye_{c,i}}{\inc \tau^m_{i,pp}}\right) $,
and the inverse matrices  $(\hat D + \hat L)^{-1}$ and 
$(\hat D + \hat U)^{-1}$ approximate $\sum_i\inc\tau^m_{i,pp}\eye_{c,i}$,
therefore
\begin{equation}
    \begin{aligned}
       \lim_{\inc\tau^m_{j,pp}\rightarrow 0^+, \forall j}\pdv{\inc \uu^{\left(s,m\right)}}{\inc\tau^m_{i,pp}}
    = & \eye_{c,i}
    \tilde{\R}^{\left(s,m\right)}.
    \end{aligned}
    \label{eq:linearTauUpdate_re_lusgs_pdvtau_limit}
\end{equation}
This limit means that although LU-SGS does not 
produce precise solution of the linearized system, 
it is consistent with the continuous local pseudo time evolution. 
From \eqref{eq:linearTauUpdate_re_lusgs_pdvtau_limit},
there is 
\begin{equation}
    \inc \uu^{\left(s,m\right)}
    =
    \sum_i{\inc\tau^m_{i,pp}\eye_{c,i}} \tilde\R^{\left(s,m\right)}
    +
    \ord\left(\left(\inc\tau^m_{i,pp}\right)^2\right)
\end{equation}
and for each cell
\begin{equation}
\label{eq:linearTauUpdate_lusgs_expand_tau}
    \inc \uu^{\left(s,m\right)}_i
    =
    \inc\tau^m_{i,pp}
    \tilde\R^{\left(s,m\right)}_i
    +
    \ord\left(\left(\inc\tau^m_{i,pp}\right)^2\right)
\end{equation}

Considering the limit in \eqref{eq:linearTauUpdate_re_lusgs_pdvtau_limit} 
and \eqref{eq:linearTauUpdate_lusgs_expand_tau}
when $\inc\tau_i\rightarrow 0$,
the difference between LU-SGS increment and 
Taylor expansion term \eqref{eq:estimation-dtau} used in pseudo time step limiting
is of order $\ord(\inc\tau^2)$. 
Meanwhile, when we use $\eta_\inc > \eta_\tau$,
and $\eta_\inc - \eta_\tau \sim \ord(1)$,
the margin between the 
lower bound of 
increment correction and pseudo time limiting
would be 
\begin{equation}
\rho_{min,\tau} - \rho_{min,\inc}= 
\rho\left(\uu^{(m)}\right) (\eta_\inc - \eta_\tau),\ \ 
p_{min,\tau} - p_{min,\inc}= 
p\left(\uu^{(m)}\right) (\eta_\inc - \eta_\tau).
\end{equation}
When pseudo time step limiting or increment correction is triggered,
the current state should be close enough to zero density or 
pressure state by the order of $\eta\inc\tau\tilde\R$.
It would be appropriate to assume 
$p\sim\ord(\inc\tau), \rho\sim\ord(1)$ 
or $p\sim\ord(1), \rho\sim\ord(\inc\tau)$
or $p\sim\ord(\inc\tau), \rho\sim\ord(\inc\tau)$
is true.
As a result,
\begin{equation}
\rho_{min,\tau} - \rho_{min,\inc} \gtrsim \ord (\inc \tau),\ \ 
p_{min,\tau} - p_{min,\inc} \gtrsim \ord (\inc \tau),
\end{equation}
which indicates asymptotically
the lower bound for density or pressure in pseudo time step limiting 
is far larger than that of increment correction, 
with a margin large enough to tolerate the error between 
linear estimation and real LU-SGS increment expressed in \eqref{eq:linearTauUpdate_lusgs_expand_tau}.
This means after pseudo time limiting, 
the increment is already positivity-preserving 
in an asymptotic manner without the involvement of 
increment limiting.

In practical computations, we use a finite $\inc \tau$, which 
does not guarantee positivity-preserving without increment correction.
However, when the solution converges, the 
residual $\tilde\R$ and increment value $\inc\UM$ both tends to zero,
which does not induce pseudo time limiting or increment correction.
The increment correction is typically activated during the early part of the inner iterations and tends to become inactive in later inner iterations. 


\subsection{Applicability of the algorithm}
% Finally, we analyze the applicability of the proposed positivity-preserving algorithm. 
% This algorithm is applicable to general implicit finite volume methods on unstructured grids for compressible flow simulations. While there is no restriction on the spatial discretization, an implicit dual-time discretization is essential.
% This algorithm is applicable to a broad range of implicit finite volume methods on unstructured grids for compressible flow simulations. Although the spatial discretization can be arbitrary, the use of an implicit dual-time stepping approach is essential for its effective implementation.

% The proposed positivity-preserving algorithm is applicable to a broad range of implicit finite volume methods on unstructured grids for compressible flow simulations. As the scaling limiter is designed to preserve positivity of reconstruction polynomials given admissible cell averages, thus any polynomial reconstruction is compatible with the proposed positivity-preserving algorithm.
% The procedures to preserve positivity of cell averages are designed in the dual-time stepping framework. Therefore, the use of an implicit dual-time stepping approach is essential for applying the positivity-preserving algorithm. There is not restrictions on the time integration scheme, any implicit time integration scheme can be used in theory. However, we need to highlight that, the ESDIRK schemes are well-suited for the proposed positivity-preserving algorithm. Compared with other high-order implicit time integration methods, such as the singly-diagonally implicit Runge–Kutta (SDIRK) methods, the $L$-stable ESDIRK methods are more robust. Furthermore, the solution at next time level of ESDIRK is taken as the converged state of the final stage, which is well-suited for our positivity-preserving algorithm, in which the intermediate states are guaranteed admissible. 

% The proposed positivity-preserving algorithm is applicable to a wide range of implicit finite volume methods on unstructured grids for compressible flow simulations. Regarding the spatial discretization, the only requirement is that a polynomial reconstruction needs to be used, since the scaling limiter is designed to preserve the positivity of reconstruction polynomials given admissible cell averages.

% Regarding time discretization, positivity preservation for cell averages is achieved within the dual-time stepping framework, making the use of an implicit dual-time stepping approach essential. Although there are no theoretical restrictions on the choice of time integration schemes and any implicit method can be used in principle, we emphasize that ESDIRK schemes are particularly well-suited for the proposed algorithm. Compared with other high-order implicit methods, such as singly diagonally implicit Runge–Kutta (SDIRK) schemes, the ESDIRK methods which are $L$-stable, offer improved robustness. Furthermore, the solution at the next time level in ESDIRK methods is defined as the converged state of the final stage, which aligns naturally with our positivity-preserving algorithm, as admissibility is maintained throughout the intermediate stages.

The proposed positivity-preserving algorithm is applicable to a wide range of implicit finite volume methods on unstructured grids for compressible flow simulations. For spatial discretization, the only requirement is the use of a polynomial reconstruction, since the scaling limiter is specifically designed to preserve the positivity of reconstructed polynomials given admissible cell averages.

For time discretization, positivity preservation of cell averages is achieved within a dual-time stepping framework, which makes the use of an implicit dual-time stepping approach essential. Although there are no theoretical restrictions on the choice of time integration scheme and any implicit method can be used in principle, we emphasize that ESDIRK schemes are particularly well-suited to the proposed algorithm. Compared to other high-order implicit methods, such as singly diagonally implicit Runge–Kutta (SDIRK) schemes, ESDIRK methods that are $L$-stable offer improved robustness. Moreover, the solution at the next time level in ESDIRK methods corresponds to the converged state of the final stage. This naturally aligns with the design of our algorithm, as admissibility is preserved throughout all intermediate stages.



\endgroup

% !TeX root = main.tex

\section{Numerical results}
\label{sec:Results}

\replaced[id=r1]{The proposed positivity-preserving algorithm is validated through a series of benchmark test cases. Most of these focus on inviscid flows, while three cases consider viscous flows, such as the hypersonic flow past an open cavity, which features a hypersonic boundary layer. The numerical scheme employed is a fourth-order accurate implicit finite volume method, which uses a cubic variational reconstruction to achieve fourth-order accuracy in space and an ESDIRK4 scheme to achieve fourth-order accuracy in time.}
{The proposed positivity-preserving algorithm is verified using a series of benchmark test cases.
The numerical scheme used is the fourth-order accurate implicit finite volume method based on a
cubic variational reconstruction and an ESDIRK4 time integration.}
In the simulations, the relaxation parameters are set as $\eta_t= 0.8$, $\eta_\tau=0.5$ and $\eta_\inc=0.9$,
unless otherwise specified.
\replaced[id=r1]{The convergence criterion for inner iteration is that
the $L^1$ norm of the pseudo-time derivative of density (referred to as the $L^1$ density residual in the convergence history plots) decreases by three orders of magnitude, unless otherwise specified.}
{The convergence criterion for inner iteration is that
the $L^1$ norm of the pseudo-time derivative decreases by three orders of magnitude, unless otherwise specified.}

\subsection{Accuracy test}
\label{ssec:accuracy-test}

The classical two-dimensional isentropic vortex problem \cite{hu1999weighted_WENO}
is modified to have low pressure and density \cite{zhang2012positivity}, serving as a test case to evaluate the accuracy of the positivity-preserving implicit finite volume method.
An isentropic vortex centered at $(x_0,y_0)$ is added to the mean flow $(\rho, u, v, p)=(1,1,1,1)$ with the following perturbations
\begin{equation}
    (\delta u, \delta v) = \frac{\epsilon}{2\pi} \exp(\frac{1-r^2}{2}) (-y+y_0, x-x_0),\ \
    \delta T = \frac{(\gamma-1)\epsilon^2}{8\gamma \pi^2}\exp(1-r^2), \ \ \delta\left(\frac{p}{\rho^\gamma}\right)=0, 
\end{equation}
where $r^2=(x-x_0)^2+(y-y_0)^2$ and $T= p/\rho$. 
The ratio of specific heat is $\gamma=1.4$. 
The exact solution is the passive convection of the vortex with the mean velocity. 
Following \cite{zhang2012positivity}, we set the vortex strength as $\epsilon = 10.0828$ 
such that the lowest density and pressure of the exact solution are 
$7.8 \times 10^{-15}$ and $1.7 \times 10^{-20}$, respectively.
The reference density and pressure used in reconstruction polynomial scaling are 
set as $\rho_0 = 10^{-10}$ and $p_0=10^{-10}$, respectively. 
%The very low reference value additionally assure that an exact initial 
% solution would not be limited for positivity preserving. 

The computational domain is defined as $[0,10]\times[0,10]$, with the vortex initially centered at $(x_0,y_0)=(5,5)$. Periodic boundary conditions are applied at the domain boundaries.  A set of successively refined rectangular meshes, with grid sizes $h=1/2$ to $h=1/32$, are used in the simulations. Following \cite{zhang2012positivity}, the final time is set as $t=0.01$. 
\replaced[id=r2]
{The maximum physical time step size is set to $\inc t_{max} = h/50$.
However, numerical experiments show that the physical time step limiting is not activated, indicating that the physical time step remains at $\inc t= \inc t_{max} = h/50$.
}{Pseudo-time step limiting, increment correction, and reconstruction polynomial 
scaling are applied during the simulations. 
Physical-time step limiting is disabled to maintain constant 
time step sizes of $\Delta t = h/50$.
}
% The time step size is proportional to the grid size, because both the
% spatial and temporal schemes have fourth order accuracy.

The CFL number for the pseudo-time step is $\CFLtau = 100$. 
The convergence criterion for inner iteration is that the $L^1$ norm of the pseudo-time derivative 
decreases by nine orders of magnitude.
% \added[id=harry]
% {
% The accuracy test is conducted using 4th order VFV and 4th order ESDIRK4 time marching. 
% }
The $L^1$ and $L^\infty$ errors in density, 
along with the corresponding convergence rates, 
are listed in Table \ref{tab:ivResults}. 
The accuracy results in Table \ref{tab:ivResults} demonstrate that the numerical scheme achieves the theoretical fourth-order accuracy.


\begin{table}[htbp!]
    \centering
    \caption{Accuracy test results for the isentropic vortex problem.} %, \added[id=harry]{4th order VFV and 4th order ESDIRK4}}
    \label{tab:ivResults}
    % \footnotesize
    % \begin{tabular}{|c|c|c|c|c|}
    \setlength{\tabcolsep}{12.5pt} % Increase column spacing
    \renewcommand{\arraystretch}{1.2}
    \begin{tabular}{ c c c c c}
        \toprule
        Grid size & $L^1$ error & Order & $L^\infty$ error & Order \\
        \midrule
        1/2 &1.85E-3 & - & 6.07E-2 & -\\
        % \hline
        1/4 &7.86E-5 & 4.55 & 4.85E-3 & 3.65\\
        % \hline
        1/8 &1.22E-6 & 6.02 & 2.12E-4 & 4.52\\
        % \hline
        1/16 &2.88E-8 & 5.40 & 1.34E-5 & 3.99\\
        % \hline
        1/32 &1.85E-9 & 3.96 & 7.14E-7 & 4.23 \\
    \bottomrule
    \end{tabular}
\end{table}

\added[id=r1]
{The convergence histories of the inner iterations are shown in Figure \ref{fig:IVres}.
As observed in the figure, the density residual decreases by nine orders of magnitude in each stage, indicating that the inner iterations reach convergence. Additionally, convergence is faster on finer meshes.
}

\replaced[id=r1]{A second-order implicit finite volume method is also tested to compare against the fourth-order method and highlight the advantages of high-order schemes. This second-order approach employs Green-Gauss reconstruction and trapezoidal-rule time integration, and the proposed positivity-preserving algorithm is also applied to ensure solution positivity.
The accuracy comparison between the second- and fourth-order methods is shown in Figure \ref{sfig:IVeff_err}, which demonstrates that both methods achieve their theoretical orders of accuracy. The efficiency comparison is shown in Figure \ref{sfig:IVeff_eff}. A numerical scheme is regarded more efficient if it achieves the same accuracy at a lower cost, or delivers higher accuracy at the same cost.  It can be observed from the error-cost plots in Figure \ref{sfig:IVeff_eff} that, for high-accuracy requirements, the fourth-order method achieves a given accuracy at significantly lower computational expense or provides much higher accuracy at equivalent computational cost. For example, at a wall-clock time of $70$ seconds, the errors of the second- and fourth-order methods are approximately $1.5 \times 10^{-6}$ and $3.0 \times 10^{-9}$, respectively. This indicates that the fourth-order method is approximately $500$ times more efficient in this scenario. Thus, the fourth-order finite volume method demonstrates substantially greater efficiency compared to its second-order counterpart for applications demanding high accuracy.
}
{In addition to accuracy test, efficiency of the current method is compared with a 2nd order counterpart.
The 2nd order test uses 2nd order Green-Gauss reconstruction and 2nd order trapezoid rule time marching. 
In the 2nd order scheme, the face flux integration uses midpoint rule. 
All positivity-preserving algorithms are in the same form.
All the tests are conducted on the same machine with the same parameters, and wall time is recorded 
for each test run. 
Figure \ref{fig:IVeff} displays convergence and efficiency of the 2nd and 4th order methods. 
$N_x$ in Figure \ref{fig:IVeff} means number of grids in one direction, or $10/h$ in other words.
In Figure \ref{sfig:IVeff_err}, the 4th order method displays 4th or higher order of accuracy, while the 2nd order 
method displays 2nd order accuracy. 
On the same $160\times160$ grid, 4th order method produces only 
less than $1/100$ of the error 2nd order method produces.
Figure \ref{sfig:IVeff_eff} indicates the efficiency of 4th order methods overtakes that of 2nd when 
the error demand is strict. 
When $10^{-4}$ density error is needed, 2nd order method uses less time,
but when $10^{-6}$ density error is needed, 4th order method uses significantly less time.
Therefore, it is demonstrated in the isentropic vortex tests that
high order positivity-preserving methods have two advantages: 
(1) high order methods can resolve the same structure better on the same grid; 
(2) high order methods consume less time and have higher efficiency 
when small error is demanded. 
}

\begin{figure}[htbp!]
    \centering
    \includegraphics[width=0.6\textwidth]{pics/PPRobust_IV_res.pdf}
    \caption{\added[id=r1]{Partial convergence history of inner iterations for the isentropic vortex problem. $N_{it}$ denotes the total number of inner iteration steps across the ESDIRK4 stages.}}
    \label{fig:IVres}
\end{figure}

\begin{figure}[htbp]
   \centering
   \begin{subfigure}{0.49\textwidth}
       \includegraphics[width=\textwidth]{pics/PPRobust_IV_err1.pdf}
       \caption[]{Accuracy}
       \label{sfig:IVeff_err}
   \end{subfigure}
   \hfill
   \begin{subfigure}{0.49\textwidth}
       \includegraphics[width=\textwidth]{pics/PPRobust_IV_eff1.pdf}
       \caption[]{Efficiency}
       \label{sfig:IVeff_eff}
   \end{subfigure}
   \caption{\added[id=r1]{Accuracy and efficiency comparisons between the fourth-order (O4) and second-order (O2) methods for the isentropic vortex problem. $N_x=10/h$ denotes the number of cells in the $x$-direction.}}
   \label{fig:IVeff}
\end{figure}

%% the following content will be presented in the response to reviewer
% \added[id=r2]
% {
% \color{violet}
% \textbf{
% We also tested on large $\inc t_{max}$ to see the impact of 
% physical time step limiting. 
% The termination time is changed to $t = 0.32$, and $\inc t_{max} =   0.64 h$ is used. 
% The minimum number of time steps is then $N_{t,min} =1/(2h)$. 
% This calculation uses 5 reconstruction iterations per evaluation of RHS, 
% and a {\it p}-multigrid method to enhance convergence. 
% Relative residual convergence threshold is set to $10^-6$.
% Results are shown in Table \ref{tab:ivResultsLargeDt}. 
% It can be seen in Table \ref{tab:ivResultsLargeDt} that the 
% physical time step limiting is activated on fine girds, and on 
% coarse grids the time step size is not limited. 
% As the average time step size still refines with grid nearly proportionally,
% the order of accuracy in the results also satisfies design order approximately.
% } 
% }


% \begin{table}[htbp!]
%     \centering
%     \caption{Accuracy test results for the isentropic vortex problem,  large time step size}
%     \label{tab:ivResultsLargeDt}
%     % \footnotesize
%     % \begin{tabular}{|c|c|c|c|c|}
%     \setlength{\tabcolsep}{12.5pt} % Increase column spacing
%     \renewcommand{\arraystretch}{1.2}
%     \begin{tabular}{ c llc c c c}
%         \toprule
%         Grid size  & $N_{t,min}$&$N_t$& $L^1$ error & Order & $L^\infty$ error & Order \\
%         \midrule
%  1/2 & 1 & 1 & 4.538e-03 & - & 9.729e-02 & -\\
%  1/4 & 2 & 2 & 2.282e-04 & 4.31 & 8.120e-03 & 3.58\\
%  1/8 & 4 & 7 & 7.204e-06 & 4.99 & 4.381e-04 & 4.21\\
%  1/16 & 8 & 9 & 1.320e-07 & 5.77 & 4.222e-05 & 3.38\\
%  1/32 & 16 & 18 & 4.956e-09 & 4.74 & 3.955e-06 & 3.42\\
%     \bottomrule
%     \end{tabular}
% \end{table}
%%%%%%%%%%%%%%%%%%%%%%%%%%%%%%

\subsection{\added[id=r1]{Le Blanc shock tube}}

\added[id=r1]{
The Le Blanc shock tube problem is a one-dimensional Riemann problem described by the Euler equations and 
generates an extremely strong shockwave \cite{toro2013riemann}. 
As a result, the Le Blanc problem is often used as 
a test case for positivity-preserving schemes \cite{hu2013positivity, chan2021positivity, huang2024general}. 
The initial conditions are
}
\begin{equation}
   (\rho,u,p) = \left\{
       \begin{array}{ll}
           (1,0,2/3\times10^{-1}),\ \ & x < 3\\
           (10^{-3},0,2/3\times10^{-10}),\ \ & x \geq 3\\
       \end{array}
   \right.
   \label{eq:leBlancCond}
\end{equation}
\added[id=r1]{The ratio of specific heat is $\gamma = 5/3$.
The computational domain is in $[0,9]$. 
Two uniform meshes are employed, consisting of $N=800$ and $N=3200$ cells, respectively. Simulations are conducted up to a final physical time of $t=6$, using two maximum physical time step sizes, $\Delta t_{max} = 0.01$ and $0.1$.
In each inner iteration, the CFL number for local pseudo time step, $\CFLtau$, is initiated as $0.5$ and increased gradually up to its maximum $10$ at the tenth iteration. 
The reference density and pressure are set as $\rho_0=1$ and $p_0=2/3\times10^{-1}$, respectively.
}

\begin{figure}[htbp]
   \centering
   \begin{subfigure}{0.33\textwidth}
       \includegraphics[width=\textwidth]{pics/PPRobust_LB_R.pdf}
       \caption[]{Density}
   \end{subfigure}\hfill
   \begin{subfigure}{0.33\textwidth}
       \includegraphics[width=\textwidth]{pics/PPRobust_LB_U.pdf}
       \caption[]{Velocity}
   \end{subfigure}\hfill
   \begin{subfigure}{0.33\textwidth}
       \includegraphics[width=\textwidth]{pics/PPRobust_LB_P.pdf}
       \caption[]{Pressure}
   \end{subfigure}
   \caption{\added[id=r1]{Numerical solutions of the Le Blanc shock tube problem at $t=6$.}}
   \label{fig:leBlanc}
\end{figure}

\added[id=r1]{
The numerical results from the four simulations, corresponding to two meshes and two maximum time step sizes, are presented in Figure \ref{fig:leBlanc}. The numerical solutions closely match the exact solution, and no negative values of pressure or density are observed. The positions of the shockwave and contact discontinuity computed on the finer mesh ($N=3200$) are significantly closer to the exact solution compared to those computed on the coarser mesh ($N=800$). Additionally, the larger $\Delta t_{max}$ produces more pronounced oscillations  oscillations near the shockwave on both meshes.
}


\begin{figure}[htbp!]
   \centering
   \begin{subfigure}{0.49\textwidth}
       \includegraphics[width=\textwidth]{pics/PPRobust_LB_dt.pdf}
       \caption[]{Time step}
       \label{sfig:leBlanc1_t}
   \end{subfigure}
   \hfill
   \begin{subfigure}{0.49\textwidth}
       \includegraphics[width=\textwidth]{pics/PPRobust_LB_res.pdf}
       \caption[]{Partial convergence history}
       \label{sfig:leBlanc1_res}
   \end{subfigure}
   \caption{\added[id=r1]{Time step evolution and inner iteration convergence history for the Le Blanc shock tube problem. $N_{it}$ denotes the total number of inner iteration steps across the ESDIRK4 stages.}}
   \label{fig:leBlanc1}
\end{figure}

\added[id=r1]{
% Figure \ref{fig:leBlanc1} illustrates history of time step size and density residual.
% Using smaller $\inc t_{max}$, the actual $\inc t$ often reaches $\inc t_{max}=0.01$ on both grids. 
% With larger $\inc t_{max}=0.1$, the actual time step $\inc t$ is constrained by the grid size near strong discontinuities.
% Figure \ref{sfig:leBlanc1_t} shows that smaller grid size imposes smaller time step size upper bound 
% in physical time step size limiting.
% The convergence history shown in Figure \ref{sfig:leBlanc1_res} indicates that the residuals
% reduce to three orders of magnitude within certain number of iteration steps.
Figure \ref{fig:leBlanc1} illustrates the history of the time step size and density residual. When using the smaller maximum time step size ($\Delta t_{max}=0.01$), the actual $\Delta t$ frequently reaches the imposed upper limit on both meshes. In contrast, with the larger $\Delta t_{max}=0.1$, the actual time step size is constrained by the grid spacing near regions with strong discontinuities. Figure \ref{sfig:leBlanc1_t} demonstrates that a smaller grid spacing imposes a stricter upper bound on the allowable physical time step size. Additionally, the convergence history presented in Figure \ref{sfig:leBlanc1_res} indicates that the residuals decrease by three orders of magnitude within a certain number of iteration steps.
}

\added[id=r2]{
% In order to investigate the effect of changing 
% relaxation parameters $\eta_t$, $\eta_\tau$ and $\eta_\Delta$,
% which are used in physical time step limiting, pseudo time step limiting and 
% increment correction respectively,
% a series of experiments are conducted on the Le Blanc problem.
% Based on the case $N=800$ and $\Delta t_{max}=0.1$, with
% the default relaxation parameters $\eta_t=0.8$, $\eta_\tau=0.5$ and $\eta_\Delta=0.9$, a series of additional test runs are performed with their results shown in Table \ref{tab:leBlancParamTest} and Figure \ref{fig:leBlancParamTest}.
To investigate the sensitivity of the positivity-preserving finite volume scheme to variations in the relaxation parameters $\eta_t$, $\eta_\tau$, and $\eta_\Delta$, which control physical time step limiting, pseudo time step limiting, and increment correction respectively, a series of numerical experiments are conducted using the Le Blanc shock tube problem. The baseline simulation is performed on the $N=800$ mesh, with default parameters $\eta_t=0.8$, $\eta_\tau=0.5$, and $\eta_\Delta=0.9$, using a maximum time step size of $\Delta t_{max}=0.1$. Additional simulations are performed by altering one parameter at a time from this baseline, enabling the assessment of sensitivity to each parameter. The statistical outcomes of these experiments are summarized in Table \ref{tab:leBlancParamTest}, while the corresponding numerical solutions are presented in Figure \ref{fig:leBlancParamTest}.
}

\added[id=r2]{
% In Table \ref{tab:leBlancParamTest}, statistics including number of total time steps $N_{step}$, average number of iterations per stage $\overline{N_{iter}}$ , 
% average time step size $\overline{\Delta t^n}$ , minimum and maximum step size $\min{\Delta t^n}$ and $\max{\Delta t^n}$  are listed. 
% It can be observed that changing $\eta_\Delta$ only slightly alters the number of time steps and iterations.
% A smaller $\eta_\tau$  increases $\overline{N_{iter}}$ while a larger $\eta_\tau$ decreases $\overline{N_{iter}}$ according to Table \ref{tab:leBlancParamTest}. 
% Using $\eta_\tau=0.5$ significantly decreases $\overline{N_{iter}}$ and improves efficiency compared to $\eta_\tau=0.2$, 
% and using $\eta_\tau=0.8$ does not improve efficiency much. 
% According to Table \ref{tab:leBlancParamTest}, the relaxation parameter $\eta_t$ primarily influences physical time step size. 
% A larger $\eta_t$ allows for larger $\Delta t^i$ and therefore fewer time steps.
% When $\eta_{t}=0.9$, $N_{step}$ decreases from $120$ to $114$, and with $\eta_t = 0.4$, $N_{step}$  is increased to $141$. 
% The benefit of further increasing $\eta_t$ from baseline setup is marginal.
Table \ref{tab:leBlancParamTest} provides key statistics including the total number of time steps ($N_{step}$), average iterations per stage ($\overline{N_{iter}}$), average time-step size ($\overline{\Delta t^n}$), minimum and maximum time-step sizes ($\min{\Delta t^n}$, $\max{\Delta t^n}$). It is observed that altering $\eta_\Delta$ only slightly affects $N_{step}$ and $\overline{N_{iter}}$. The parameter $\eta_\tau$ significantly influences the average number of iterations per stage: decreasing $\eta_\tau$ increases $\overline{N_{iter}}$, whereas increasing $\eta_\tau$ decreases it. Specifically, setting $\eta_\tau=0.5$ notably improves efficiency compared to $\eta_\tau=0.2$, while further increasing $\eta_\tau$ to 0.8 yields limited additional benefit.
According to Table \ref{tab:leBlancParamTest}, the relaxation parameter $\eta_t$ primarily affects the physical time-step size. A larger $\eta_t$ permits a larger $\Delta t^n$, resulting in fewer total steps. For instance, increasing $\eta_t$ from $0.8$ to $0.9$ reduces $N_{step}$ from $120$ to $114$, whereas decreasing $\eta_t$ to $0.4$ increases $N_{step}$ to $141$. Further increases in $\eta_t$ beyond the baseline setting offer marginal improvements.
}

\added[id=r2]{
% As illustrated in Figure \ref{fig:leBlancParamTest}, all relaxation parameters produce nearly identical solutions, with correct prediction of the 
% discontinuities as well as positivity of density and pressure.
% In summary, changing relaxation parameters  $\eta_t$, $\eta_\tau$ and $\eta_\Delta$ in a reasonable 
% range doe not affect convergence and accuracy of the method, 
% and the current choice of parameters is efficient enough while 
% being relatively conservative. 
Figure \ref{fig:leBlancParamTest} demonstrates that varying the relaxation parameters within reasonable ranges produces nearly identical numerical solutions. The discontinuities are accurately captured, and positivity of density and pressure is maintained. In summary, adjusting the parameters $\eta_t$, $\eta_\tau$, and $\eta_\Delta$ within practical limits does not significantly affect convergence or accuracy. Thus, the current parameter selection represents an efficient yet conservative choice.
}



\begin{table}[htbp!]
    \centering
    \caption{\added[id=r2]{Statistical results from the parameter sensitivity analysis with the Le Blanc shock tube problem.}}
    \label{tab:leBlancParamTest}
    % \footnotesize
    % \begin{tabular}{|c|c|c|c|c|}
    % \setlength{\tabcolsep}{12.5pt} % Increase column spacing
    \renewcommand{\arraystretch}{1.2}
    \begin{tabular}{ c c c c c c c c c}
        \toprule
        Case name& $\eta_t$ & $\eta_\tau$ &$\eta_\Delta$ & $N_{step}$ & $\overline{N_{iter}}$ & $\overline{\Delta t^n}$& $\min{\Delta t^n}$& $\max{\Delta t^n}$\\
        \midrule
Base  & 0.8 & 0.5 & 0.9 & 120 & 27.2 & 5.00E-02 & 2.58E-02 & 1.00E-01 \\
$\eta_{\Delta}=0.5$ & 0.8 & 0.5 & 0.5 & 118 & 27.5 & 5.08E-02 & 2.50E-02 & 1.00E-01 \\
$\eta_{\Delta}=0.95$ & 0.8 & 0.5 & 0.95 & 119 & 27.3 & 5.04E-02 & 2.42E-02 & 1.00E-01 \\
$\eta_{\tau}=0.2$ & 0.8 & 0.2 & 0.9 & 116 & 33.7 & 5.17E-02 & 2.74E-02 & 1.00E-01 \\
$\eta_{\tau}=0.8$ & 0.8 & 0.8 & 0.9 & 120 & 26.6 & 5.00E-02 & 2.61E-02 & 1.00E-01 \\
$\eta_{t}=0.4$ & 0.4 & 0.5 & 0.9 & 141 & 24.4 & 4.26E-02 & 2.20E-02 & 1.00E-01 \\
$\eta_{t}=0.9$ & 0.9 & 0.5 & 0.9 & 114 & 30.8 & 5.26E-02 & 2.81E-02 & 1.00E-01 \\
    \bottomrule
    \end{tabular}
\end{table}

\begin{figure}[htbp]
   \centering
   \begin{subfigure}{0.33\textwidth}
       \includegraphics[width=\textwidth]{pics/PPRobust_LBPT_R.pdf}
       \caption[]{Density}
   \end{subfigure}\hfill
   \begin{subfigure}{0.33\textwidth}
       \includegraphics[width=\textwidth]{pics/PPRobust_LBPT_U.pdf}
       \caption[]{Velocity}
   \end{subfigure}\hfill
   \begin{subfigure}{0.33\textwidth}
       \includegraphics[width=\textwidth]{pics/PPRobust_LBPT_P.pdf}
       \caption[]{Pressure}
   \end{subfigure}
   \caption{\added[id=r2]{Numerical solutions of the Le Blanc shock tube problem at $t=6$ on a $N=800$ mesh, using a maximum time step size of $\Delta t_{max}=0.1$, with different relaxation parameters.}}
   \label{fig:leBlancParamTest}
\end{figure}


\subsection{Double rarefaction}

The one-dimensional double rarefaction problem \cite{hu2004kineticDoubleRare}
is a Riemann problem with the following initial conditions
\begin{equation}
    (\rho,u,p) = \begin{dcases}
        (1,-2,0.1),\ \  & \mathrm{if} \ x < 0.5,       \\
        (1,2,0.1),\ \   & \mathrm{else}.
    \end{dcases}
    %    \left\{
    %        \begin{array}{ll}
    %            (1,-2,0.1),\ \ & x < 0.5,\\
    %            (1,2,0.1),\ \ &  \mathrm{else}.\\
    %        \end{array}
\end{equation}
The flow is inviscid and the ratio of specific heat is $\gamma=1.4$.
In this problem, a vacuum lies in in the middle, adjacent to the ends of two rarefaction waves.
The computational domain is $[0,1]$.
We perform a numerical simulation on a uniform mesh with a grid size of $\inc x = 1/400$, up to $t=0.1$
 using a maximum time step size of $\inc t_{max}= 5 \times 10 ^{-3}$.
\added[id=r1]
{The numerical results show that the physical time step limiting procedure does not reduce the time step in this case, resulting in a uniform $\Delta t$ throughout the simulation.}
In each inner iteration, the CFL number for local pseudo time step $\CFLtau$ is initiated as $0.1$ and
increased gradually up to its maximum $10$ at the tenth iteration.
The reference pressure and density values are set as $\rho_0=1$ and $p_0=0.1$, respectively.
%The convergence criterion for inner iteration is that the norm of the pseudo time derivative decreases by three orders of magnitude. 
The numerical results are presented in Figure \ref{fig:doubleRare}, which shows that the numerical solution is essentially oscillation-free and agrees well with the exact solution.
\added[id=r1]
{Figure \ref{fig:DRres} shows the convergence history of the inner iterations, indicating that convergence is achieved in just a few steps.
% Figure \ref{fig:DRres} illustrates the convergence history of inner iterations. It is observed from the figure that, the inner iterations reach convergence in only a few steps.
% In each stage of each ESDIRK step, it is shown 
% that the solution converges in rather few iterations. 
}

\begin{figure}[htbp]
    \centering
    \begin{subfigure}{0.33\textwidth}
        \includegraphics[width=\textwidth]{pics/PPRobust_DR_R.pdf}
        \caption[]{Density}
    \end{subfigure}\hfill
    \begin{subfigure}{0.33\textwidth}
        \includegraphics[width=\textwidth]{pics/PPRobust_DR_U.pdf}
        \caption[]{Velocity}
    \end{subfigure}\hfill
    \begin{subfigure}{0.33\textwidth}
        \includegraphics[width=\textwidth]{pics/PPRobust_DR_P.pdf}
        \caption[]{Pressure}
    \end{subfigure}
    \caption{Results of the double rarefaction problem at $t=0.1$.}
    \label{fig:doubleRare}
\end{figure}

\begin{figure}[htbp!]
    \centering
    \includegraphics[width=0.5\textwidth]{pics/PPRobust_DR_res.pdf}
    \caption{\added[id=r1]{Partial convergence history of inner iterations for the double rarefaction problem. $N_{it}$ denotes the total number of inner iteration steps across the ESDIRK4 stages.}}
    \label{fig:DRres}
\end{figure}




\subsection{Sedov blast wave}
\label{ssec:sedov}

The Sedov blast wave problem \cite{zhang2012positivity,vilar2016positivity} is a popular case to test positivity-preserving properties of
numerical schemes.
The governing equations are the Euler equations and the ratio of specific heat is $\gamma = 1.4$. The computational domain is $[0,1.1]\times[0,1.1]$, partitioned into rectangular cells with a grid size of $\inc x = \inc y = 1.1/160$. The initial conditions are
\begin{equation}
    (\rho,u,v,p) = \begin{dcases}
        (1,0,0,4\times 10 ^{-9}),\ \                                          & \text{if } x > \inc x \ \mathrm{or} \ y > \inc y, \\
        (1,0,0,\dfrac{\left(\gamma-1\right)\varepsilon^0}{\inc x\inc y}),\ \  & \text{else},                                      \\
    \end{dcases}
\end{equation}
where $\varepsilon^0$ is the total amount of release energy.
By choosing $\varepsilon^0= 2.44816\times 10^5$,
the solution consists of a diverging infinite strength shock wave
whose front is located at radius $r=1$ at $t=10^{-3}$, with a peak density reaching $6$.
%The final computational time is $t = 1\times10^{-3}$,
%when the exact solution has the shock wave at radius of $1$. 
All boundaries are slip walls.
The reference density and pressure are set as $\rho_0=1$ and $p_0=4\times10^{-9}$, respectively.
%Fourth order variational reconstruction finite volume with ESDIRK4
% discretization is used. 
%The relaxation parameters and internal CFL number are the same as 1-D problems.
%Convergence threshold for pseudo time iteration is $10^{-3}$. 
%The convergence criterion for inner iteration is that the norm of the pseudo time derivative decreases by three orders of magnitude. 
%Maximum time step sizes of $\inc t_{max}=10^{-6}$ and $\inc t_{max}=10^{-5}$
%are tested. 
%The computation takes 1026 steps with $\inc t_{max}=10^{-6}$, 
%and 297 steps with $\inc t_{max}=10^{-5}$.

Two simulations are performed up to $t=10^{-3}$,
using $1014$ time steps with $\inc t_{max} = 10^{-6}$
and  $161$ time steps with $\inc t_{max} = 10^{-5}$, respectively.
\added[id=r1]
{The evolution of the time step and the convergence history of inner iterations are presented in Figure \ref{fig:sedov1}. As shown in Figure \ref{sfig:sedov1_dt}, the initial Sedov blast imposes strong restrictions on the time step size. As the blast evolves, these restrictions are gradually relaxed until the predefined upper bound $\Delta t_{max}$ is reached. Figure \ref{sfig:sedov1_res} shows that the inner iterations converge rapidly within each ESDIRK4 stage.
% The evolution of time step and the convergence history of inner iterations are shown in 
% Figure \ref{fig:sedov1}. 
% Figure \ref{sfig:sedov1_dt} indicates that the initial Sedov blast
% leads to strong restrictions on time step sizes, and with 
% the evolution of the blast, the time step limitation is relaxed 
% until the predefined upper bound $\inc t_{max}$ is reached. In each ESDIRK4 stage, the inner iteration reaches convergence rapidly, as shown in Figure \ref{sfig:sedov1_res}.
}

The computed density contours at $t=10^{-3}$ are shown in Figure \ref{fig:sedov}. It is observed that the numerical solutions are essentially oscillation-free. To check the solutions in a more intuitive way, we plot both the analytical and numerical density distributions along the $y=x$ line in Figure \ref{fig:sedovLine}.
It is shown in Figure \ref{fig:sedovLine} that both numerical solutions agree well with the analytic solution.
% and the one with the smaller time step size has higher resolution.
%With larger time step size limit $\inc t_{max}=10^{-5}$, the 
%solution appears slightly distorted, but the overall 
%position and strength of the shock front is correct.

A computational cost analysis is performed to validate the computational efficiency of the proposed positivity-preserving algorithm. Table \ref{tab:cpu-time-cost} reports the CPU time for the final physical time step in the case with $\Delta t_{\text{max}} = 10^{-5}$, where the reconstruction polynomial scaling is performed on $22\%$ of the cells. As listed in Table \ref{tab:cpu-time-cost} that, the positivity-preserving algorithm accounts for only $10\%$ of the total CPU time.
This finding indicates that the algorithm incurs relatively small additional computational cost, thereby confirming its efficiency.

\begin{figure}[htbp]
    \centering
    \begin{subfigure}{0.49\textwidth}
        \includegraphics[width=\textwidth]{pics/PPRobust_Sedov_dt.pdf}
        \caption[]{Time step}
        \label{sfig:sedov1_dt}
    \end{subfigure}
    \hfill
    \begin{subfigure}{0.49\textwidth}
        \includegraphics[width=\textwidth]{pics/PPRobust_Sedov_res.pdf}
        \caption[]{Partial convergence history}
        \label{sfig:sedov1_res}
    \end{subfigure}
    \caption{\added[id=r1]{Time step evolution and inner iteration convergence history for the Sedov blast wave problem. $N_{it}$ denotes the total number of inner iteration steps across the ESDIRK4 stages.}}
    \label{fig:sedov1}
\end{figure}

\begin{figure}[htbp]
    \centering
    \begin{subfigure}{0.5\textwidth}
        \includegraphics[width=\textwidth]{pics/PPRobust_SedovDT1.png}
        \caption[]{$\inc t_{max}=10^{-6}$}
    \end{subfigure}\hfill
    \begin{subfigure}{0.5\textwidth}
        \includegraphics[width=\textwidth]{pics/PPRobust_SedovDT10.png}
        \caption[]{$\inc t_{max}=10^{-5}$}
    \end{subfigure}
    \caption{Density contours of the Sedov blast wave problem.}
    \label{fig:sedov}
\end{figure}

\begin{figure}[htbp]
    \centering
    \includegraphics[width=0.6\textwidth]{pics/PPRobust_Sedov.pdf}
    \caption{Density distributions of the Sedov blast wave problem along the diagonal line.}
    \label{fig:sedovLine}
\end{figure}

\begin{table}[htbp!]
    \centering
    \caption{Computational cost for the final physical time step in the case with $\Delta t_{\text{max}} = 10^{-5}$.}
    \label{tab:cpu-time-cost}
    % \footnotesize
    % \begin{tabular}{|c|c|c|c|c|}
    \setlength{\tabcolsep}{12.5pt} % Increase column spacing
    \renewcommand{\arraystretch}{1.2}
    \begin{tabular}{l c c}
        \toprule
        Procedure & CPU times (s) & Percentage (\%) \\
        \midrule
        Positivity preserving & 7.3566 & 10.35\\
        % \hline
        Variational reconstruction & 4.9303 & 6.94\\
        % \hline
        WBAP limiting & 41.7281 & 58.71\\
        % \hline
        Numerical flux & 13.0100 & 18.31\\
        % \hline
        Linear solving and others &  4.0440 & 5.69\\
        % \hline
        % Positivity preserving & 7.3566 & 10.35\\
        \midrule
        {Total} & 71.069 & 100 \\
    \bottomrule
    \end{tabular}
\end{table}

\subsection{Mach 2000 jet}

The Mach 2000 jet problem \cite{zhang2010positivity} is a challenging case as it has extremely strong discontinuities.
Following the practice of \cite{huang2024general},
the current paper studies two cases with $\Re=\infty$ and $\Re=100$ to demonstrate the capability of the proposed positivity-preserving algorithm to deal with inviscid and viscous compressible flows, respectively.
\replaced[id=r1]{The governing equations are the Euler equations for the inviscid case and the Navier–Stokes equations for the viscous case. The ratio of specific heat is $\gamma=5/3$. 
For the $\Re=100$ case, the dynamic viscosity is set as a constant $\mu=\rho_{\infty} u_{\infty} L/\Re$, where $\rho_{\infty}=1$, $u_{\infty}=1$ and $L=1$, following \cite{huang2024general}.
The Prandtl number for the viscous case is set to $Pr=0.7$.
% In the $Re=100$ case, the dynamic viscosity is set as a constant $\mu=\rho_{\infty} u_{\infty} L/Re$ because the test case is artificial.
}
{The governing equations are the Navier-Stokes equations with $\gamma=5/3$.}
The computational domain is $[0,1]\times[-0.25,0.25]$.
Far-field boundary conditions are imposed on the upper, lower and right boundaries. The following inflow conditions are imposed on the left boundary:
\begin{equation}
    (\rho,u,v,p) = \left\{
    \begin{array}{ll}
        (5,800,0,0.4127),\ \  & \text{if } -0.05 \leq y \leq 0.05, \\
        (0.5,0,0,0.4127),\ \  & \text{else}.                       \\
    \end{array}
    \right.
\end{equation}
The initial conditions are $(\rho,u,v,p)=(0.5,0,0,0.4127)$.
The numerical simulations are performed on a uniform rectangular mesh with $800 \times 800$ cells, up to $t=10^{-3}$ using a maximum time step size of $\inc t_{max} = 1\times10^{-6}$.
%$\CFLtau$ is initially $0.1$ at the beginning of each stage,
%and reaches $2.0$ at the 10th pseudo time iteration. 
In each inner iteration, the CFL number for local pseudo time step $\CFLtau$ is initiated as $0.1$
and increased gradually up to its maximum $2$ at the tenth iteration.
The reference density and pressure are
$\rho_0=0.5$ and $p_0=0.4127$, respectively.

The computed density and pressure contours at $t=10^{-3}$ in logarithmic scales are shown in Figures \ref{fig:M2000_ReInf_R} to \ref{fig:M2000_Re1E2_P}.
It is observed from these figures that the numerical results are essentially non-oscillatory, and small-scale flow structures are smeared in the $Re=100$ case due to viscous effects.

\begin{figure}[htbp]
    \centering
    \includegraphics[trim={5px 0 5px 0},clip,width=0.9\textwidth]{pics/PPRobust_M2000_ReInf_R.png}
    \caption{$\log_{10}(\rho)$ of $\Re=\infty$ Mach 2000 jet, 40 contour lines from -1.5 to 1.5.}
    \label{fig:M2000_ReInf_R}
\end{figure}

\begin{figure}[htbp]
    \centering
    \includegraphics[trim={5px 0 5px 0},clip,width=0.9\textwidth]{pics/PPRobust_M2000_Re1E2_R.png}
    \caption{$\log_{10}(\rho)$ of $\Re=100$ Mach 2000 jet, 40 contour lines from -1.5 to 1.5.}
    \label{fig:M2000_Re1E2_R}
\end{figure}

\begin{figure}[htbp]
    \centering
    \includegraphics[trim={5px 0 5px 0},clip,width=0.9\textwidth]{pics/PPRobust_M2000_ReInf_P.png}
    \caption{$\log_{10}(p)$ of $\Re=\infty$ Mach 2000 jet, 40 contour lines from -1.4 to 5.3.}
    \label{fig:M2000_ReInf_P}
\end{figure}

\begin{figure}[htbp]
    \centering
    \includegraphics[trim={5px 0 5px 0},clip,width=0.9\textwidth]{pics/PPRobust_M2000_Re1E2_P.png}
    \caption{$\log_{10}(p)$ of $\Re=100$ Mach 2000 jet, 40 contour lines from -1.4 to 5.3.}
    \label{fig:M2000_Re1E2_P}
\end{figure}

\subsection{Shock diffraction}

The shock diffraction problem \cite{zhang2010positivity}, in which a shock passes a backward facing corner,
is used to test the positivity-preserving capability of the proposed algorithm.
The governing equations are the Euler equations with $\gamma = 1.4$.
The computational domain is the union of $[0,1]\times[6,11]$ and $[1,13]\times[0,11]$.
The initial condition is a Mach $5.09$ shock located at $x=0.5$,
moving into undisturbed air ahead of the shock at a state $(\rho, u, v, p)=(1.4,0,0,1)$.
%{\color{r1color}The boundary conditions are inflow at $x = 0, \ 6\leq y \leq 11$, outflow at $x = 13, \ 0\leq y \leq 11$, $1 \leq x \leq 13, \ y = 0$ and $0 \leq x \leq 13, \ y = 11$, and non-slip at the walls $0\leq x \leq 1, \ y = 6$ and $x = 1, \ 0\leq y \leq 6$.}
Inflow and slip solid wall boundary conditions are imposed on
the left side $x = 0$, $6\leq y \leq 11$ and other boundaries, respectively.
%The boundary at $x=0$ is set to match the left state of the shock, and 
%all other boundaries are set as inviscid wall.
The simulation is performed on a rectangular mesh with grid size $\inc x = \inc y = 1/80$,
up to $t=2.3$ with $\inc t_{max}=1\times 10^{-3}$.
In each inner iteration, the CFL number for local pseudo time step $\CFLtau$ is initiated as $0.1$
and increased gradually up to its maximum $2$ at the tenth iteration.
The reference density and pressure are $\rho_0=1.4$ and $p_0=1$, respectively.

\begin{figure}[htbp]
    \centering
    \begin{subfigure}{0.5\textwidth}
        \includegraphics[width=\textwidth]{pics/PPRobust_Corner_R.png}
        \caption[]{$\log_{10}(\rho)$, 40 contour lines from -0.8 to 0.8}
    \end{subfigure}\hfill
    \begin{subfigure}{0.5\textwidth}
        \includegraphics[width=\textwidth]{pics/PPRobust_Corner_P.png}
        \caption[]{$\log_{10}(p)$, 40 contour lines from -0.6 to 1.6}
    \end{subfigure}
    \caption{Density and pressure contours of the shock diffraction problem.}
    \label{fig:Corner}
\end{figure}

The computed density and pressure contours in logarithmic scales at $t=2.3$ are shown in Figure \ref{fig:Corner}.
It is observed from Figure \ref{fig:Corner} that,
the numerical solutions are essentially oscillation-free and small-scale shock structures are well resolved.


\subsection{Shock reflection and diffraction around a wedge}

The shock reflection and diffraction problem \cite{zhang2017positivity} is similar to the shock diffraction problem,
with the corner replaced by a wedge. The computational domain is $[0,3]\times[0,2]$
with a $30^\circ$ wedge placed at $0.2 \leq x \leq 1.2$, while the
tip of the wedge is at $(x,y) = (1.2, 1/\sqrt{3})$.
Initially, there is a Mach $10$ shock at $x=0.2$ moving into undisturbed air in a state $(\rho, u, v, p) = (1.4,0,0,1)$. The $x=0$ boundary on
the left matches the downstream state of the shock, and all other boundaries
are inviscid walls.
Before the diffraction of the shock, the development
of the shock structure is identical with that in the Mach $10$ double Mach reflection problem \cite{woodward1984dmr}.
Similar to the Mach $2000$ jet example, $\Re=\infty$ and $\Re=100$ cases
are investigated.
\added[id=r1]{
For all the cases, the ratio of specific heat is $\gamma=1.4$.
% For the $\Re=100$ viscous case, 
% the dynamic viscosity $\mu$ is taken as a constant with the
% reference state of $\Re$ being $\rho=1,u=1,L=1$, following \cite{zhang2017positivity}. 
% The Prandtl number in the viscous case is set to $Pr=0.7$.
For the $\Re=100$ case, the dynamic viscosity is set as a constant $\mu=\rho_{\infty} u_{\infty} L/\Re$, where $\rho_{\infty}=1$, $u_{\infty}=1$ and $L=1$, following \cite{huang2024general}.
The Prandtl number for the $\Re=100$ case is set to $Pr=0.7$.
% The gas viscosity at $\Re=100$ is set as constant value $\mu=\rho_{\infty} u_{\infty} L/\Re$ considering the viscous test case is artificial.
% The Prandtl number is set to $Pr=0.7$.
}
A triangular mesh with grid size $h = 1/320$,
shown in Figure \ref{fig:wedgeMesh}, is used in the simulations.
The simulations are performed up to $t=0.245$ using
a maximum time step size of $\inc t_{max}= 2\times10^{-4}$.
In each inner iteration, the CFL number for local pseudo time step $\CFLtau$ is initiated as $0.1$
and increased gradually up to its maximum $2$ at the tenth iteration.
The reference density and pressure values are $\rho_0=1.4$ and $p_0=1$, respectively.

\begin{figure}[htbp]
    \centering
    \includegraphics[trim={5px 0 5px 0},clip,width=0.6\textwidth]{pics/PPRobust_WedgeMesh.png}
    \caption{Mesh used in the shock reflection and diffraction problem.}
    \label{fig:wedgeMesh}
\end{figure}

The computed density and pressure contours are shown in Figure \ref{fig:wedgeReInf} and \ref{fig:wedgeRe1E2}.
It is observed that the numerical solutions are essentially non-oscillatory.
The Kelvin-Helmholtz instability induced in the shear layer is well resolved in Figure \ref{fig:wedgeReInf} for the $\Re=\infty$ case.
While in Figure  \ref{fig:wedgeRe1E2} for the $\Re=100$ case,
the small-scale flow structures around the shear layer and the wedge are smeared  due to the existence of physical viscosity.

\begin{figure}[htbp!]
    \centering
    \begin{subfigure}{0.5\textwidth}
        \includegraphics[width=\textwidth]{pics/PPRobust_WedgeReInf_R_Bi.png}
        \caption[]{$\rho$, 60 contour lines from 0 to 22}
    \end{subfigure}\hfill
    \begin{subfigure}{0.5\textwidth}
        \includegraphics[width=\textwidth]{pics/PPRobust_WedgeReInf_P_Bi.png}
        \caption[]{$\log_{10}(p)$, 60 contour lines from -1.7 to 2.7}
    \end{subfigure}
    \caption{Density and pressure contours for the shock reflection and diffraction problem, $\Re=\infty$.}
    \label{fig:wedgeReInf}
\end{figure}

\begin{figure}[htbp!]
    \centering
    \begin{subfigure}{0.5\textwidth}
        \includegraphics[width=\textwidth]{pics/PPRobust_WedgeRe1E2_R_Bi.png}
        \caption[]{$\rho$, 60 contour lines from 0 to 22}
    \end{subfigure}\hfill
    \begin{subfigure}{0.5\textwidth}
        \includegraphics[width=\textwidth]{pics/PPRobust_WedgeRe1E2_P_Bi.png}
        \caption[]{$\log_{10}(p)$, 60 contour lines from -1.7 to 2.7}
    \end{subfigure}
    \caption{Density and pressure contours for the shock reflection and diffraction problem, $\Re=100$.}
    \label{fig:wedgeRe1E2}
\end{figure}

\subsection{\added[id=r1]{Mach 20 shock over a half cylinder}}

\added[id=r1]
{
The Mach 20 shock over a half cylinder test case \cite{gallice2022entropy,cossart2025toward} is used to validate the proposed positivity-preserving algorithm in the context of steady-state hypersonic flows involving strong shock waves. The governing equations are the Euler equations. The ratio of specific heat is $\gamma=1.4$. 
% Three numerical schemes are tested, including the first-, second- and fourth-order 
% finite volume methods using reconstruction polynomials of degree zero, one, and three, respectively. 
% The degree one and three polynomials are obtained through Green-Gauss and variational reconstructions, respectively.
Three numerical schemes are tested: first-, second-, and fourth-order finite volume methods, 
using reconstruction polynomials of degree zero, one, and three, respectively. The degree-one and degree-three piecewise polynomials 
are respectively obtained using the Green–Gauss and variational reconstructions.
As described in Section \ref{ssec:application-to-steady-state}, for this steady-state problem, pseudo time step limiting, increment correction and reconstruction polynomial scaling are employed to preserve the positivity of density and pressure. 
Physical time step limiting is not required in this context. 
The inflow condition is $(\rho, u, v, p) = (1, 1, 0, 1.78571\times10^-3)$.
The computational mesh consists of $40 \times 100$ quadrilateral cells, 
as shown in Figure \ref{sfig:CylinderFront_mesh}.
The CFL number for local pseudo time step is $\CFLtau=10$. 
The reference density and pressure are set as $\rho_0=1$ and $p_0=1.78571\times10^{-3}$, respectively.
% A inviscid Ma 20 flow over a two-dimensional round head is tested in the 
% current subsection following literature that investigated steady state positivity-preserving
% methods . 
% Although sometimes time-accurate temporal integration is used to obtain a steady state solution, 
% the current test is performed using steady state iteration to illustrate the capability 
% of current algorithm to directly handle steady state computation. 
% In other words, the time-accurate time marching schemes like ESDIRK are replaced with 
% a single implicit Euler step with infinite time step size.
% All the iterations are in pseudo time and correspond to the inner iterations in the 
% unsteady problems. 
% Therefore, the physical time step limiting is not present in a steady state computation. 
% Pseudo time step limiting, increment correction and polynomial scaling are the same 
% as those in unsteady problems.
}

% \added[id=r1]
% {
% The mesh used in Ma 20 bow shock is a $40\times100$ structured grid as shown in 
% Figure \ref{sfig:CylinderFront_mesh}.
% The inflow condition is $(\rho, u, v, p) = (1, 1, 0, 1.78571\times10^{-3})$ and $\gamma=1.4$.
% The reference density and pressure are set as $\rho_0=1$ and $p_0=1.78571\times10^{-3}$, respectively.
% Steady calculations are performed using 1st order, 2nd order and 4th order spatial discretizations.
% The 4th order spatial discretization is 4th order VFV.
% The 2nd order spatial discretization is 2nd order FV using Green Gauss reconstruction.
% The 1st order spatial discretization is 1st order FV using piecewise constant reconstruction.
% The 1st and 2nd order methods use midpoint rule instead of high order quadrature rule.
% All spatial discretizations still use LLF flux as numerical inviscid flux.
% }

\added[id=r1]
{
The convergence history is shown in Figure \ref{fig:CylinderFront_res}. The plots indicate that the first-order method converges rapidly, with the residual dropping to $10^{-8}$ of its peak value within 400 iterations. In contrast, the residual for the second-order method decreases by only five orders of magnitude, and for the fourth-order method by only three. Such slow or limited convergence is commonly observed when simulating shock-dominated flows using second- or higher-order methods with limiters.
% The convergence history is displayed in Figure \ref{fig:CylinderFront_res}. It is observed from the convergence plots that, the first-order method converges rapidly and the residual decreases to $10^{-8}$ of the peak value within $400$ iterations.
% Residual of the second-order method can only decrease by five orders of magnitude,
% while the fourth-order one by only three.
% This troubled convergence is common in simulating flows with shocks using second or higher order methods equipped with limiters.
}

\added[id=r1]
{
The computed density contours are shown in Figure \ref{fig:CylinderFront}, demonstrating that the Mach 20 bow shock is accurately captured in all three simulations. The use of the local Lax–Friedrichs (LLF) flux effectively avoids the carbuncle phenomenon, a well-known shock instability. The contours also show that shock resolution improves with increasing order of the numerical method.
% Positivity-preserving algorithms seem not to have affected 
% the quality of converged results.  
% TODO: carbuncle, resolution
% By using the local Lax-Friedrichs (LLF) flux, the infamous carbuncle phenomenon, which is a kind of shock instabilities, is avoided. It is observed from the density contours that, the shock resolution is improved by using the higher-order methods.
% Width of the shock is rather large in the 1st order result, due to 
% the use of LLF flux.
% Such smearing is reduced in 2nd and 4th order results. 
% The shock 
% is the sharpest in the 4th order result, 
% and the density peak near stagnation point is also 
% highest in the 4th order result.
}

\begin{figure}[htbp]
    \centering
    \includegraphics[width=0.5\textwidth]{pics/PPRobust_CylinderFront_res.pdf}
    \caption{\added[id=r1]{Convergence history of the Mach $20$ shock over a half cylinder problem. In figure legends, \textquote{O1}, \textquote{O2} and \textquote{O4} denote the first-, second- and fourth-order finite volume schemes.}}
    \label{fig:CylinderFront_res}
\end{figure}


\begin{figure}[htbp!]
    \centering
    \begin{subfigure}{0.24\textwidth}
        \includegraphics[width=\textwidth]{pics/CylinderFront_O1.png}
        \caption[]{First-order}
        \label{sfig:CylinderFront_O1}
    \end{subfigure}
    \begin{subfigure}{0.24\textwidth}
        \includegraphics[width=\textwidth]{pics/CylinderFront_O2.png}
        \caption[]{Second-order}
        \label{sfig:CylinderFront_O2}
    \end{subfigure}
    \begin{subfigure}{0.24\textwidth}
        \includegraphics[width=\textwidth]{pics/CylinderFront_O4.png}
        \caption[]{Fourth-order}
        \label{sfig:CylinderFront_O4}
    \end{subfigure}
    \begin{subfigure}{0.24\textwidth}
        \includegraphics[width=\textwidth]{pics/CylinderFront_Mesh.png}
        \caption[]{Mesh}
        \label{sfig:CylinderFront_mesh}
    \end{subfigure}
    \caption{\added[id=r1]{Computed density contours and computational mesh for the Mach $20$ shock over a half cylinder problem.}}
    \label{fig:CylinderFront}
\end{figure}


\subsection{\added[id=r2]{Hypersonic flow past an open cavity}}

\added[id=r2]
{
The hypersonic flow past an open cavity problem \cite{morgenstern1994hypersonic} involves high-speed air flowing over a rectangular cavity embedded in a wall aligned with the flow direction. The flow is modeled using the compressible Navier–Stokes equations and features a hypersonic boundary layer. This test case is used to evaluate the capability of the proposed positivity-preserving implicit finite volume schemes in handling high-Reynolds-number viscous flows.
% Due to the significance of viscosity and hypersonic transient 
% procedures in this problem, the hypersonic cavity is used in the current 
% research to investigate the performance of implicit positivity preserving 
% algorithms.
}

\added[id=r2]{
Following earlier computational studies \cite{morgenstern1994hypersonic}, we adopt an inflow Mach number of \(6.3\) and a Reynolds number per unit length of \(\Re = 4.084 \times 10^6\). The ratio of specific heat is \(\gamma = 1.4\). The inflow stagnation temperature is \(1110\,\text{K}\), and the wall temperature is set to \(300\,\text{K}\). In this test, viscosity follows Sutherland's law for air, and the Prandtl number is taken as \(0.72\).
The cavity has a length-to-depth ratio of \(L/D = 10.67\), with a dimensional depth of \(D = 19.1\,\text{mm}\). For numerical convenience, the inflow density, velocity, and cavity depth are normalized to unity.
The maximum non-dimensional physical time-step size is set to \(\Delta t_{max} = 0.01\) to adequately resolve unsteady flow features, based on prior results \cite{morgenstern1994hypersonic}. In each inner iteration, the CFL number for the local pseudo-time step, \(\text{CFL}_\tau\), is initialized at 0.5 and gradually increased to a maximum of 10 by the tenth iteration.
Although the setup largely follows the reference study \cite{morgenstern1994hypersonic}, certain aspects remain ambiguous. These include the specification of the inlet velocity profile, the boundary layer mesh resolution near the wall, and the choice of time-step size. To the authors' knowledge, these details have not been explicitly discussed in the existing literature.
}

% \added[id=r2]
% {
% Following early computational research \cite{morgenstern1994hypersonic}, 
% we use an inflow condition of $Ma=6.3$, Reynolds number per unit length (meter) is  $\Re=4.084\times10^6$. 
% Air specific heat ratio and Prandtl number are taken as $\gamma = 1.4$ and $Pr=0.7$.
% Inflow has stagnation temperature $1110\unit{K}$ and wall temperature is $300\unit{K}$.
% In this test, viscosity obeys standard air's Sutherland's law and Prandtl number is set as $0.72$.
% The cavity has length-to-depth ratio $L/D=10.67$ and the dimensional depth is $D=19.1\unit{mm}$.
% In actual computation, 
% inflow density, velocity and cavity depth are normalized as $1$.
% The reference density and pressure are set as $\rho_0=1$ and $p_0=1.7997\times10^{-2}$, respectively.
% Maximum non-dimensional time step is set to $\inc t_{max}=0.01$ to capture most of the unsteady features 
% according to previous results \cite{morgenstern1994hypersonic}. 
% In each inner iteration, the CFL number for local pseudo time step $\CFLtau$ is initiated as $0.5$
% and increased gradually up to its maximum $10$ at the tenth iteration.
% Although the conditions above are all derived from the reference 
% computation \cite{morgenstern1994hypersonic}, there are still some ambiguities
% in the computational setup including 
% the inlet velocity profile, the boundary layer mesh height and 
% the choice of time step size, which has not been discussed in the literature
% to the author's knowledge.
% }

\begin{figure}[htbp]
    \centering
    \includegraphics[width=0.85\textwidth]{pics/Cavity_Mesh.png}
    \caption{\added[id=r2]{Computational mesh near the cavity for the hypersonic flow past an open cavity problem.}}
    \label{fig:Cavity_mesh}
\end{figure}

\added[id=r2]
{
A mixed two-dimensional unstructured mesh consisting of 121,747 quadrilateral and triangular cells is used for the simulation. 
The height of the first off-wall grid point is set to $10^{-4}D$, and the maximum element aspect ratio reaches $5400$.
% Due to the boundary layer mesh, the maximum aspect ratio in all cells is $5400$,
% which induces strong geometrical stiffness.
% maximum. 
A portion of the mesh is shown in Figure \ref{fig:Cavity_mesh}.
The simulation is initialized with freestream conditions in the region $y \geq 0$. In the region $y < 0$, freestream pressure, wall temperature, and zero velocity are prescribed as the initial condition. The simulation proceeds until the initial transients have dissipated.
The positivity-preserving fourth-order variational finite volume method, combined with ESDIRK4 time integration, is employed for this test.
% All positivity-preserving algorithms, including physical time step limiting,
% pseudo time step limiting, increment correction and reconstruction polynomial
% scaling are applied. 
The reference density and pressure are set to \(\rho_0 = 1\) and \(p_0 = 1.7997 \times 10^{-2}\), respectively.
}

\begin{figure}[htbp]
    \centering
    \includegraphics[width=0.85\textwidth]{pics/Cavity_T.png}
    \caption{\added[id=r2]{Time-averaged temperature contour of the hypersonic flow past an open cavity. The plot includes 24 contour lines ranging from 100 to 560.}}
    \label{fig:Cavity_T}
\end{figure}

\begin{figure}[htbp]
    \centering
    \includegraphics[width=0.85\textwidth]{pics/Cavity_P.png}
    \caption{\added[id=r2]{Contour of time-averaged normalized pressure for the hypersonic flow past an open cavity. The normalized pressure is computed as $p/(\rho_\infty U_\infty^2)$. The The plot includes 27 contour lines ranging from 0.01 to 0.07.}}
    \label{fig:Cavity_P}
\end{figure}

\begin{figure}[htbp!]
    \centering
    \begin{subfigure}{0.5\textwidth}
        \includegraphics[width=\textwidth]{pics/PPRobust_Cavity_q.pdf}
        \caption[]{Heat transfer rate}
        \label{sfig:Cavity1_q}
    \end{subfigure}\hfill
    \begin{subfigure}{0.5\textwidth}
        \includegraphics[width=\textwidth]{pics/PPRobust_Cavity_p.pdf}
        \caption[]{Pressure}
        \label{sfig:Cavity1_p}
    \end{subfigure}
    \caption{\added[id=r2]{Time-averaged wall heat transfer rate and pressure.}}
    % Compared with experiment \cite{hahn1969experimental} and computation \cite{morgenstern1994hypersonic}.}}
    \label{fig:Cavity1}
\end{figure}


\added[id=r2]{
After the initial transients subside, time-averaged flow properties are analyzed. Figures \ref{fig:Cavity_T} and \ref{fig:Cavity_P} show the time-averaged temperature and pressure distributions, respectively. The results generally agree with those reported in the reference computation \cite{morgenstern1994hypersonic}.
Figure \ref{fig:Cavity1} compares the computed wall heat flux and pressure with experimental data from \cite{hahn1969experimental}. For comparison, both quantities are normalized using values at the upstream station $x/D = -1.33$. The computed heat flux aligns well with both the experimental data and the reference computation. The variation near the trailing edge of the cavity is attributed to large vortex structures; the experimental setup may not have provided sufficient spatial resolution to capture this behavior \cite{morgenstern1994hypersonic}.
The pressure distribution along the cavity floor agrees closely with the reference simulation. However, the experimental pressure values are significantly lower near the cavity leading edge. 
This discrepancy may be due to differences in flow conditions or setup details between the current simulation and the experimental study.
% It is noted that, the discrepancy between the current and reference simulation results may be partially due to differences in flow conditions or setup details between the current simulation and the reference study.
}

% \added[id=r2]
% {
% After initial transients are eliminated, time-averaged properties 
% are investigated. Figure \ref{fig:Cavity_T} and Figure \ref{fig:Cavity_P}
% demonstrate time-averaged temperature and pressure distribution.  
% % Temperature in Figure \ref{fig:Cavity_T} is in Kelvins and pressure in 
% %  Figure \ref{fig:Cavity_P} is normalized value $p/(\rho_\infty U_\infty^2)$.
% The time averaged field results generally agree with those from the reference computation \cite{morgenstern1994hypersonic}.
% The computed wall heat flux rate and pressure are compared with the experimental data \cite{hahn1969experimental} in Figure \ref{fig:Cavity1}. To compare the wall heat flux rate and pressure with experimental data
% \cite{hahn1969experimental}, 
% computed values are normalized with heat transfer rate and 
% pressure at a station $x/D=-1.33$. 
% From Figure \ref{fig:Cavity1}, the computed heat transfer rate 
% matches well with experimental data and reference computation.
% The variation near the end of the cavity results from large 
% vortex structures, and the experiment might have failed to 
% provide enough resolution \cite{morgenstern1994hypersonic}.
% The pressure distribution at the bottom of the cavity matches 
% the reference computation, while the experimental pressure values 
% are significantly lower at the start of the cavity. 
% The difference between the current result and 
% the reference computation might be a result of difference in 
% flow condition setups.
% }

% \added[id=r2]
% {
% To investigate the profit from using implicit time marching, 
% time steps and projected explicit time steps $\inc t_E$ are plotted in
% Figure \ref{fig:Cavity_time}. 
% The projected explicit time step size $\inc t_E$ is 
% calculated explicit physical time step CFL constraint being
% \begin{equation}
%     \label{eq:physical-time-step-CFL}
%     \inc t_E = \min_i{\left\{
%     \frac{\CFL \overline{\OO}_i }
%     {\sum_{f \in \partial \OO_i}{A_f}\lambda_{f}}
%     \right\}} 
% \end{equation} 
% using $\CFL = 1$.
% Even without any positivity-preserving restrictions, 
% explicit time marching schemes rely on this criterion to maintain 
% linear stability. It can be observed in Figure \ref{fig:Cavity_time} that 
% after initial transients, implicit time step size maintains a value of $0.01$, 
% while $\inc t_E$ stabilizes lower than $1.2\times 10^{-6}$. 
% This means implicit time marching has generally more than $8000$ times larger
% time steps compared with explicit time marching.
% Considering that each implicit ESDIRK4 step has 5 stages and each stage 
% takes at most 40 inner iterations to solve, and each implicit inner iteration
% is at most $20\%$ more expensive than explicit stages according to Table \ref{tab:cpu-time-cost},
% the implicit scheme still has efficiency more than 30 times higher than explicit ones. 
% If the cost of multi-stage explicit methods and the time step restriction based on some specific 
% positivity-preserving method are considered, the time step efficiency advantage
% of the implicit scheme might be even more significant.
% }

% \begin{figure}[htbp!]
%     \centering
%     \includegraphics[width=0.5\textwidth]{pics/PPRobust_Cavity_t.pdf}
%     \caption{Time step history and projected explicit time step in the 
%     hypersonic cavity problem.}
%     \label{fig:Cavity_time}
% \end{figure}

\added[id=r2]{
A significant advantage of implicit time stepping over explicit time stepping is its ability to accommodate larger time steps, particularly for stiff problems such as high-Reynolds number flows on meshes with large aspect-ratio grids near the wall. To validate this benefit, we conduct two additional simulations of hypersonic flow over an open cavity using second-order finite volume methods with explicit and implicit time integration, respectively.
Both methods employ Green-Gauss reconstruction. Variational reconstruction is not used due to its implicit nature, which requires iterative procedures at each time step, making it inefficient for explicit finite volume schemes due to increased computational cost.
The explicit method uses the third-order strong stability preserving Runge-Kutta (SSPRK3) scheme, while the implicit method uses the fourth-order explicit first-stage, singly diagonally implicit Runge-Kutta (ESDIRK4) scheme. The positivity-preserving algorithm proposed in this work is applied to the implicit scheme, whereas the positivity-preserving algorithm proposed in \cite{zhang2017positivity} is applied to the explicit method.
The algorithm in \cite{zhang2017positivity} employs an LLF-type positivity-preserving flux, a scaling limiter, and the SSPRK time integration ensure  a weak positivity property of finite volume type schemes for compressible Navier-Stokes equations. The time step size for the explicit finite volume scheme is computed as
\begin{equation}
    \label{eq:physical-time-step-CFL}
    \inc t_E = \min_i{\left\{
    \frac{\CFL \overline{\OO}_i }
    {\sum_{f \in \partial \OO_i}{A_f}\lambda_{f}}
    \right\}} ,
\end{equation} 
using a CFL number of $\CFL = 1$, where the spectral radius is computed as described in \eqref{eq:lambda-face-estimation}. According to the positivity-preserving algorithm in \cite{zhang2017positivity}, f negative values arise during the simulation, the time step is halved, and the entire SSPRK3 time integration for that step is recomputed.
}

\added[id=r2]
{The simulations using the two second-order finite volume methods are performed up to $t = 0.1$. The implicit positivity-preserving finite volume method requires only $13$ time steps to reach $t = 0.1$, whereas the explicit positivity-preserving finite volume method requires $153,213$ steps. The evolution of the time step size during the simulation is illustrated in Figure~\ref{fig:Cavity_time_explicitTest}, which indicates that the time step size of the implicit method is approximately four orders of magnitude larger than that of the explicit method. The numerical solutions at $t = 0.1$ are presented in Figure~\ref{fig:Cavity_0d1}, demonstrating that the results obtained from both methods are nearly identical.
In terms of computational cost, the explicit method requires $10,497$ seconds of CPU time, while the implicit method completes the simulation in just $73.8$ seconds, yielding an acceleration factor of more than $140$. These results clearly demonstrate the significant efficiency advantage of implicit time stepping for high-Reynolds number flows on large aspect-ratio grids.
% The simulations using the two second-order finite volume methods are performed to $t=0.1$. 
% The implicit ESDIRK4 positivity-preserving method takes $13$ steps to 
% reach $t=0.1$, while SSPRK3 needs $153,213$ steps. The time step evolution is shown in Figure \ref{fig:Cavity_time_explicitTest}. The numerical solutions at $t=0.1$ are shown in Figure \ref{fig:Cavity_0d1}, which demonstrate that numerical solutions of the two methods are almost identical.
% The CPU time cost of the explicit method is $10,497$ seconds, while only
% $73.8$ seconds for the implicit method, indicating an acceleration factor of more than $140$ by using implicit positivity-preserving time stepping. These results demonstrate the significant advantage of the implicit time stepping in solving high-Reynolds number flows on large aspect ratio grids.
}


% \added[id=r2]
% {
% In order to further investigate the performance of implicit positivity-preserving time marching, 
% an explicit SSPRK3 positivity-preserving scheme based on weak monotonicity of 
% LLF-form viscous flux \cite{zhang2017positivity} is implemented in our finite volume code and run on the hypersonic cavity problem. 
% The original explicit Navier-Stokes positivity-preserving algorithm \cite{zhang2017positivity} adopts a very similar CFL number estimation 
% with the current work, with the $O(\Re \Delta x^2)$ term \cite{zhang2017positivity} provided by the viscous spectral radius estimation
% in \eqref{eq:lambda-face-estimation}.
% The current explicit time step size \eqref{eq:physical-time-step-CFL} should be at least $1/2$ of that in the original work \cite{zhang2017positivity}.
% According to the original work \cite{zhang2017positivity}, in SSPRK3 explicit 
% positivity-preserving method, when updated cell average values encounter 
% negative values, the time step size should be halved and the current time 
% step should be restarted.
% }

% \added[id=r2]
% {
% In the explicit SSPRK3 test, time is only integrated to $0.1$.
% The spatial discretization is switched to 2nd order Green-Gauss reconstruction for 
% both implicit and explicit positivity-preserving time marching.
% Using an explicit reconstruction avoids extra low efficiency on explicit time marching caused by variational reconstruction's implicit nature.
% The time step size history is shown in Figure \ref{fig:Cavity_time_explicitTest}.
% The implicit ESDIRK4 positivity-preserving method takes $13$ steps to 
% reach $t=0.1$, while SSPRK3 needs $153,213$ steps. 
% The actual time step is even smaller than the ones projected from the 
% implicit solution in Figure \ref{fig:Cavity_time}.
% From Figure \ref{fig:Cavity_0d1}, at $t=0.1$, the transient 
% solutions obtained with SSPRK3 and ESDIRK4 are almost identical.
% With the same test environment including hardware, software and 
% solver configuration, SSPRK3 takes $10,497$ seconds while 
% ESDIRK4 takes $73.8$, which means adopting an implicit positivity-preserving 
% time marching in the hypersonic cavity problem grants an acceleration factor of more than $140$.
% }

\begin{figure}[htbp!]
    \centering
    \includegraphics[width=0.55\textwidth]{pics/PPRobust_Cavity_t_explicitTest.pdf}
    \caption{\added[id=r2]{Time step evolution in the simulations of the hypersonic flow past an open cavity problem using the explicit and implicit second-order positivity-preserving finite volume methods.}}
    \label{fig:Cavity_time_explicitTest}
\end{figure}

\begin{figure}[htbp]
    \centering
    \begin{subfigure}{0.5\textwidth}
        \includegraphics[width=\textwidth]{pics/PPRobust_Cavity_0d1_Ma_SSPRK3.png}
        \caption[]{Explicit second-order FV}
    \end{subfigure}\hfill
    \begin{subfigure}{0.5\textwidth}
        \includegraphics[width=\textwidth]{pics/PPRobust_Cavity_0d1_Ma_ESDIRK4.png}
        \caption[]{Implicit second-order FV}
    \end{subfigure}
    \caption{\added[id=r2]{Mach number contours near the trailing edge of the cavity at $t = 0.1$, using 32 contour levels from 0.2 to 6.2.}}
    \label{fig:Cavity_0d1}
\end{figure}





\subsection{Three-dimensional Noh problem}

In the Noh problem \cite{noh1987errors}, an implosion at the origin generates a spherical shock wave with an infinite Mach number, propagating outward from the origin at a constant speed. 
Due to its extreme conditions, the Noh problem has been widely used to evaluate the effectiveness of positivity-preserving numerical schemes for three-dimensional Euler equations \cite{hu2013positivity}.
Following \cite{johnsen2010assessment}, 
in our simulation, the computation domain is set as
$[0,0.256]\times[0,0.256]\times[0,0.256]$. The ratio of specific heat is $\gamma=5/3$.
The initial conditions are
\begin{equation}
\label{eq:initial-condition-Noh}
    \begin{aligned}
        \rho & = 1, \\
        \mathbf{u}  &= -\mathbf{x} /\|\mathbf{x}\|_2, \\
        p & = 10^{-6},\\
    \end{aligned}
\end{equation}
where $\|\mathbf{x}\|_2$ is the distance to the 
origin. The pressure is nominally zero and leads to an infinite Mach number for the imploding flow. While in practical computations, a lower bound on the pressure, such as $10^{-6}$ in \eqref{eq:initial-condition-Noh}, is imposed to prevent complex eigenvalues which would make the problem ill-posed \cite{johnsen2010assessment}.
Symmetric boundary conditions are applied on the three boundary planes passing through the origin. For the remaining boundary planes, inflow pressure and velocity are specified based on the initial conditions, while the time-dependent density is set from the analytical solution.
The analytic solution of density \cite{noh1987errors} is 
\begin{equation}
    \rho = 
    \begin{dcases}
        64, & \text{if } \ \|\mathbf{x}\|_2<t/3, \\
        (1+t/\|\mathbf{x}\|_2)^2,  & \text{else}.
    \end{dcases}
\end{equation}
The simulation is performed on a uniform hexahedral mesh with a grid size of $\inc x = \inc y= \inc z = 10^{-3}$, using a maximum time step size of $\inc t_{max}=5 \times 10^{-4}$. The simulation is terminated at $t=0.6$ to ensure that the 
shock does not reach the inflow boundaries. 
In each inner iteration, the CFL number for local pseudo time step $\CFLtau$ is initiated as $0.1$
and increased gradually up to its maximum $2$ at the tenth iteration.
The reference density and pressure values used in reconstruction polynomial scaling are $\rho_0=1$ and $p_0=10^{-6}$, respectively. 

Figures \ref{fig:noh0} and \ref{fig:noh1} present the numerical solution at $t = 0.6$, showing that neither negative density nor negative pressure appears.
% The density profiles in Figure \ref{sfig:noh0-b} and the internal energy profiles in Figure \ref{sfig:noh0-e} are
% close to those presented in \cite{hu2013positivity}. 
The profiles in Figure \ref{fig:noh1} confirm that the computed shock position and the fluid conditions behind the shock are accurately captured. The discrepancy from the analytical solution near the origin arises from the three-dimensional nature of the system.
Compared with the results computed on a $\inc x = \inc y= \inc z = 2 \times 10^{-3}$ Cartesian mesh using fifth- to tenth-order hybrid WENO/central difference schemes in \cite{johnsen2010assessment}, 
the density profiles in Figure \ref{sfig:noh0-r} exhibit smaller errors near the origin.  
Likewise, compared with the results computed on a $200 \times 30$ polar mesh by using a second-order Lagrangian discontinuous Galerkin scheme in \cite{li2014cell},
the internal energy profiles in Figure \ref{sfig:noh0-e} show notably lower errors near the origin.
Overall, these results for the Noh problem underscore the effectiveness of the proposed positivity-preserving algorithm for three-dimensional cases.

% \begin{figure}[htbp]
%     \centering
%     \begin{subfigure}{0.4\textwidth}
%         \includegraphics[width=\textwidth]{pics/PPRobust_NohR-256.png}
%         \caption[]{density}
%         \label{sfig:noh0-a}
%     \end{subfigure}
%     \hfill
%     \begin{subfigure}{0.45\textwidth}
%         \includegraphics[width=\textwidth]
%         {pics/PPRobust_Noh-p-256.pdf}
%         \caption[]{pressure}
%         \label{sfig:noh0-p}
%     \end{subfigure}
%     \caption{Density contour and pressure profiles of the Noh problem at $t=0.6$.}
%     \label{fig:noh0}
% \end{figure}

\begin{figure}[htbp!]
    \centering
    \includegraphics[width=0.49\textwidth]{pics/PPRobust_NohR-256.png}
    \caption{Density contour of the Noh problem at $t=0.6$.}
    \label{fig:noh0}
\end{figure}

\begin{figure}[htbp!]
    \centering
    \begin{subfigure}{0.49\textwidth}
        \includegraphics[width=\textwidth]
        {pics/PPRobust_Noh-p-256.pdf}
        \caption[]{Pressure}
        \label{sfig:noh0-p}
    \end{subfigure}
    \hfill
    \begin{subfigure}{0.49\textwidth}
        \includegraphics[width=\textwidth]
        {pics/PPRobust_Noh-rho-256.pdf}
        \caption[]{Density}
        \label{sfig:noh0-r}
    \end{subfigure}
    \hfill
    \vspace{5mm}
    \begin{subfigure}{0.49\textwidth}
        \includegraphics[width=\textwidth]
        {pics/PPRobust_Noh-e-256.pdf}
        \caption[]{Internal energy}
        \label{sfig:noh0-e}
    \end{subfigure}
    \caption{Solution profiles along the $y=z=0$ (axial) and $y=x,z=0$ (diagonal) lines of the Noh problem at $t=0.6$.}
    \label{fig:noh1}
\end{figure}
% !TeX root = main.tex

\section{Conclusions}
\label{sec:Conclusions}

This paper presents a positivity-preserving algorithm for finite volume methods using dual-time stepping for compressible flow simulations.
In the positivity-preserving algorithm, admissible cell averages are obtained by limiting solution changes,
which is accomplished by controlling physical and pseudo time step sizes.
To overcome the difficulty of unknown solution changes,
we employ explicit time discretizations to obtain efficient estimations of future states.
The allowable time step sizes are determined by limiting the relative solution changes.
Given positive cell averages,
admissible reconstruction polynomials can be obtained by applying a positivity-preserving scaling limiter.
The proposed positivity-preserving algorithm is accuracy-preserving.
Numerical results for a series of benchmark test cases demonstrate the high accuracy, high resolution and robustness of the positivity-preserving implicit high-order finite volume method.
% !TeX root = main.tex

\section*{Acknowledgments}

This work is supported by National Natural Science Foundation of China (Grants 12372284 and U2230402).
% % !TeX root = main.tex

% Start of Appendix
\appendix

\section{Lower bound of the limited physical time step size}
\label{sec:lower-bound-physical}
An analysis can be performed on the lower bound of the physical time step size.
The right-hand-side of the semi-discrete finite volume scheme is computed by
\begin{equation}
    \label{eq:Semi-FV-1}
    \R_i = -\frac{1}{\overline{\OO}_i} \oint_{\partial \OO_i} \left(\tilde{\F} - \tilde{\F}_v \right) \cdot \n \ \dd A.
\end{equation}
On a cell interface $f$, a spectral radius $\tilde{\lambda}_f>0$ can be found such that
\begin{equation}
    \left\| \left(\tilde{\F} - \tilde{\F}_v \right) \cdot \n \right\| \leq \tilde{\lambda}_f \min \left\{\|\UM_L\|, \|\UM_R\| \right\},
\end{equation}
where $\|\cdot\|$ denotes the $L^2$ norm, if the ratio $\max \left\{\|\UM_L\|, \|\UM_R\| \right\}/\min \left\{\|\UM_L\|, \|\UM_R\| \right\}$ is bounded.
Therefore, we have
\begin{equation}
    \left\| \R_i \right\| 
    \leq \frac{1}{\overline{\OO}_i} \sum_{f \in \partial \Omega_i}{
        A_{f} \tilde{\lambda}_{f} \min \left\{\|\UM_L\|, \|\UM_R\| \right\}
    } 
    \leq \frac{\sum_{f \in \partial \Omega_i} A_f\tilde{\lambda}_f}{\overline{\OO}_i}\|\UM_i\|
    =
    \frac{\CFL_i}{\inc t_{max}}\|\UM_i\|,
\end{equation}
where $\CFL_i= \inc t_{max}\sum_{f \in \partial \Omega_i} A_f \tilde{\lambda}_f/\overline{\OO}_i$. 
Given that $\UM_i$ is finite and non-singular, we can find $\mathcal{O}(1)$ coefficients $C^{\rho}_i$, $C^{\rho E}_i$ and $C^{\rho \uv}_i$ such that
\begin{equation}
    \begin{aligned}
        \left| \rho\left( \frac{\inc t_{max}}{\CFL_i} \R_i \right) \right|
        & \leq
        C^\rho_i \rho\left( \UM_i \right),\\
        \left| \rho E\left( \frac{\inc t_{max}}{\CFL_i} \R_i \right) \right|
        & \leq
        C^{\rho E}_i \rho E\left( \UM_i \right),\\
        \left\| \rho \uv\left( \frac{\inc t_{max}}{\CFL_i} \R_i \right) \right\|
        & \leq
        C^{\rho \uv}_i \left\| \rho \uv\left( \UM_i \right)              \right\|,\\
    \end{aligned}
\end{equation}
where $\rho(\U)$, $\rho E(\U)$ and $\rho \uv(\U)$ are linear functions as they take directly the components of $\U$.
% $C^{\rho}_i$, $C^{\rho E}_i$ and $C^{\rho \uv}_i$ are $\mathcal{O}(1)$ coefficients determined by $\UM_i$.
We can show that $\alpha_{t,i}^{\rho}$ has a lower bound.
By substituting the following inequality 
\begin{equation}
    |\delta \rho_i^n| = \left|\rho\left(\UM^n_i + \inc t_{max} \R^n_i\right) - \rho\left(\UM^n_i\right)\right|
    =
    \left|\rho\left(\inc t_{max} \R^n_i\right)\right|
    \leq
    \CFL_i C^\rho_i \rho\left(\UM^n_i\right),
\end{equation}
into \eqref{eq:alpha-t-rho}, we have
\begin{equation}
    \label{eq:alpha-t-rho-lb}
    \alpha_{t,i}^{\rho} \geq \frac{\eta_t}{\CFL_i C_i^\rho},
\end{equation}
which indicates a finite lower bound for $\alpha_{t,i}^{\rho}$. 

\newcommand{\uincT}{\inc\U_i^{n,\rho}}
Next, we derive a lower bound for $\alpha_{t,i}^{p}$ using a similar approach.
We define $\inc\U_i^{n,\rho} =\alpha_{t,i}^{\rho} \inc t_{max} \R^n_i$.
According to \eqref{eq:alpha-t-rho-ineq}, we have 
\begin{equation}
    \rho\left(\UM^n_i+\uincT  \right) \geq (1-\eta_t)\rho\left(\UM^n_i\right).
\end{equation}
By applying the Cauchy-Schwarz inequality and the Triangle inequality, we obtain
% {\small
\begin{equation}
    \begin{aligned}
        \dfrac{|\delta p_i^n|}{\gamma-1} 
        &=  \dfrac{1}{\gamma-1} \left| p \left(\UM^n_i + \inc\U_i^{n,\rho}\right) -  p \left(\UM^n_i\right) \right| \\
        &=  \left| \rho E\left( \uincT \right) 
         - \frac{
         \rho\uv\left(\uincT\right)^2
         +
         \rho\uv\left(\UM^n_i\right)^2
         +
         2\rho\uv\left(\uincT\right)\cdot\rho\uv\left(\UM^n_i\right)
         }{
         2\rho\left(\UM^n_i+\uincT  \right)
         }
         + \frac{ \rho\uv\left(\UM^n_i\right)^2}{2\rho\left(\UM^n_i\right)}
        \right| \\
        &\leq  \left| \rho E\left( \uincT \right) \right|
         +
         \dfrac{
         \rho\uv\left(\uincT\right)^2
         +
         \rho\uv\left(\UM^n_i\right)^2
         +
         2\rho\uv\left(\uincT\right)\cdot\rho\uv\left(\UM^n_i\right)
         }{
         2\rho\left(\UM^n_i+\uincT  \right)
         }
         +
          \frac{ \rho\uv\left(\UM^n_i\right)^2}{2\rho\left(\UM^n_i\right)} \\
        &\leq  \left| \rho E\left( \uincT \right) \right|
         +
         \frac{1}{1-\eta_t}
         \frac{
         \rho\uv\left(\uincT\right)^2
         +
         \rho\uv\left(\UM^n_i\right)^2
         +
         2\rho\uv\left(\uincT\right)\cdot\rho\uv\left(\UM^n_i\right)
         }{
         2\rho\left(\UM^n_i \right)
         }
         +
          \frac{ \rho\uv\left(\UM^n_i\right)^2}{2\rho\left(\UM^n_i\right)} \\
        &\leq \alpha_{t,i}^\rho\CFL_i C^{\rho E}_i  \rho E\left( \UM^n_i \right) 
        +\left(
        \frac{(\alpha_{t,i}^\rho\CFL_i C^{\rho \uv}_i)^2}{1-\eta_t}
        +\frac{2\alpha_{t,i}^\rho\CFL_i C^{\rho \uv}_i}{1-\eta_t}
        + \frac{2-\eta_t}{1-\eta_t}
        \right)
        \frac{ \rho\uv\left(\UM^n_i\right)^2}{2\rho\left(\UM^n_i\right)},
    \end{aligned}
\end{equation}
% }
where $\rho\uv()^2$ is the short form of $\rho\uv()\cdot\rho\uv()$.
As $p\left(\UM^n_i\right) > 0$, we have
\begin{equation}
    \frac{ \rho\uv\left(\UM^n_i\right)^2}{2\rho\left(\UM^n_i\right)} < \rho E\left(\UM^n_i\right),
\end{equation}
and thus
\begin{equation}
    \begin{aligned}
        |\delta p_i^n| &\leq
        (\gamma-1)\left[
         \alpha_{t,i}^\rho\CFL_i C^{\rho E}_i  
        +\left(
        \frac{(\alpha_{t,i}^\rho\CFL_i C^{\rho \uv}_i)^2}{1-\eta_t}
        +\frac{2\alpha_{t,i}^\rho\CFL_i C^{\rho \uv}_i}{1-\eta_t}
        + \frac{2-\eta_t}{1-\eta_t}
        \right)
        \right]
        \rho E\left( \UM^n_i \right).
    \end{aligned}
\end{equation}
According to \eqref{eq:alpha-t-p}, we obtain
\begin{equation}
    \begin{aligned}
        \label{eq:alpha-t-p-lb}
        \alpha_{t,i}^{p} 
        & \geq \dfrac{\eta_t}{\gamma-1} \frac{p\left(\UM^n_i\right)}{
            \rho E\left(\UM^n_i\right)
        }
        \frac{1}{
         \alpha_{t,i}^\rho\CFL_i C^{\rho E}_i  
            +\left(
            \frac{(\alpha_{t,i}^\rho\CFL_i C^{\rho \uv}_i)^2}{1-\eta_t}
            +\frac{2\alpha_{t,i}^\rho\CFL_i C^{\rho \uv}_i}{1-\eta_t}
            + \frac{2-\eta_t}{1-\eta_t}
            \right)
        } \\
        & = \eta_t \frac{1}{
           1 + \frac{\gamma (\gamma-1) Ma_i^2}{2}
        }
        \frac{1}{
         \alpha_{t,i}^\rho\CFL_i C^{\rho E}_i  
            +\left(
            \frac{(\alpha_{t,i}^\rho\CFL_i C^{\rho \uv}_i)^2}{1-\eta_t}
            +\frac{2\alpha_{t,i}^\rho\CFL_i C^{\rho \uv}_i}{1-\eta_t}
            + \frac{2-\eta_t}{1-\eta_t}
            \right)
        },
    \end{aligned}
\end{equation}
where $Ma_i$ is the Mach number based on $\UM^n_i$. 
Consequently, the global time step size is bounded below by
\begin{equation}
    \label{eq:delta-t-lb}
    \frac{\inc t^n}{\inc t_{max}} \geq 
    \eta_t^2
    \min_i
    \left\{
    \frac{1}{
           1 + \frac{\gamma (\gamma-1) Ma_i^2}{2}
        }
    \frac{1}{\CFL_i C_i^\rho}
    \frac{1}{
     \alpha_{t,i}^\rho\CFL_i C^{\rho E}_i  
        +\left(
        \frac{(\alpha_{t,i}^\rho\CFL_i C^{\rho \uv}_i)^2}{1-\eta_t}
        +\frac{2\alpha_{t,i}^\rho\CFL_i C^{\rho \uv}_i}{1-\eta_t}
        + \frac{2-\eta_t}{1-\eta_t}
        \right)
    }
    \right\}.
\end{equation}
We assume that the state $\UM^n_i$ yields a finite Mach number $Ma_i$.
Under the conditions that $\eta_t\sim\mathcal{O}(1)$, $1-\eta_t\sim\mathcal{O}(1)$,
$\CFL_i\sim\mathcal{O}(1)$ and ${1}/\left(1 + \frac{\gamma (\gamma-1) Ma_i^2}{2}\right)\sim\mathcal{O}(1)$,
$\inc t^n$ is not infinitely small compared to $\inc t_{max}$.
In other words, when the CFL number $\CFL_i$ based on $\inc t_{max}$ is $\mathcal{O}(1)$, 
the CFL number determined by the scaled $\inc t^n$ also remains $\mathcal{O}(1)$.


%%%% BIBLIOGRAPHY
\bibliographystyle{elsarticle-num}
\bibliography{refs}

\end{document}
