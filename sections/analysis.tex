% !TeX root = main.tex
\begingroup
\color{r2color}

\section{Analysis of the positivity-preserving algorithm}

This section analyzes the properties of the positivity-preserving algorithm introduced in Section \ref{sec:PP}, including its accuracy-preserving behavior, sensitivity to parameter selection, bounds on the time step size, and overall applicability.

\subsection{Accuracy-preserving property}

The positivity-preserving algorithm consists of four fundamental procedures: physical time step limiting, pseudo time step limiting, increment correction, and reconstruction polynomial scaling. In this subsection, we analyze the impact of these procedures on the accuracy of the overall implicit finite volume method.

It is well-established that reducing the physical time step size does not impact the order of accuracy of the time integration.
% Generally, pseudo time step limiting and increment correction may be activated during the early stages of the inner iteration to stabilize the solution and are deactivated in later stages as the solution converges. 
% {\color{r1color}Even if pseudo-time step limiting and increment correction remain active, the unchanged convergence criterion ensures that the converged solution satisfies the nonlinear equation system with sufficient accuracy.}
% Therefore, these techniques do not affect the accuracy of the converged solution in the inner iteration process.
In general, pseudo time step limiting and increment correction are often activated during the initial stages of the inner iteration to stabilize the solution. These techniques are typically deactivated in later stages as the solution converges. Importantly, even if pseudo time step limiting and increment correction remain active, the unchanged convergence criterion ensures that the converged solution satisfies the nonlinear equation system with sufficient accuracy. As a result, these stabilization techniques in the inner iteration process do not compromise the accuracy of the converged solution.
Furthermore, as noted earlier, the scaling limiter effectively preserves the 
accuracy of the reconstruction in smooth regions \cite{zhang2010positivity}.

In conclusion, the positivity-preserving algorithm preserves accuracy in both space and time. The accuracy-preserving property is demonstrated through an accuracy test on a smooth vortex problem in Section \ref{ssec:accuracy-test}.

\subsection{Parameter sensitivity}
\label{ssec:influence-parameters}

The positivity-preserving algorithm includes three important parameters, $\eta_t$, $\eta_{\tau}$, and $\eta_{\inc}$, which are used to control the allowable relative changes in the solution. These parameters are all defined in the range $(0,1)$. Larger values of $\eta_t$ and $\eta_{\tau}$ result in larger physical and pseudo time step sizes, respectively. Larger values of $\eta_{\inc}$ results in larger increments, which may lead to faster convergence in later iteration steps while higher stiffness in earlier iteration steps. 

There is an upper bound for $\eta_t$ beyond which the time step limiting may fail to guarantee positivity of the future state, which is the solution at the next time level. As mentioned in Section \ref{ssec:physical-limiting}, to ensure positivity of the future state, a lower bound is imposed on the estimated future state, and the lower bound must be larger than the estimation error. An analysis of the positivity preservation of density is performed to show the properties of the upper bound of $\eta_t$. The estimated future state is computed by using a simple linear approximation as below:
\begin{equation} 
    \uu^{n+1,*}=\uu^n + \inc t^n \R^n,
\end{equation}
which has an error of order $\mathcal{O}\left(\left(\inc t^n\right)^2\right)$.
As any time integration scheme with at least 
1st order accuracy has truncation error 
of order $\mathcal{O}\left(\left(\inc t^n\right)^k\right),k\geq2$
(ESDIRK4's error being $\mathcal{O}\left(\left(\inc t^n\right)^5\right)$),
the error between real numerical implicit solution and estimated
solution is still $\mathcal{O}\left(\left(\inc t^n\right)^2\right)$.
It is observed from \eqref{eq:alpha-t-rho} that, the limited time step size $\inc t^n$ is proportional to $\eta_t$ and $\inc t_{max}$. Therefore, the estimation error can be expressed as
\begin{equation}
    E_\rho= K_\rho \left(\eta_t\right)^2 \rho\left(\uu^n\right) , \quad K_\rho \sim \mathcal{O}\left(\left(\inc t_{max}\right)^2\right).
\end{equation}
A lower bound is imposed on the density of the estimated future state, i.e.,
\begin{equation}
    \rho\left(\uu^{n+1,*}\right) \geq \rho_{min}= \left(1-\eta_t\right) \rho\left(\uu^n\right).
\end{equation}
To ensure the positivity of the future state, it is required that
\begin{equation}
    \rho_{min}= \left(1-\eta_t\right) \rho\left(\uu^n\right) > E_\rho= K_\rho \left(\eta_t\right)^2 \rho\left(\uu^n\right),
\end{equation}
which is simplified as 
\begin{equation}
    1-\eta_t > K_\rho \left(\eta_t\right)^2.
\end{equation}
Therefore, there is a critical upper bound $\eta^c_t$ that satisfies 
\begin{equation}
    1-\eta^c_t = K_\rho \left(\eta^c_t\right)^2
    % ,\ \ 
    % \eta_t^c = \frac{\sqrt{1 + 4K_\rho}-1}{2K_\rho}\approx1-K_\rho
\end{equation}
as shown in Figure \ref{fig:eta-t-curves}.
The parameter $\eta_t$ should be selected in the range $\left(0,\eta^c_t\right)$. 
% It is observed from Figure \ref{fig:eta-t-curves} that, 
It is obtained through analysis that, 
a smaller $K_\rho$ yields a larger $\eta^c_t$,
which can also be observed from Figure \ref{fig:eta-t-curves}. 
As $K_\rho$ is of 
order $\mathcal{O}\left(\left(\inc t_{max}\right)^2\right)$
and $1-\eta_t^c\sim K_\rho$,
the upper bound of $\eta_t$ is close to one, allowing an easy parameter selection. 
A similar analysis can be performed to determine the upper bound of $\eta_t$ that can further ensure positivity of pressure. 
In actual implementation, it is difficult to compute the solution dependent coefficient $K_\rho$, thus not feasible to obtain the exact upper bound $\eta_t$. 
Meanwhile, the real future state is 
obtained through a series of positivity-preserving inner iterations,
thus the only criterion for detecting an inadmissible real future 
state is the divergence of inner iterations.
It is observed in numerical results, when physical time step is 
too large (using no limiting for example), the solution might   
fail to converge and the norm of residual blows up. 
In the diverged scenarios, due to the usage of pseudo time limiting 
and increment correction, the latest solution is always admissible, 
but exponentially growing pressure or density is observed, rendering the 
solution non-physical. 
Based on the fact that the upper bound is quite close to one, an 
iterative posteriori approach can be used to determine $\eta_t$ as follows:
% \begin{enumerate} [label=(\alph*)]
%     \item try a parameter value $\eta^0_t$ to perform one time step computation;
%     \item if (a) fails, go back to the initial state, try with a halved parameter value;
%     \item if (b) fails, try (b) again.
% \end{enumerate}
\begin{enumerate} [label=(\alph*)]
    \item initialize the parameter as $\eta_t= \eta^0_t \in \left(0,1\right)$;
    \item compute a limited time step size $\inc t^n$, and then perform one step time integration to obtain stage values $\uu^{(s)},s=2,3\dots6$ and updated solution $\uu^{n+1}$;
    \item if $\uu^{(s)},s=2,3\dots6$ with $\uu^{(6)}=\uu^{n+1}$ all converged, proceed to next step; otherwise, try (b) again with a halved $\eta_t$.
\end{enumerate}
Since the upper bound $\eta^c_t$ is of order $\mathcal{O}(1)$, a value of $\eta_t$ that preserves positivity can always be found within a finite number of iterations. 
In the numerical experiments presented in Section~\ref{sec:Results}, the parameter is initialized as $\eta_t = 0.8$ for all test cases. 
This choice does not lead to solution divergence and therefore the iterative posteriori search for this parameter is not triggered.

\begin{figure}[htbp!]
    \centering
    \includegraphics[width=0.45\linewidth]{pics/eta-t-curves.pdf}
    \caption{Lower and upper bounds of the parameter $\eta_t$.}
    \label{fig:eta-t-curves}
\end{figure}

% There is not an upper bound for the parameter $\eta_\tau$ to be used to preserve positivity, since the pseudo time step limiting is followed by an increment correction that can guarantee positivity of the updated intermediate states. There is also not an upper bound for the parameter $\eta_{\inc}$, as the lower bounds are directly imposed on the updated intermediate states.
There is no upper bound required for the parameter $\eta_\tau$ to preserve positivity, 
as the pseudo-time step limiting is followed by an increment correction 
that ensures the positivity of the updated intermediate states. 
Similarly, no upper bound is needed for the parameter $\eta_{\inc}$, 
since the lower bounds are directly enforced on the updated intermediate states. 
However, there exists an asymptotic relation between $\eta_t$ and $\eta_\inc$,
which is discussed in Section \ref{ssec:analysis-time-step}.
A preferable choice is to make $\eta_\inc > \eta_\tau$, so that,
asymptotically, increment correction does not take effect after 
pseudo time limiting and primarily act as a safe guard.
In summary, the current work 
chooses both $\eta_\tau$ and $\eta_{\inc}$ to be within the range $(0,1)$, 
with $\eta_\inc > \eta_\tau$ required. 

\subsection{Time step size bounds}
\label{ssec:analysis-time-step}
In this subsection, we analyze the upper and lower bounds of the physical and pseudo time step sizes. As mentioned in Section \ref{ssec:physical-limiting}, the upper bound of the limited physical time step is the user-defined constant $\inc t_{max}$.
It is observed from \eqref{eq:alpha-tau} that, the upper bound of the limited pseudo time step for cell $i$ is the original pseudo time step $\inc \tau_i$, which is computed according to \eqref{eq:local-pseudo-time-step}. 
The following of this subsection analyzes the lower bounds
of the physical time step and pseudo time step after positivity-preserving limiting. 
Along with the pseudo time step's lower bound, the effect of 
increment correction is also investigated.

\subsubsection{Lower bound for physical time step limiting}

An analysis can be performed on the lower bound of the physical time step size.
The right-hand side of the semi-discrete finite volume scheme is computed by
\begin{equation}
    \label{eq:Semi-FV-1}
    \R_i = -\frac{1}{\overline{\OO}_i} \oint_{\partial \OO_i} \left(\tilde{\F} - \tilde{\F}_v \right) \cdot \n \ \dd A.
\end{equation}
On a cell interface $f$, a spectral radius $\tilde{\lambda}_f>0$ can be found such that
\begin{equation}
    \left\| \left(\tilde{\F} - \tilde{\F}_v \right) \cdot \n \right\| \leq \tilde{\lambda}_f \min \left\{\|\UM_L\|, \|\UM_R\| \right\},
\end{equation}
where $\|\cdot\|$ denotes the $L^2$ norm, if the ratio $\max \left\{\|\UM_L\|, \|\UM_R\| \right\}/\min \left\{\|\UM_L\|, \|\UM_R\| \right\}$ is bounded.
Therefore, we have
\begin{equation}
\label{eq:upper-bound-residual}
    \left\| \R_i \right\| 
    \leq \frac{1}{\overline{\OO}_i} \sum_{f \in \partial \Omega_i}{
        A_{f} \tilde{\lambda}_{f} \min \left\{\|\UM_L\|, \|\UM_R\| \right\}
    } 
    \leq \frac{\sum_{f \in \partial \Omega_i} A_f\tilde{\lambda}_f}{\overline{\OO}_i}\|\UM_i\|
    =
    \frac{\CFL_i}{\inc t_{max}}\|\UM_i\|,
\end{equation}
where $\CFL_i= \inc t_{max}\sum_{f \in \partial \Omega_i} A_f \tilde{\lambda}_f/\overline{\OO}_i$. 
Given that $\UM_i$ is finite and non-singular, we can find $\mathcal{O}(1)$ coefficients $C^{\rho}_i$, $C^{\rho E}_i$ and $C^{\rho \uv}_i$ such that
\begin{equation}
    \begin{aligned}
    \label{eq:inc_upper_cond_const_coef}
        \left| \rho\left( \inc t_{max} \R_i \right) \right|
        & \leq
        C^\rho_i    \CFL_i \rho\left( \UM_i \right),\\
        \left| \rho E\left( \inc t_{max} \R_i \right) \right|
        & \leq
        C^{\rho E}_i \CFL_i \rho E\left( \UM_i \right),\\
        \left\| \rho \uv\left( \inc t_{max} \R_i \right) \right\|
        & \leq 
        C^{\rho \uv}_i \CFL_i  \left\| \rho \uv\left( \UM_i \right)              \right\|,\\
    \end{aligned}
\end{equation}
where $\rho(\U)$, $\rho E(\U)$ and $\rho \uv(\U)$ are linear functions as they take directly the components of $\U$.
% $C^{\rho}_i$, $C^{\rho E}_i$ and $C^{\rho \uv}_i$ are $\mathcal{O}(1)$ coefficients determined by $\UM_i$.
We can show that $\alpha_{t,i}^{\rho}$ has a lower bound.
By substituting the following inequality 
\begin{equation}
    |\delta \rho_i^n| = \left|\rho\left(\UM^n_i + \inc t_{max} \R^n_i\right) - \rho\left(\UM^n_i\right)\right|
    =
    \left|\rho\left(\inc t_{max} \R^n_i\right)\right|
    \leq
    \CFL_i C^\rho_i \rho\left(\UM^n_i\right),
\end{equation}
into \eqref{eq:alpha-t-rho}, we have
\begin{equation}
    \label{eq:alpha-t-rho-lb}
    \alpha_{t,i}^{\rho} \geq \frac{\eta_t}{\CFL_i C_i^\rho},
\end{equation}
which indicates a finite lower bound for $\alpha_{t,i}^{\rho}$. 

\newcommand{\uincT}{\inc\U_i^{n,\rho}}
Next, we derive a lower bound for $\alpha_{t,i}^{p}$ using a similar approach.
We define $\inc\U_i^{n,\rho} =\alpha_{t,i}^{\rho} \inc t_{max} \R^n_i$.
According to \eqref{eq:alpha-t-rho-ineq}, we have 
\begin{equation}
    \rho\left(\UM^n_i+\uincT  \right) \geq (1-\eta_t)\rho\left(\UM^n_i\right).
\end{equation}
By applying the Cauchy-Schwarz inequality and the Triangle inequality, we obtain
% {\small
\begin{equation}
    \begin{aligned}
        \dfrac{|\delta p_i^n|}{\gamma-1} 
        &=  \dfrac{1}{\gamma-1} \left| p \left(\UM^n_i + \inc\U_i^{n,\rho}\right) -  p \left(\UM^n_i\right) \right| \\
        &=  \left| \rho E\left( \uincT \right) 
         - \frac{
         \rho\uv\left(\uincT\right)^2
         +
         \rho\uv\left(\UM^n_i\right)^2
         +
         2\rho\uv\left(\uincT\right)\cdot\rho\uv\left(\UM^n_i\right)
         }{
         2\rho\left(\UM^n_i+\uincT  \right)
         }
         + \frac{ \rho\uv\left(\UM^n_i\right)^2}{2\rho\left(\UM^n_i\right)}
        \right| \\
        &\leq  \left| \rho E\left( \uincT \right) \right|
         +
         \dfrac{
         \rho\uv\left(\uincT\right)^2
         +
         \rho\uv\left(\UM^n_i\right)^2
         +
         2\rho\uv\left(\uincT\right)\cdot\rho\uv\left(\UM^n_i\right)
         }{
         2\rho\left(\UM^n_i+\uincT  \right)
         }
         +
          \frac{ \rho\uv\left(\UM^n_i\right)^2}{2\rho\left(\UM^n_i\right)} \\
        &\leq  \left| \rho E\left( \uincT \right) \right|
         +
         \frac{1}{1-\eta_t}
         \frac{
         \rho\uv\left(\uincT\right)^2
         +
         \rho\uv\left(\UM^n_i\right)^2
         +
         2\rho\uv\left(\uincT\right)\cdot\rho\uv\left(\UM^n_i\right)
         }{
         2\rho\left(\UM^n_i \right)
         }
         +
          \frac{ \rho\uv\left(\UM^n_i\right)^2}{2\rho\left(\UM^n_i\right)} \\
        &\leq \alpha_{t,i}^\rho\CFL_i C^{\rho E}_i  \rho E\left( \UM^n_i \right) 
        +\left(
        \frac{(\alpha_{t,i}^\rho\CFL_i C^{\rho \uv}_i)^2}{1-\eta_t}
        +\frac{2\alpha_{t,i}^\rho\CFL_i C^{\rho \uv}_i}{1-\eta_t}
        + \frac{2-\eta_t}{1-\eta_t}
        \right)
        \frac{ \rho\uv\left(\UM^n_i\right)^2}{2\rho\left(\UM^n_i\right)},
    \end{aligned}
\end{equation}
% }
where $\rho\uv()^2$ is the short form of $\rho\uv()\cdot\rho\uv()$.
As $p\left(\UM^n_i\right) > 0$, we have
\begin{equation}
    \frac{ \rho\uv\left(\UM^n_i\right)^2}{2\rho\left(\UM^n_i\right)} < \rho E\left(\UM^n_i\right),
\end{equation}
and thus
\begin{equation}
    \begin{aligned}
        |\delta p_i^n| &\leq
        (\gamma-1)\left[
         \alpha_{t,i}^\rho\CFL_i C^{\rho E}_i  
        +\left(
        \frac{(\alpha_{t,i}^\rho\CFL_i C^{\rho \uv}_i)^2}{1-\eta_t}
        +\frac{2\alpha_{t,i}^\rho\CFL_i C^{\rho \uv}_i}{1-\eta_t}
        + \frac{2-\eta_t}{1-\eta_t}
        \right)
        \right]
        \rho E\left( \UM^n_i \right).
    \end{aligned}
\end{equation}
According to \eqref{eq:alpha-t-p}, we obtain
\begin{equation}
    \begin{aligned}
        \label{eq:alpha-t-p-lb}
        \alpha_{t,i}^{p} 
        & \geq \dfrac{\eta_t}{\gamma-1} \frac{p\left(\UM^n_i\right)}{
            \rho E\left(\UM^n_i\right)
        }
        \frac{1}{
         \alpha_{t,i}^\rho\CFL_i C^{\rho E}_i  
            +\left(
            \frac{(\alpha_{t,i}^\rho\CFL_i C^{\rho \uv}_i)^2}{1-\eta_t}
            +\frac{2\alpha_{t,i}^\rho\CFL_i C^{\rho \uv}_i}{1-\eta_t}
            + \frac{2-\eta_t}{1-\eta_t}
            \right)
        } \\
        & = \eta_t \frac{1}{
           1 + \frac{\gamma (\gamma-1) Ma_i^2}{2}
        }
        \frac{1}{
         \alpha_{t,i}^\rho\CFL_i C^{\rho E}_i  
            +\left(
            \frac{(\alpha_{t,i}^\rho\CFL_i C^{\rho \uv}_i)^2}{1-\eta_t}
            +\frac{2\alpha_{t,i}^\rho\CFL_i C^{\rho \uv}_i}{1-\eta_t}
            + \frac{2-\eta_t}{1-\eta_t}
            \right)
        },
    \end{aligned}
\end{equation}
where $Ma_i$ is the Mach number based on $\UM^n_i$. 
Consequently, the global time step size is bounded below by
\begin{equation}
    \label{eq:delta-t-lb}
    \frac{\inc t^n}{\inc t_{max}} \geq 
    \eta_t^2
    \min_i
    \left\{
    \frac{1}{
           1 + \frac{\gamma (\gamma-1) Ma_i^2}{2}
        }
    \frac{1}{\CFL_i C_i^\rho}
    \frac{1}{
     \alpha_{t,i}^\rho\CFL_i C^{\rho E}_i  
        +\left(
        \frac{(\alpha_{t,i}^\rho\CFL_i C^{\rho \uv}_i)^2}{1-\eta_t}
        +\frac{2\alpha_{t,i}^\rho\CFL_i C^{\rho \uv}_i}{1-\eta_t}
        + \frac{2-\eta_t}{1-\eta_t}
        \right)
    }
    \right\}.
\end{equation}
We assume that the state $\UM^n_i$ yields a finite Mach number $Ma_i$.
Under the conditions that $\eta_t\sim\mathcal{O}(1)$, $1-\eta_t\sim\mathcal{O}(1)$,
$\CFL_i\sim\mathcal{O}(1)$ and ${1}/\left(1 + \frac{\gamma (\gamma-1) Ma_i^2}{2}\right)\sim\mathcal{O}(1)$,
$\inc t^n$ is not infinitely small compared to $\inc t_{max}$.
In other words, when the CFL number $\CFL_i$ based on $\inc t_{max}$ is $\mathcal{O}(1)$, 
the CFL number determined by the scaled $\inc t^n$ also remains $\mathcal{O}(1)$.

\subsubsection{Lower bound for pseudo time step limiting}

A similar analysis can be carried out for the lower bounds of the local pseudo-time step sizes. In this case, the residual $\R_i$ is simply replaced with $\tilde{\R}_i^{(s,m)}$, for which an inequality analogous to \eqref{eq:upper-bound-residual} can be derived in the same spirit. 

For $\inc\tau$ analysis, 
the implicit time-stepping  for stage $s$ in \eqref{eq:define-inc}
has the form of
\begin{equation}
    \begin{aligned}
        \tilde{\R}_i\supsm & = \sum_{q=1}^{s-1} a_{sq} \R_i^{\left(q\right)} + 
    a_{ss} \R_i\supsm - 
    \dfrac{\uu_i\supsm - \uu_i^{n}}{\inc t^n} \\
    & =  a_{ss} \R_i\supsm - 
    \dfrac{\uu_i\supsm}{\inc t^n}
    + \hat{\R}_i,
    \end{aligned}
\end{equation}
where $\hat{\R}_i$ is the constant term independent of the unknown
$\uu_i\supsm$. 
Therefore, \eqref{eq:upper-bound-residual} becomes
\begin{equation}
\label{eq:upper-bound-residual-tau}
\begin{aligned}
    \left\| \tilde{\R}_i \right\| 
    \leq & \frac{a_{ss}}{\overline{\OO}_i} \sum_{f \in \partial \Omega_i}{
        A_{f} \tilde{\lambda}_{f} \min \left\{\|\UM_L\supsm\|, \|\UM_R\supsm\| \right\}
    } 
    + \dfrac{\left\|\uu_i\supsm\right\|}{\inc t^n}
    + \|\hat{\R}_i\|
    \\
    \leq & \left(
    \frac{1}{\inc t^n} + 
    \frac{a_{ss}\sum_{f \in \partial \Omega_i} A_f\tilde{\lambda}_f}{\overline{\OO}_i}
    \right)
    \|\UM_i\supsm\|
    + \|\hat{\R}_i\|\\
    = &
    \left(\frac{1}{\inc t^n} + \frac{a_{ss}\CFL_\tau}{\inc \tau_i} \right)\|\UM_i\supsm\| 
    + \|\hat{\R}_i\|\\
    = &
     \CFL_\tau\frac{1 / \CFL_i + a_{ss}}{\inc \tau_i}
     \|\UM_i\supsm\| 
    + \|\hat{\R}_i\|,
\end{aligned}
\end{equation}
with $\CFL_i$ being the CFL number corresponding to the limited
physical time step
$\inc t^n$.
Similar to \eqref{eq:inc_upper_cond_const_coef}, 
the residual value for each conservative variable can be 
bounded by the current values $\UM_i\supsm$ with additional 
intercept values $\left| \rho\left(\hat\R_i\right) \right|$, 
$\left | \rho E\left(\hat\R_i\right) \right|$, 
$\left\|\rho\uv\left(\hat\R_i\right) \right\|$. 
Non-singularity of the current state $\UM_i\supsm$ leads to 
\begin{equation}
    \begin{aligned}
    \label{eq:inc_upper_cond_const_coef_tau}
        \left| \rho\left( \inc\tau_i \tilde\R_i \right) \right|
        & \leq
        C^{\rho,\tau}_i \CFL_\tau(1 / \CFL_i + a_{ss})
        \rho\left( \UM_i\supsm \right)
        + \left| \rho\left(\inc\tau_i\hat\R_i\right) \right|,\\
        \left| \rho E\left(\inc\tau_i \tilde\R_i \right) \right|
        & \leq
        C^{\rho E,\tau}_i \CFL_\tau(1 / \CFL_i + a_{ss})
        \rho E\left( \UM_i\supsm \right)
        + \left | \rho E\left(\inc\tau_i\hat\R_i\right) \right|,\\
        \left\| \rho \uv\left( \inc\tau_i \tilde\R_i \right) \right\|
        & \leq
        C^{\rho \uv,\tau}_i \CFL_\tau(1 / \CFL_i + a_{ss})
        \left\| \rho \uv\left( \UM_i\supsm \right)              \right\| + \left\|\rho\uv\left(\inc\tau_i\hat\R_i\right) \right\|\\
    \end{aligned}
\end{equation}
with coefficients $C^{\rho ,\tau}_i$,$C^{\rho E,\tau}_i$ and $C^{\rho \uv,\tau}_i$
of order $\ord(1)$.
Making
\begin{equation}
    \begin{aligned}
    \label{eq:inc_upper_cond_const_coef_tau_recast_method}
        D^{\rho,\tau}_i
        & =
        C^{\rho,\tau}_i (1 / \CFL_i + a_{ss})
        + \frac{\left| \rho\left(\inc\tau_i \hat\R_i\right) \right|}
        { \CFL_\tau\rho\left( \UM_i\supsm \right)},
        \\
        D^{\rho E,\tau}_i
        & =
        C^{\rho E,\tau}_i (1 / \CFL_i + a_{ss})
        + \frac{\left| \rho E\left(\inc\tau_i \hat\R_i\right) \right|}
        { \CFL_\tau\rho E\left( \UM_i\supsm \right)},
        \\
        D^{\rho \uv,\tau}_i
        & =
        C^{\rho \uv,\tau}_i (1 / \CFL_i + a_{ss})
        + \frac{\left\|\rho\uv\left(\inc\tau_i \hat\R_i\right) \right\|}
        { \CFL_\tau\left\|\rho \uv\left( \UM_i\supsm \right)\right\|}
        ,\\
    \end{aligned}
\end{equation}
the inequalities \eqref{eq:inc_upper_cond_const_coef_tau} can be 
rewritten into 
\begin{equation}
    \begin{aligned}
    \label{eq:inc_upper_cond_const_coef_tau_recasted}
        \left| \rho\left( \inc\tau_i \tilde\R_i \right) \right|
        & \leq
        D^{\rho,\tau}_i \CFL_\tau
        \rho\left( \UM_i\supsm \right)
        ,\\
        \left| \rho E\left(\inc\tau_i \tilde\R_i \right) \right|
        & \leq
        D^{\rho E,\tau}_i \CFL_\tau
        \rho E\left( \UM_i\supsm \right)
        ,\\
        \left\| \rho \uv\left( \inc\tau_i \tilde\R_i \right) \right\|
        & \leq
        D^{\rho \uv,\tau}_i \CFL_\tau
        \left\| \rho \uv\left( \UM_i\supsm \right)              \right\| 
        ,\\
    \end{aligned}
\end{equation}
which is of the same form as \eqref{eq:inc_upper_cond_const_coef}.
The constant term  $\hat{\R}_i$ is
\begin{equation}
        \hat{\R}_i  = \sum_{q=1}^{s-1} a_{sq} \R_i^{\left(q\right)} +
    \dfrac{\uu_i^{n}}{\inc t^n} 
\end{equation}
and therefore
\begin{equation}
        \inc\tau_i\hat{\R}_i  = \inc\tau_i\sum_{q=1}^{s-1} a_{sq} \R_i^{\left(q\right)} +
    \dfrac{\CFL_\tau}{\CFL_i}\uu_i^{n} 
\end{equation}
where $\CFL_i$ is determined by the already limited $\inc t^n$.
Assuming the previous stage RHS terms are non-singular, while 
last step solution $\UM_i^n$ and 
latest current stage solution $\UM_i\supsm$ are non-singular, clearly the 
non-dimensional ratios 
$\frac{\left| \rho\left(\inc\tau_i \hat\R_i\right) \right|}
{ \CFL_\tau\rho\left( \UM_i\supsm \right)}$,
$\frac{\left| \rho E\left(\inc\tau_i \hat\R_i\right) \right|}
{ \CFL_\tau\rho E\left( \UM_i\supsm \right)}$ and 
$\frac{\left\|\rho\uv\left(\inc\tau_i \hat\R_i\right) \right\|}
{ \CFL_\tau\left\|\rho \uv\left( \UM_i\supsm \right)\right\|}$
are of order $\ord(1)$. 
Moreover, we also know that physical and pseudo-time CFL number $\CFL_i$, $\CFL_\tau$ and 
Runge-Kutta coefficient $a_{ii}$.
In summary, the new coefficients $D^{\rho,\tau}_i$, $D^{\rho E,\tau}_i$ 
and $D^{\rho \uv,\tau}_i$ defined in \eqref{eq:inc_upper_cond_const_coef_tau_recast_method}
are of order $\ord(1)$. 

As the inequalities \eqref{eq:inc_upper_cond_const_coef_tau_recasted}
are of the same form as \eqref{eq:inc_upper_cond_const_coef}, 
subsequent analysis remains the same as that for $\inc t$ analysis. 
It can be derived that similar to $\inc t$, the CFL number 
for $\inc \tau _i$ after limiting is still of order $\ord(1)$.
In other words,
under the same assumptions we used to analyze physical time step limiting,
the limited pseudo time step size $\inc \tau_{i,pp}$ is not infinitely small compared to $\inc \tau_i$.

% Assuming that $a_{ss}$ and $\CFL_i$ are $\ord(1)$, 
% while obviously constant term $\hat{\R}_i$ has $\ord(1)$ value, 
% a non-singular $\UM_i\supsm$ leads to finite constant coefficients in 
% \eqref{eq:inc_upper_cond_const_coef_tau}. 
% The only difference between \eqref{eq:inc_upper_cond_const_coef} and 
% \eqref{eq:inc_upper_cond_const_coef_tau} is that \eqref{eq:inc_upper_cond_const_coef_tau}
% has additional intercept values on the right hand side.

% The remainder of the analysis remains unchanged, as a linear approximation is still used to estimate the future state in the pseudo-time direction.
% The lower bounds of $\alpha_{\tau,i}^\rho$ and $\alpha_{\tau,i}^{p}$ are similar to \eqref{eq:alpha-t-rho-lb} and \eqref{eq:alpha-t-p-lb}.
% The additional constant intercept values in \eqref{eq:inc_upper_cond_const_coef_tau} will lead to additional constant values added to the denominators $\alpha_{\tau,i}^\rho$ and $\alpha_{\tau,i}^{p}$'s lower bounds. 
% However, it has be shown that any additional constants added to the denominators
% will be of $\ord(1)$ magnitude, which does not affect the qualitative result of 
% the lower bounds.
% It can be concluded that, the limited pseudo time step size $\inc \tau_{i,pp}$ is not infinitely small compared to $\inc \tau_i$.

\subsubsection{Analysis of increment correction}

A rigorous analysis of the lower bound of the relaxation factor $\alpha_{\inc,i}$ in the increment correction, computed according to \eqref{eq:alpha-inc}–\eqref{eq:alpha-inc-p}, is not feasible. 
This is because the increment of the intermediate state is obtained using LU-SGS approximation, and therefore does not represent the exact solution of the linearized system \eqref{eq:linearTauUpdate}. 
The exact increment calculated by LU-SGS relies on 
the ordering of cells, which is hard to analyze 
for a general case.
As a result, the discrepancy between the theoretical and actual increments cannot be accurately estimated. 
Nevertheless, since the increment is computed using a limited pseudo-time step size, it remains inherently bounded. 

Consider the increment defined in the linear system \eqref{eq:linearTauUpdate},
which can be rewritten to a more precise expression
\begin{equation}
    % \label{eq:pseudo-time-equation}
    \left(\sum_i\left(\frac{\eye_{c,i}}{\inc \tau^m_{i,pp}}\right) + \frac{\eye}{\inc t^n} -a_{ss}\partialderivative{\R }{\uu} \right) \inc \uu^{\left(s,m\right)}
    = \tilde{\R}^{\left(s,m\right)},
    \label{eq:linearTauUpdate_re}
\end{equation}
where $\eye_{c,i}$ is 
the identity matrix for cell $i$ and zero for others.
Then, the LU-SGS leads to the increment being
\begin{equation}
    \begin{aligned}
        \inc \uu^{\left(s,m\right)}
     = &
    \left(\sum_i\left(\frac{\eye_{c,i}}{\inc \tau^m_{i,pp}}\right) + \frac{\eye}{\inc t^n} -a_{ss}(D+U) \right)^{-1}
    \left(\sum_i\left(\frac{\eye_{c,i}}{\inc \tau^m_{i,pp}}\right) + \frac{\eye}{\inc t^n} -a_{ss}D \right)\\
    &
    \left(\sum_i\left(\frac{\eye_{c,i}}{\inc \tau^m_{i,pp}}\right) + \frac{\eye}{\inc t^n} -a_{ss}(D+L) \right)^{-1}
    \tilde{\R}^{\left(s,m\right)} \\
    = & (\hat{D}+\hat{U})^{-1}\hat{D}(\hat{D}+\hat{L})^{-1}
    \tilde{\R}^{\left(s,m\right)} , 
    \end{aligned}
    \label{eq:linearTauUpdate_re_lusgs}
\end{equation}
where $L$, $U$ and $D$ are strict block-lower, strict block-upper and block-diagonal 
parts of Jacobian $\partialderivative{\R }{\uu}$ and $\partialderivative{\R }{\uu} = L + D + U$.
The $\hat L$, $\hat U$ and $\hat D$ are strict block-lower, strict block upper and block-diagonal 
parts of the Jacobian $-\partialderivative{\tilde\R }{\uu}$ in \eqref{eq:linearTauUpdate_re}.
Taking partial derivative of $\inc\tau^m_{i,pp}$ on \eqref{eq:linearTauUpdate_re_lusgs} leads to
\begin{equation}
    \begin{aligned}
       \pdv{\inc \uu^{\left(s,m\right)}}{\inc\tau^m_{i,pp}}
    = & \frac{1}{\left(\inc\tau^m_{i,pp}\right)^2}
    (\hat{D}+\hat{U})^{-1}\eye_{c,i}(\hat{D}+\hat{U})^{-1}\hat{D}(\hat{D}+\hat{L})^{-1}
    \tilde{\R}^{\left(s,m\right)} \\
    - & \frac{1}{\left(\inc\tau^m_{i,pp}\right)^2}
    (\hat{D}+\hat{U})^{-1}\eye_{c,i}(\hat{D}+\hat{L})^{-1}
    \tilde{\R}^{\left(s,m\right)} \\
    + & \frac{1}{\left(\inc\tau^m_{i,pp}\right)^2}
    (\hat{D}+\hat{U})^{-1}\hat{D}(\hat{D}+\hat{L})^{-1}\eye_{c,i}(\hat{D}+\hat{L})^{-1}
    \tilde{\R}^{\left(s,m\right)}.
    \end{aligned}
    \label{eq:linearTauUpdate_re_lusgs_pdvtau}
\end{equation}
When $\inc\tau^m_{i,pp}\rightarrow 0^+$, clearly 
$\hat D $, $\hat D + \hat L$ and $\hat D + \hat U$ all 
approximate
$\sum_i\left(\frac{\eye_{c,i}}{\inc \tau^m_{i,pp}}\right) $,
and the inverse matrices  $(\hat D + \hat L)^{-1}$ and 
$(\hat D + \hat U)^{-1}$ approximate $\sum_i\inc\tau^m_{i,pp}\eye_{c,i}$,
therefore
\begin{equation}
    \begin{aligned}
       \lim_{\inc\tau^m_{j,pp}\rightarrow 0^+, \forall j}\pdv{\inc \uu^{\left(s,m\right)}}{\inc\tau^m_{i,pp}}
    = & \eye_{c,i}
    \tilde{\R}^{\left(s,m\right)}.
    \end{aligned}
    \label{eq:linearTauUpdate_re_lusgs_pdvtau_limit}
\end{equation}
This limit means that although LU-SGS does not 
produce precise solution of the linearized system, 
it is consistent with the continuous local pseudo time evolution. 
From \eqref{eq:linearTauUpdate_re_lusgs_pdvtau_limit},
there is 
\begin{equation}
    \inc \uu^{\left(s,m\right)}
    =
    \sum_i{\inc\tau^m_{i,pp}\eye_{c,i}} \tilde\R^{\left(s,m\right)}
    +
    \ord\left(\left(\inc\tau^m_{i,pp}\right)^2\right)
\end{equation}
and for each cell
\begin{equation}
\label{eq:linearTauUpdate_lusgs_expand_tau}
    \inc \uu^{\left(s,m\right)}_i
    =
    \inc\tau^m_{i,pp}
    \tilde\R^{\left(s,m\right)}_i
    +
    \ord\left(\left(\inc\tau^m_{i,pp}\right)^2\right)
\end{equation}

Considering the limit in \eqref{eq:linearTauUpdate_re_lusgs_pdvtau_limit} 
and \eqref{eq:linearTauUpdate_lusgs_expand_tau}
when $\inc\tau_i\rightarrow 0$,
the difference between LU-SGS increment and 
Taylor expansion term \eqref{eq:estimation-dtau} used in pseudo time step limiting
is of order $\ord(\inc\tau^2)$. 
Meanwhile, when we use $\eta_\inc > \eta_\tau$,
and $\eta_\inc - \eta_\tau \sim \ord(1)$,
the margin between the 
lower bound of 
increment correction and pseudo time limiting
would be 
\begin{equation}
\rho_{min,\tau} - \rho_{min,\inc}= 
\rho\left(\uu^{(m)}\right) (\eta_\inc - \eta_\tau),\ \ 
p_{min,\tau} - p_{min,\inc}= 
p\left(\uu^{(m)}\right) (\eta_\inc - \eta_\tau).
\end{equation}
When pseudo time step limiting or increment correction is triggered,
the current state should be close enough to zero density or 
pressure state by the order of $\eta\inc\tau\tilde\R$.
It would be appropriate to assume 
$p\sim\ord(\inc\tau), \rho\sim\ord(1)$ 
or $p\sim\ord(1), \rho\sim\ord(\inc\tau)$
or $p\sim\ord(\inc\tau), \rho\sim\ord(\inc\tau)$
is true.
As a result,
\begin{equation}
\rho_{min,\tau} - \rho_{min,\inc} \gtrsim \ord (\inc \tau),\ \ 
p_{min,\tau} - p_{min,\inc} \gtrsim \ord (\inc \tau),
\end{equation}
which indicates asymptotically
the lower bound for density or pressure in pseudo time step limiting 
is far larger than that of increment correction, 
with a margin large enough to tolerate the error between 
linear estimation and real LU-SGS increment expressed in \eqref{eq:linearTauUpdate_lusgs_expand_tau}.
This means after pseudo time limiting, 
the increment is already positivity-preserving 
in an asymptotic manner without the involvement of 
increment limiting.

In practical computations, we use a finite $\inc \tau$, which 
does not guarantee positivity-preserving without increment correction.
However, when the solution converges, the 
residual $\tilde\R$ and increment value $\inc\UM$ both tends to zero,
which does not induce pseudo time limiting or increment correction.
The increment correction is typically activated during the early part of the inner iterations and tends to become inactive in later inner iterations. 


\subsection{Applicability of the algorithm}
% Finally, we analyze the applicability of the proposed positivity-preserving algorithm. 
% This algorithm is applicable to general implicit finite volume methods on unstructured grids for compressible flow simulations. While there is no restriction on the spatial discretization, an implicit dual-time discretization is essential.
% This algorithm is applicable to a broad range of implicit finite volume methods on unstructured grids for compressible flow simulations. Although the spatial discretization can be arbitrary, the use of an implicit dual-time stepping approach is essential for its effective implementation.

% The proposed positivity-preserving algorithm is applicable to a broad range of implicit finite volume methods on unstructured grids for compressible flow simulations. As the scaling limiter is designed to preserve positivity of reconstruction polynomials given admissible cell averages, thus any polynomial reconstruction is compatible with the proposed positivity-preserving algorithm.
% The procedures to preserve positivity of cell averages are designed in the dual-time stepping framework. Therefore, the use of an implicit dual-time stepping approach is essential for applying the positivity-preserving algorithm. There is not restrictions on the time integration scheme, any implicit time integration scheme can be used in theory. However, we need to highlight that, the ESDIRK schemes are well-suited for the proposed positivity-preserving algorithm. Compared with other high-order implicit time integration methods, such as the singly-diagonally implicit Runge–Kutta (SDIRK) methods, the $L$-stable ESDIRK methods are more robust. Furthermore, the solution at next time level of ESDIRK is taken as the converged state of the final stage, which is well-suited for our positivity-preserving algorithm, in which the intermediate states are guaranteed admissible. 

% The proposed positivity-preserving algorithm is applicable to a wide range of implicit finite volume methods on unstructured grids for compressible flow simulations. Regarding the spatial discretization, the only requirement is that a polynomial reconstruction needs to be used, since the scaling limiter is designed to preserve the positivity of reconstruction polynomials given admissible cell averages.

% Regarding time discretization, positivity preservation for cell averages is achieved within the dual-time stepping framework, making the use of an implicit dual-time stepping approach essential. Although there are no theoretical restrictions on the choice of time integration schemes and any implicit method can be used in principle, we emphasize that ESDIRK schemes are particularly well-suited for the proposed algorithm. Compared with other high-order implicit methods, such as singly diagonally implicit Runge–Kutta (SDIRK) schemes, the ESDIRK methods which are $L$-stable, offer improved robustness. Furthermore, the solution at the next time level in ESDIRK methods is defined as the converged state of the final stage, which aligns naturally with our positivity-preserving algorithm, as admissibility is maintained throughout the intermediate stages.

The proposed positivity-preserving algorithm is applicable to a wide range of implicit finite volume methods on unstructured grids for compressible flow simulations. For spatial discretization, the only requirement is the use of a polynomial reconstruction, since the scaling limiter is specifically designed to preserve the positivity of reconstructed polynomials given admissible cell averages.

For time discretization, positivity preservation of cell averages is achieved within a dual-time stepping framework, which makes the use of an implicit dual-time stepping approach essential. Although there are no theoretical restrictions on the choice of time integration scheme and any implicit method can be used in principle, we emphasize that ESDIRK schemes are particularly well-suited to the proposed algorithm. Compared to other high-order implicit methods, such as singly diagonally implicit Runge–Kutta (SDIRK) schemes, ESDIRK methods that are $L$-stable offer improved robustness. Moreover, the solution at the next time level in ESDIRK methods corresponds to the converged state of the final stage. This naturally aligns with the design of our algorithm, as admissibility is preserved throughout all intermediate stages.



\endgroup
